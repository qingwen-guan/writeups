\documentclass[UTF8,titlepage]{ctexart}
\usepackage{hyperref}
\usepackage{amsmath}
\usepackage{amssymb}
\usepackage{mathrsfs}
\usepackage[dvipsnames, svgnames, x11names]{xcolor}  % 一般放得靠前
%\usepackage{xcolor}
%\usepackage{marginnote}
\usepackage{wasysym}
\usepackage{geometry}
\usepackage{ntheorem}
%\usepackage{amsthm}
\usepackage{enumerate}
\usepackage[shortlabels]{enumitem} %https://tex.stackexchange.com/questions/229540/customized-enumerate-items
\usepackage{forest}
\usepackage{blkarray,bigstrut}
%\usepackage{txfonts}
\usepackage{imakeidx}
\usepackage{verbatim}
\usepackage{relsize} % for \mathlarger


\usetikzlibrary{decorations.pathreplacing}

%\geometry{a4paper,left=2cm,right=4cm,top=2cm,bottom=1cm,marginparwidth=3cm}
%\geometry{a4paper,left=2cm,right=2cm,top=2cm,bottom=1cm}
\geometry{a4paper,centering,scale=0.8}

\newcommand{\vect}[1]{\boldsymbol{#1}} % Uncomment for BOLD vectors.

\theorembodyfont{\upshape}
\newtheorem{Definition}{定义}%[subsection]
\newtheorem{Theorem}[Definition]{定理}
\newtheorem{thm}[Definition]{定理}

\newtheorem{Remark}[Definition]{注意}
\newtheorem*{RemarkNon}[Definition]{注意}

\newtheorem{Lemma}[Definition]{引理}
\newtheorem{lemma}[Definition]{引理}
\newtheorem{Corollary}[Definition]{推论}
\newtheorem{Proposition}[Definition]{命题}
\newtheorem{prop}[Definition]{命题}
\newtheorem{Example}[Definition]{例子}


\newtheorem{Property}[Definition]{性质}
\newtheorem*{PropertyNon}[Definition]{性质}

\newtheorem{Note}[Definition]{说明}
\newtheorem*{NoteNon}[Definition]{说明}

\setenumerate[1]{itemsep=0pt,partopsep=0pt,parsep=\parskip,topsep=0pt}
\setitemize[1]{itemsep=0pt,partopsep=0pt,parsep=\parskip,topsep=0pt}

%\renewcommand\marginfont{%
%    \scriptsize
%}

%\setlength{\parskip}{0pt}
%  \setlength{\topskip}{0pt}

\newenvironment{tightcenter}{%
  \setlength\topsep{0pt}
  \setlength\parskip{0pt}
  \begin{center}
}{%
  \end{center}
}

\newenvironment{tighteq*}{%
  \begin{tightcenter}
  $ \displaystyle 
}{%
  $
  \end{tightcenter}
}

\DeclareMathOperator{\rank}{rank}
\newcommand*\norm[1]{\lvert#1\rvert}
\newcommand\alphas[2][1]{\alpha_{#1}, \cdots, \alpha_{#2}}

\makeindex

\begin{document}

\title{高等代数笔记}
\author{管清文}
\maketitle
\tableofcontents
\clearpage

% http://blog.sina.com.cn/s/blog_4b3651ad0101nk4o.html
\begin{PropertyNon}[Property]
结果值得一记, 但是没有定理深刻.
\end{PropertyNon}

\begin{RemarkNon}[Remark]
涉及到一些结论,更像是非正式的定理.
\end{RemarkNon}

\begin{NoteNon}[Note]
就是注解.
\end{NoteNon}

\setcounter{Definition}{0}

%\section{线性方程组解法}

%\subsection{解线性方程组的矩阵消元法}

\begin{Note}
线性代数的主线是: 研究线性空间和线性映射.
\end{Note}

\section{$n$阶行列式}

\begin{Proposition}
$n$阶上三角形行列式的值,等于它的主对角线上的$n$个元素的乘积.
\end{Proposition}

\begin{Property}
$\mathrm{det}A = \mathrm{det}A'$
\end{Property}

\begin{Property}
行列式一行的公因子可以提出去, 即
$$
\left| \begin{matrix}
a_{11} & \cdots & a_{1n} \\
\vdots & \vdots & \vdots \\
ka_{i1} &\cdots & k a_{in} \\
\vdots & \vdots & \vdots \\
a_{n1} & \cdots & a_{nn} \\
\end{matrix} \right| = k \left|
\begin{matrix}
a_{11} & \cdots & a_{1n} \\
\vdots & \vdots & \vdots \\
a_{i1} & \cdots & a_{in} \\
\vdots & \vdots & \vdots \\
a_{n1} & \cdots & a_{nn} \\
\end{matrix} \right| (k \in K)$$
\end{Property}

\begin{Property}
$$
\begin{blockarray}{c|ccc|}
&a_{11} & \cdots & a_{1n} \\
&\vdots & \vdots & \vdots \\
\text{(row $i$)} &b_{1} + c_1 & \cdots & b_{n} +c_{n} \\
&\vdots & \vdots & \vdots \\
&a_{n1} & \cdots & a_{nn} \\
\end{blockarray}
= 
\begin{blockarray}{|ccc|}
a_{11} & \cdots & a_{1n} \\
\vdots & \vdots & \vdots \\
b_{1} & \cdots & b_{n} \\
\vdots & \vdots & \vdots \\
a_{n1} & \cdots & a_{nn} \\
\end{blockarray}
+
\begin{blockarray}{|ccc|}
a_{11} & \cdots & a_{1n} \\
\vdots & \vdots & \vdots \\
c_{1} & \cdots & c_{n} \\
\vdots & \vdots & \vdots \\
a_{n1} & \cdots & a_{nn} \\
\end{blockarray}
$$
\end{Property}

\begin{Property}
两行互换, 行列式反号.
$$
\begin{blockarray}{c|ccc|}
&a_{11} & \cdots & a_{1n} \\
&\vdots & \vdots & \vdots \\
\text{(row $i$)} &a_{k1} & \cdots & a_{kn} \\
&\vdots & \vdots & \vdots \\
\text{(row $k$)} &a_{j1} & \cdots & a_{jn} \\
&\vdots & \vdots & \vdots \\
&a_{n1} & \cdots & a_{nn} \\
\end{blockarray}
= 
-\;
\begin{blockarray}{|ccc|c}
a_{11} & \cdots & a_{1n} &\\
\vdots & \vdots & \vdots &\\
a_{i1} & \cdots & a_{in} & \text{(row $i$)}\\
\vdots & \vdots & \vdots &\\
a_{k1} & \cdots & a_{kn} & \text{(row $k$)}\\
\vdots & \vdots & \vdots &\\
a_{n1} & \cdots & a_{nn} &\\
\end{blockarray}
$$
\end{Property}

\begin{Property}
两行成比例, 行列式值为$0$.
$$
\begin{blockarray}{c|ccc|}
&a_{11} & \cdots & a_{1n} \\
&\vdots & \vdots & \vdots \\
\text{(row $i$)} &a_{i1} & \cdots & a_{in} \\
&\vdots & \vdots & \vdots \\
\text{(row $k$)} &k a_{i1} & \cdots & k a_{in} \\
&\vdots & \vdots & \vdots \\
&a_{n1} & \cdots & a_{nn} \\
\end{blockarray}
= 
0
$$
\end{Property}


\begin{Property}
把一行的倍数加到另一行上,行列式值不变.
$$
\begin{blockarray}{c|ccc|}
&a_{11} & \cdots & a_{1n} \\
&\vdots & \vdots & \vdots \\
\text{(row $i$)} &a_{i1} & \cdots & a_{in} \\
&\vdots & \vdots & \vdots \\
\text{(row $k$)} & a_{k1} +\ell a_{i1} & \cdots & a_{kn} +\ell a_{in} \\
&\vdots & \vdots & \vdots \\
&a_{n1} & \cdots & a_{nn} \\
\end{blockarray}
= 
\begin{blockarray}{|ccc|c}
a_{11} & \cdots & a_{1n} &\\
\vdots & \vdots & \vdots &\\
a_{i1} & \cdots & a_{in} & \text{(row $i$)}\\
\vdots & \vdots & \vdots &\\
a_{k1} & \cdots & a_{kn} & \text{(row $k$)}\\
\vdots & \vdots & \vdots &\\
a_{n1} & \cdots & a_{nn} &\\
\end{blockarray}
$$
\end{Property}

\begin{Proposition}
$$
\begin{vmatrix}
k       & \lambda & \lambda & \cdots & \lambda \\
\lambda & k       & \lambda & \cdots & \lambda \\
\vdots  & \vdots  & \vdots  & \vdots & \vdots  \\
\lambda & \lambda & \lambda & \cdots & k       \\
\end{vmatrix} = \left[ k + (n-1) \lambda \right] \left( k - \lambda \right)^{n-1}
$$
\end{Proposition}

\begin{Definition}
$n$阶矩阵$A = (a_{ij})$, 划去A的第$i$行第$j$列剩下元素按原来顺序排列构成余子式$M_{ij}$, 代数余子式$A_{ij} = {(-1)}^{i+j} M_{ij}$.
\end{Definition}

\begin{Theorem}
$n$阶行列式$|A|$等于它的第$i$行元素与自己的代数余子式的乘积之和, 即
$$
\mathrm{det} A = \sum_{j=1}^n a_{ij} A_{ij}.
$$
\end{Theorem}


\begin{Theorem}
$n$阶行列式$|A|$等于它的第$j$列元素与自己的代数余子式的乘积之和, 即
$$
\mathrm{det} A = \sum_{i=1}^n a_{ij} A_{ij}.
$$
\end{Theorem}

\begin{Theorem}
$n$阶矩阵$A = (a_{ij})$, $i \neq k$时$a_{i1} A_{k1} + a_{i2} A_{k2} + \cdots + a_{in} A_{kn} = 0$.
\end{Theorem}

\begin{Theorem}
$n$阶矩阵$A = (a_{ij})$, $j \neq \ell$时$a_{1j} A_{1\ell} + a_{2j} A_{2\ell} + \cdots + a_{nj} A_{n\ell} = 0$.
\end{Theorem}

\begin{Proposition} $n$阶范德蒙行列式($n \ge 2$)
$$
\begin{vmatrix} 
1      & 1      & \cdots & 1 \\
a_1    & a_2    & \cdots & a_n \\
\vdots & \vdots & \vdots & \vdots \\
a_1^n  & a_2^n  & \cdots & a_n^n \\
\end{vmatrix} = \prod_{1 \le j < i \le n} {(a_i - a_j)}
$$
\end{Proposition}

\begin{Theorem}
数域$K$上的$n$个方程的$n$元线性方程组有唯一解的充要条件是他的系数行列式不等于$0$.
\end{Theorem}

\begin{Corollary}
数域$K$上的$n$个方程的$n$元齐次线性方程组只有零解的充要条件是它的系数行列式不等于0; 它有非零解的充分必要条件是它的系数行列式等于0.
\end{Corollary}

\begin{Note}
克莱姆法则等到第四章再给.
\end{Note}

\begin{Note}
数学上凡是重要的概念, 虽然可能是因为研究某一类内容提出的, 但是它一经提出来就不仅仅是解决这个问题了, 它可能解决高等代数里很多其他特定的问题, 而且还能解决几何学、分析学的好多问题.
\end{Note}

\begin{Definition}
$n$阶行列式$|A|$中任意取定$k$行$k$列($1 \le k < n$), 位于这些行和列的交叉处的$k^2$个元素按原来的排法组成的$k$阶行列式,
称为$|A|$的一个\textbf{$k$阶子式}\index{$k$阶子式}. 取定$|A|$的第$i_1, i_2, \cdots i_k$行($i_1 < i_2 < \cdots < i_k$), 第$j_1, j_2, \cdots, j_k$列($j_1 < j_2 < \cdots < j_k$), 所得到的$k$阶子式记作
\begin{equation} \label{eq:1}
A \begin{pmatrix}
i_1, i_2, \cdots, i_k \\
j_1, j_2, \cdots, j_k \\
\end{pmatrix}
\end{equation}
划去这个$k$阶子式所在的行和列, 剩下的预算按照原来的排法组成的$(n-k)$阶行列式, 称为子式(\ref{eq:1})的\textbf{余子式},
令
$$
\begin{aligned}
\{ i'_1, i'_2, \cdots, i'_{n-k} \} &= \{ 1, 2, \cdots, n \} - \{ i_1, i_2, \cdots, i_k \} \\
\{ j'_1, j'_2, \cdots, j'_{n-k} \} &= \{ 1, 2, \cdots, n \} - \{ j_1, j_2, \cdots, j_k \} \\
\end{aligned}
$$
其中$i'_1 < i'_2 < \cdots < i'_{n-k}, j'_1 < j'_2 < \cdots < j'_{n-k}$, 则子式(\ref{eq:1})的余子式为
$$
A \begin{pmatrix}
i'_1, i'_2, \cdots, i'_{n-k} \\
j'_1, j'_2, \cdots, j'_{n-k} \\
\end{pmatrix}
$$
它的前面乘以 
$${(-1)}^{(i_1 + i_2 + \cdots + i_k ) + (j_1 +j_2 + \cdots + j_k)}$$
则称为子式(\ref{eq:1})的\textbf{代数余子式}.
\end{Definition}

\begin{Theorem}[Laplace定理]
在$n$阶行列式$|A| = |a_{ij}|$中, 取定第$i_1, i_2, \cdots, i_k$行($i_1 < i_2 < \cdots < i_k$), 则
这$k$行元素形成的所有$k$阶子式与它们自己的代数余子式的乘积之和等于$|A|$, 即
$$
|A| = \sum_{1 \le j_1 < j_2 < \cdots < j_k \le n} A\begin{pmatrix}
i_1, i_2, \cdots, i_k \\
j_1, j_2, \cdots, j_k \\
\end{pmatrix} (-1)^{(i_1 + i_2 + \cdots + i_k) + (j_1 + j_2 + \cdots + j_k)} A \begin{pmatrix}
i'_1, i'_2, \cdots, i'_{n-k} \\
j'_1, j'_2, \cdots, j'_{n-k} \\
\end{pmatrix}
$$
\end{Theorem}

\begin{Corollary}
$$
\begin{vmatrix}
a_{11} & \cdots & a_{1k} & 0      & \cdots & 0 \\
\vdots &        & \vdots & \vdots &        & \vdots \\
a_{k1} & \cdots & a_{kk} & 0      & \cdots & 0 \\
c_{11} & \cdots & c_{1k} & d_{11} & \cdots & d_{1r} \\
\vdots &        & \vdots & \vdots &        & \vdots \\
c_{r1} & \cdots & c_{rk} & d_{r1} & \cdots & d_{rr} 
\end{vmatrix} = \begin{vmatrix}
a_{11} & \cdots & a_{1k}  \\
\vdots &        & \vdots  \\
a_{k1} & \cdots & a_{kk} \\
\end{vmatrix} \cdot
\begin{vmatrix}
d_{11} & \cdots & d_{1r}  \\
\vdots &        & \vdots  \\
d_{r1} & \cdots & d_{rr} \\
\end{vmatrix}
$$
也记成
$$
\begin{vmatrix}
A_{k \times k} & 0 \\
C & D_{r \times r} \\
\end{vmatrix} = |A| \cdot |D|
$$
\end{Corollary}

\section{线性空间}
\subsection{域上的线性空间的定义}

\begin{Definition}[线性空间]
一个\;\fbox{非空}\;集合$V$, 如果它有加法运算(即$V \times V \rightarrow V$: $(\alpha, \beta) \mapsto \alpha + \beta$), 其元素与域$F$的元素之间有纯量乘法运算
(即$F \times V \rightarrow V$: $k \times \alpha \mapsto k \alpha$), 并且满足下面的$8$条运算法则($\forall \alpha, \beta, \gamma \in V, 
\forall k, \ell \in F$),
\begin{enumerate}[(1)]
	\item 加法交换: $\alpha + \beta = \beta + \alpha$
	\item II(加法结合): $(\alpha + \beta) + \gamma = \alpha + (\beta + \gamma)$
	\item IV: $\exists \mathbf{0} \in V$使得$\alpha + \mathbf{0} = \alpha$, 具有该性质的元素$\mathbf{0}$称为$V$的\textbf{零元}
	\item V: $\forall \alpha \in V, \exists \beta \in V$使得$\alpha + \beta = \mathbf{0}$, 具有该性质的元素$b$称为$a$的\textbf{负元}
	\item $\mathbf{1} \alpha = \alpha$, 其中$\mathbf{1}$是$F$的单位元.
	\item $(k \ell) \alpha = k (\ell \alpha)$
	\item $(k+\ell) \alpha = k \alpha + \ell \alpha$
	\item $k(\alpha + \beta) = k \alpha + k\beta$
\end{enumerate}
则称$V$是域$F$上的一个\textbf{线性空间}.
\end{Definition}

\begin{Note}
线性空间$V$对于加法构成一个群.
\end{Note}

\begin{Property}
设$V$是域$K$上的一个线性空间,则它满足以下性质($\forall k \in F, \forall \alpha \in V$):
\begin{enumerate}[(1)]
\item $V$的零元唯一
\item $\alpha$的负元唯一
\item $0 \alpha = \mathbf{0}$
\item $k \mathbf{0} = \mathbf{0}$
\item $k \alpha = 0 \Rightarrow k = 0$或$\alpha = \mathbf{0}$
\item $(-1) \alpha = - \alpha$
\end{enumerate}
\end{Property}

\subsection{线性子空间}

\begin{Definition}[线性子空间]
设$V$是域$F$上的一个线性空间, $U$是$V$的一个\;\fbox{非空}\;子集, 如果$U$对于$V$的加法和纯量乘法, 也构成域$F$上的一个线性空间, 则称$U$是$V$的一个\textbf{线性子空间}, 简称\textbf{子空间}.
\end{Definition}

\begin{Theorem}[!!!]
设$U$是域$F$上的线性空间$V$的一个非空子集, 则$U$是$V$的一个子空间的充分必要条件是: $U$对于$V$的加法和纯量乘法都封闭, 即
$$
\begin{aligned}
\alpha, \beta \in U \Rightarrow \alpha + \beta \in U \\
k \in F, \alpha \in U \Rightarrow k \alpha \in U
\end{aligned}
$$
\end{Theorem}

\begin{Note}
$\{ 0 \}, V$都是$V$的子空间,称为$V$的\textbf{平凡子空间}, $\{ 0 \}$称为\textbf{零子空间}, 可记作$\mathfrak{0}$.
\end{Note}

\begin{Definition}
对于$V$中的一组向量$\alpha_1, \alpha_2, \cdots, \alpha_s$和域$F$中的一组元素$k_1, k_2, \cdots, k_s$, 作纯量乘法和加法得到
$$
k_1 \alpha_1 + k_2 \alpha_2 + \cdots +k_s \alpha_s
$$
仍然是$V$中的一个向量, 称这个向量是$\alpha_1, \alpha_2, \cdots, \alpha_s$的一个\textbf{线性组合}.
像$\alpha_1, \alpha_2, \cdots, \alpha_s$这样按照一定顺序写出的\;\fbox{有限}\;多个向量(允许有相同的向量)称为$V$的一个\textbf{向量组}.
\end{Definition}

\begin{Definition}[线性表出]
$V$中的一个向量$\beta$如果能够表示成向量组$\alpha_1, \alpha_2, \cdots, \alpha_s$的一个线性组合, 那么称$\beta$可由向量组$\alpha_1, \alpha_2, \cdots, \alpha_s$\textbf{线性表出}.
\end{Definition}

\begin{Definition}[由向量组生成的线性子空间]
设$V$是域$F$上的一个线性空间, 给定$V$的一个向量组$\alpha_1, \alpha_2, \cdots, \alpha_s$. 考虑$\alpha_1, \alpha_2, \cdots, \alpha_s$的所有线性组合构成的集合
$$
W = \{ k_1 \alpha_1 + k_2 \alpha_2 + \cdots + k_s \alpha_s \mid k_i \in F, i = 1, \cdots, s\}
$$
则$W$是$V$的一个子空间. 把$W$称为向量组\textbf{生成}(或\textbf{张成})\textbf{的线性子空间},
记作$\langle \alpha_1, \alpha_2, \cdots, \alpha_s \rangle$或者$L(\alpha_1, \alpha_2, \cdots, \alpha_s)$.
\end{Definition}

%\subsection{线性方程组的列视角}

\begin{Note}
对于数域$K$上的$n$元线性方程组
$$
\left\{
\begin{matrix}
a_{11} x_1 + \cdots + a_{1n} x_n = b_1 \\
\vdots \\
a_{s1} x_1 + \cdots + a_{sn} x_n = b_n \\
\end{matrix}
\right.
$$%
可以写成
$$
x_1 \begin{pmatrix} a_{11} \\ \vdots \\ a_{s1} \end{pmatrix}
+\cdots
+x_n \begin{pmatrix} a_{1n} \\ \vdots \\ a_{sn} \end{pmatrix}
= \begin{pmatrix} b_{1} \\ b_{2} \\ \vdots \\ b_{n} \end{pmatrix}
$$
即
$$
x_1 \mathbf{\alpha_1} +
\cdots + 
x_n \mathbf{\alpha_n} = 
\mathbf{\beta}
$$
其中
$\mathbf{\alpha_1}, \mathbf{\alpha_2}, \cdots, \mathbf{\alpha_n}$
是原线性方程组的系数矩阵的列向量组; $\beta$是由常数项组成的列向量.
\end{Note}

\begin{Note} \ \\
$$
\begin{aligned}
& \text{数域$K$上线性方程组\;} x_1 \mathbf{\alpha_1} +
\cdots + 
x_n \mathbf{\alpha_n} = 
\mathbf{\beta}\text{\;有解} \\
\Leftrightarrow\;
& \beta \text{\;可以由\;} \mathbf{\alpha_1}, \cdots, \alpha_n \text{\;线性表出} \\
\Leftrightarrow\; & \beta \in \langle \alpha_1, \cdots, \alpha_n \rangle 
\end{aligned}
$$
\end{Note}

\begin{Note}于是下一步的任务: 研究线性空间和它的子空间的结构.
\end{Note}

\subsection{线性相关与线性无关的向量组}

\begin{Definition}[线性相关]
设$V$是数域$K$上的一个线性空间, 对于$V$中的一个向量组$\mathbf{\alpha_1}, \mathbf{\alpha_2}, \cdots, \mathbf{\alpha_s}$ ($s \ge 0$)
如果有$K$中不全为$0$的数$k_1, k_2, \cdots, k_s$, 使得
$$
k_1 \alpha_1 + k_2 \alpha_2 + \cdots + k_s \alpha_s = 0
$$
被称向量组$\mathbf{\alpha_1}, \mathbf{\alpha_2}, \cdots, \mathbf{\alpha_s}$\textbf{线性相关} 
; 否则(也就是$k_1 \mathbf{\alpha}_1 + k_2 \mathbf{\alpha}_2 + \cdots + k_s \mathbf{\alpha}_s = 0 \Rightarrow k_1 = k_2 = \cdots = k_s = 0$), 称$\alpha_1, \alpha_2, \cdots, \alpha_s$\textbf{线性无关}.
\end{Definition}

\begin{Note}[!]
$K^s$中, 列向量组$\alpha_1, \cdots, \alpha_n$线性相关 \\
$\Leftrightarrow$ $K$上的$n$元齐次线性方程组$x_1 \alpha_1 + \cdots +x_n \alpha_n = 0$有非零解
\end{Note}

\begin{Note}[!]
$K^s$中, 列向量组$\alpha_1, \cdots, \alpha_n$线性无关 \\
$\Leftrightarrow$ $K$上的$n$元齐次线性方程组$x_1 \alpha_1  + \cdots +x_n \alpha_n = 0$只有零解
\end{Note}

\begin{Note}
$K^{n}$中, 列向量组$\alpha_1, \cdots, \alpha_n$线性相关 \\
$\Leftrightarrow$ $K$上的$n$元齐次线性方程组$x_1 \alpha_1 + \cdots +x_n \alpha_n = 0$有非零解 \\
$\Leftrightarrow$ 以$\alpha_1, \cdots, \alpha_n$为列向量组的矩阵$A$的行列式$=0$.
\end{Note}

\begin{Note}[!!!!]
$K^{n}$中, 
\begin{tightcenter}
列向量组$\alpha_1, \cdots, \alpha_n$线性无关
$\Leftrightarrow$ 以$\alpha_1, \cdots, \alpha_n$为列向量组的矩阵$A$的行列式$\neq 0$.
\end{tightcenter}
\end{Note}

\begin{Note}
设$V$是数域$K$上的一个线性空间, 
\begin{enumerate}[(1)]
\item $\alpha$线性相关$\Leftrightarrow$有$k \neq 0$使得$k \alpha = 0$ 
$\Leftrightarrow \alpha = \mathbf{0}$
\item 从而, $\alpha$线性无关 $\Leftrightarrow \alpha \neq \mathbf{0}$
\item[! (3)] 向量组$\alpha_1, \alpha_2, \cdots, \alpha_s$如果有一个部分组线性相关, 那么向量组$\alpha_1, \alpha_2, \cdots, \alpha_s$线性相关.
\item[(4)] 向量组$\alpha_1, \alpha_2, \cdots, \alpha_s$如果线性无关, 那么$\alpha_1, \alpha_2, \cdots, \alpha_s$任何一个部分组都线性无关.
\item[(5)] 含有$\mathbf{0}$的任何一个向量组都线性相关.
\item[(6)] 向量组$\alpha_1, \alpha_2, \cdots, \alpha_s$ ($s \ge 2$)线性相关 $\Leftrightarrow$ 其中至少有一个向量可以由其余的向量线性表出.
\item[(7)] 向量组$\alpha_1, \alpha_2, \cdots, \alpha_s$ ($s \ge 2$)线性无关 $\Leftrightarrow$ 其中每一个向量都不能由其余的向量线性表出.
\item[!!!! (8)] 如果向量组线性无关, 那么把每个向量填上$m$个分量(所填分量的位置对于每个向量都一样)得到的\textbf{延伸组}也线性无关.
\item[! (9)] 如果向量组线性相关, 那么把每个向量去掉$m$个分量(去掉的分量的位置对于每个向量都一样)得到的\textbf{缩短组}也线性相关.
\end{enumerate}
\end{Note}

\begin{Proposition}[!!]
设$\beta$可以由向量组$\alpha_1, \alpha_2, \cdots, \alpha_s$线性表出, 则
\begin{tightcenter}
表出方式唯一$\Leftrightarrow$
$\alpha_1, \alpha_2, \cdots, \alpha_s$线性无关. 
\end{tightcenter}
\end{Proposition}

\begin{Proposition}[!!!]
设向量组$\alpha_1, \alpha_2, \cdots, \alpha_s$线性无关, 如果
$\alpha_1, \alpha_2, \cdots, \alpha_s, \beta$线性相关, 则$\beta$可以由$\alpha_1, \alpha_2, \cdots, \alpha_s$线性表出.
\end{Proposition}

\subsection{极大线性无关组, 向量组的秩}

\begin{Definition}[极大线性无关组]
向量组的一个部分组如果满足以下两个条件
\begin{enumerate}[(1)]
\item 这个部分组本身是线性无关的
\item 从这个向量组的其余向量(如果还有的话)中任取一个添进去, 得到的新的部分组都线性相关.
\end{enumerate}
那么称这个部分组是向量组的一个\textbf{极大线性无关组}.
\end{Definition}

\begin{Definition}
若向量组$\alpha_1, \cdots, \alpha_s$中每一个向量都可以由向量组$\beta_1, \cdots, \beta_r$线性表出, 则称
向量组$\alpha_1, \cdots, \alpha_s$可以由向量组$\beta_1,\cdots, \beta_r$线性表出. 如果向量组$\alpha_1, \cdots, \alpha_s$
和向量组$\beta_1, \cdots, \beta_r$可以互相线性表出, 那么称向量组$\alpha_1, \cdots, \alpha_s$与向量组$\beta_1, \cdots, \beta_r$\textbf{等价}, 记作
$$\{ \alpha_1, \cdots, \alpha_s \} \cong \{ \beta_1, \cdots, \beta_r \}$$

\end{Definition}

\begin{Remark}
$\{ \mathbf{0} \}$不能由空集线性表出.
\end{Remark}

\begin{Proposition}[!]
一个向量组和它的任意一个极大线性无关组等价.
\end{Proposition}

\begin{Corollary}
向量组的任意两个极大线性无关组等价.
\end{Corollary}

%\begin{Corollary}
%$\beta$可以由向量组$\alpha_1, \cdots, \alpha_s$线性表出 $\Leftrightarrow$ $\beta$可以由$\alpha_1, \cdots, \alpha_s$的一个极大线性无关组线性表出.
%\end{Corollary}

\begin{Lemma}[!!]
设向量组$\beta_1, \cdots, \beta_r$可以由向量组$\alpha_1, \cdots, \alpha_s$线性表出, 
\begin{tightcenter}
$r > s \Rightarrow \beta_1, \cdots, \beta_r$线性相关.
\end{tightcenter}
\end{Lemma}

\begin{Corollary}
设向量组$\beta_1, \cdots, \beta_r$可以由向量组$\alpha_1, \cdots, \alpha_s$线性表出, 
\begin{tightcenter}
$\beta_1, \cdots, \beta_r$线性无关$\Rightarrow r \le s$.
\end{tightcenter}
\end{Corollary}

\begin{Corollary}
等价的线性无关的向量组的个数相等.
\end{Corollary}

\begin{Corollary}
向量组的任意两个极大线性无关组所含向量的个数相等.
\end{Corollary}

\begin{Definition}[!!!!!, 向量组的秩]
向量组$\alpha_1, \cdots, \alpha_s$的极大线性无关组所含向量的个数称为向量组$\alpha_1, \cdots, \alpha_s$的\textbf{秩}, 记作
$$ \rank \{ \alpha_1, \cdots, \alpha_s \}  $$
全由零向量租组成向量组的秩规定为$0$.
\end{Definition}

\begin{Proposition}
向量组的任意一个线性无关组都可以扩充成一个极大线性无关组.
\end{Proposition}

\begin{Proposition}[!!!]
向量组$\alpha_1, \cdots, \alpha_s$线性无关 $\Leftrightarrow $ $\text{rank} \{ \alpha_1, \cdots, \alpha_s\} = s$. 
\end{Proposition}

\begin{Proposition}[!!]
向量组(I)可以由向量组(II)线性表出, 则$\text{rank(I)} \le \text{rank(II)}$.
\end{Proposition}

\begin{Corollary}[!!]
等价的向量组的秩相等.
\end{Corollary}

\subsection{基与维数、坐标}

\begin{Definition}[!]
设$V$是数域$K$上的线性空间, 
\begin{tightcenter}
$V$的一个有限多重子集$[[ \alpha_1, \cdots, \alpha_s ]]$线性相关 $\stackrel{\text{def}}{\iff}$ 向量组$\alpha_1, \cdots, \alpha_s$线性相关. \\
$V$的一个有限多重子集$[[ \alpha_1, \cdots, \alpha_s ]]$线性无关 $\stackrel{\text{def}}{\iff}$ 向量组$\alpha_1, \cdots, \alpha_s$线性无关. \\
$V$的一个无限多重子集$\mathfrak{S}$线性相关 $\stackrel{\text{def}}{\iff}$ $\mathfrak{S}$有一个有限子集是线性相关的. \\
$V$的一个无限多重子集$\mathfrak{S}$线性无关 $\stackrel{\text{def}}{\iff}$ $\mathfrak{S}$任何一个有限子集都线性无关. \\
$\mathfrak{S} = \emptyset$ $\stackrel{\text{def}}{\iff}$ $\mathfrak{S}$是线性无关的.
\end{tightcenter}
\end{Definition}

\begin{Definition}[!!, 基] \label{def:base}
设$V$是数域$K$上的线性空间, $V$的一个多重子集$\mathfrak{S}$如果满足下述两个条件
\begin{enumerate}[(1)]
	\item $\mathfrak{S}$是线性无关的
	\item $V$中任一向量可以由$\mathfrak{S}$中的\;\fbox{有限多个向量}\;线性表出
\end{enumerate}
则称$\mathfrak{S}$是$V$中的\;\fbox{一个}\;\textbf{基}. $\{ 0 \}$的一个基\;\fbox{规定}\;为$\emptyset$.
\end{Definition}

\begin{Definition}[有序基]
在~\textbf{定义\ref{def:base}}中, 若$S = \{ \alpha_1, \cdots, \alpha_s \}$, 
则向量组 $\alpha_1, \cdots, \alpha_s$
是$V$中的一个\textbf{(有序)基}.
\end{Definition}


\begin{Theorem}[!依赖选择公理]
任何一个数域上的任一线性空间都有一个基.
\end{Theorem}

\begin{Definition}
若$V$有一个基是有限子集, 则称$V$是\textbf{有限维}的; 若$V$有一个基是无限子集, 则称$V$是\textbf{无限维}的. 
\end{Definition}

\begin{Theorem}
若$V$是有限维的, 则$V$的任意两个基所含向量的个数相等.
\end{Theorem}

\begin{Corollary}
若$V$是无限维的, 则$V$的任何一个基都是无限子集.
\end{Corollary}

\begin{Definition}[!!!, 维数]
设$V$是有限维的, 则把$V$的一个基所含向量的个数称为$V$的\textbf{维数}, 记作${\dim}_K V$或$\dim V$;
若$V$是无限维的,则把$V$的维数 $\dim = \infty$ (只是暂不区分无限可数和无限不可数).%; $\{ 0 \}$的维数为$0$.
\end{Definition}

\begin{Proposition}[!]
设$\text{dim}V = n$, 则$V$中任意$n+1$个向量都线性相关.
\end{Proposition}

\begin{Definition}[!, 坐标]
设$\text{dim}V = n$, 取$V$的一个(有序)基$\alphas{n}$.
则$V$中任一向量可以由$\alpha_1, \cdots, \alpha_n$表出,$\alpha = c_1 \alpha_1 + \cdots + c_n \alpha_n$且表出方式唯一. 
\begin{tightcenter}
把$\begin{pmatrix} c_1 \\ \vdots \\ c_n \end{pmatrix}$称为$\alpha$在(有序)基$\alphas{n}$下的\textbf{坐标} \index{坐标}.
\end{tightcenter}
\end{Definition}

\begin{Remark}
坐标要写成列向量.
\end{Remark}

\begin{Example}
$K^n$中,向量组
$$
\epsilon_1 = \begin{pmatrix} 1 \\ \vdots \\ 0 \end{pmatrix},
\cdots,
\epsilon_n = \begin{pmatrix} 0 \\ \vdots \\ 1 \end{pmatrix}
$$
是$K^n$的一个基, 称为\textbf{标准基}.
 $K^n$中任一向量 $\displaystyle \alpha = \begin{pmatrix} c_1 \\ \vdots \\ c_n \end{pmatrix} $在基$\epsilon_1, \cdots, \epsilon_n$下的坐标是
$
\begin{pmatrix} c_1 \\ \vdots \\ c_n
\end{pmatrix} = \alpha
$
\end{Example}

\begin{Proposition}[!!]
设$\dim V = n$, 则$V$中任意$n$个线性无关的向量都是$V$的一个基.
\end{Proposition}

\begin{Proposition}[!]
设$\dim V = n$, 则
\begin{tightcenter}
$V$中的每一个向量可以由向量组$\alphas{n}$线性表出 $\implies$ $\alphas{n}$是$V$的一个基.
\end{tightcenter}
\end{Proposition}

\begin{Proposition}
设$\dim V = n$, 则
$V$中任意一个线性无关的向量组都可以扩充成$V$的一个基.
\end{Proposition}

\begin{Proposition}[!!]
设$\dim V = n$, $W$是$V$的一个子空间, 则$\dim W \le \dim V$; 若$\dim W = \dim V$, 则$W = V$.
\end{Proposition}

\begin{Definition}[极大线性无关集]
设$V$是数域$K$上的线性空间, $V$的一个子集$S$如果满足,
\begin{enumerate}[(1)]
\item $S$是线性无关的
\item 对于$\beta \notin S$(如果存在的话), 有$S \cup \{ \beta \}$线性相关
\end{enumerate}
那么称$S$是$V$的\textbf{极大线性无关集}.
\end{Definition}

\begin{Proposition}
$S$是$V$的一个基 $\iff$ $S$是$V$的一个极大线性无关集.
\end{Proposition}

\begin{Proposition}[!!]
在数域$K$中, 向量组$\alphas{s}$的一个极大线性无关组, 是$\langle \alphas{s} \rangle$的一个基. 从而$\dim \langle \alphas{s} \rangle = \rank \{
\alphas{s} \}$.
\end{Proposition}

\begin{Proposition}
$ \langle \alpha_1, \cdots, \alpha_s \rangle = \langle \beta_1, \cdots, \beta_s \rangle 
\iff 
\{ \alpha_1, \cdots, \alpha_s \} \cong  \{ \beta_1, \cdots, \beta_s \} $
\end{Proposition}

\subsection{矩阵的秩}

\begin{Definition}
$$
A = 
\begin{blockarray}{cccc}
\alpha_1 & \cdots & \alpha_n & \\
\begin{block}{(ccc)c}
c_{11}     & \cdots & c_{1n} & \gamma_1 \\
\cdots     &        & \vdots & \vdots \\
c_{s1}     & \cdots & c_{sn} & \gamma_s \\
\end{block}
\end{blockarray}
$$
对于$A$的列向量组$ \alpha_1, \cdots, \alpha_n$, $\text{rank} \{ \alpha_1, \cdots, \alpha_n \}$称为$A$的\textbf{列秩}, $\langle \alpha_1, \cdots, \alpha_n \rangle$称为$A$的\textbf{列空间}.
对于$A$的行向量组$ \gamma_1, \cdots, \gamma_n$, $\text{rank} \{ \gamma_1, \cdots, \gamma_n \}$称为$A$的\textbf{行秩}, $\langle \gamma_1, \cdots, \gamma_n \rangle$称为$A$的\textbf{行空间}.

\end{Definition}

\begin{Theorem}
阶梯型矩阵$J$的行秩 $=$ 列秩 $=$ $J$的非零行的个数; 并且$J$的主元所在的列构成列向量的一个极大线性无关组.
\end{Theorem}

\begin{Theorem}
矩阵的初等行变换不改变矩阵的行秩.
\end{Theorem}

\begin{Theorem}
矩阵的初等行变换不改变矩阵的列向量组的线性相关性, 从而不改变矩阵的列秩.
\end{Theorem}

\begin{Theorem}
任一矩阵$A$的行秩等与它的列秩.
\end{Theorem}

\begin{Definition}
矩阵$A$的行秩与列秩统称为$A$的秩, 记作$\text{rank}(A)$
\end{Definition}

\begin{Corollary}
设矩阵$A$经过初等行变换化成阶梯型矩阵$J$, 则$\text{rank}(A) =$ $J$的非零行个数. 设$J$的主元所在的列是$j_1, j_2, \cdots, j_r$列, 则$A$的第$j_1, j_2, \cdots, j_r$列构成$A$的列向量组的一个极大线性无关组.
\end{Corollary}

\begin{Corollary}
$\text{rank} A = \text{rank} A' $
\end{Corollary}

\begin{Corollary}
矩阵的初等列变换不改变矩阵的秩.
\end{Corollary}

\begin{Theorem}[!]
任一非零矩阵$A$的秩等于它的不为零的子式的最高阶数.
\end{Theorem}

\begin{Corollary}
设$\text{rank}(A) = r$, 则$A$的不为$0$的$r$阶子式所在的列(行)构成$A$的列(行)向量组的一个极大线性无关组.
\end{Corollary}

\begin{Definition}[!, 满秩矩阵]
$n$级矩阵$A$的秩如果等于它的级数$n$, 则$A$称为\textbf{满秩矩阵}.
\end{Definition}

\begin{Corollary}[!]
$n$级矩阵满秩 $\Leftrightarrow |A| \neq 0$
\end{Corollary}

\subsection{线性方程组有解判别定理}

\begin{Theorem}
数域$K$上$n$元线性方程组有解 $\iff$ 增广矩阵$\tilde{A}$的秩 $=$ 系数矩阵$A$的秩.
\end{Theorem}

\begin{Theorem}
数域$K$上$n$元线性方程组有解时, 如果它的系数矩阵$A$的秩$=n$, 那么原方程组有唯一解; 如果$A$的秩$<n$, 那么原方程组有无穷多个解.
\end{Theorem}

\begin{Corollary}
数域$K$上$n$元其次线性方程组$Ax = 0$
\begin{tightcenter}
有非零解 $\iff$ $\rank A < n$
\end{tightcenter}
\end{Corollary}

\subsection{其次线性方程组解集结构}

\begin{Note}
数域$K$上$n$元其次线性方程组
\begin{equation} \label{eq:lin_eq}
x_1 \alpha_1 + \cdots +x_n \alpha_n = \textbf{0}
\end{equation}
的解集记作$W$. 则$W \subseteq K^n$, 且满足以下性质
\begin{enumerate}[(1)]
\item $\forall \eta, \delta \in W \implies \eta + \delta \in W$
\item $\gamma \in W, k \in K \implies k \gamma \in W$
\end{enumerate}
则$W$是$K^n$的一个子空间, 称$W$为(\ref{eq:lin_eq})
的\textbf{解空间}. 
\end{Note}

\begin{Note}
目标: 当(\ref{eq:lin_eq})有非零解时, 求$W$的一个基和维数.
\end{Note}

\begin{Theorem}
设$K$上$n$元其次线性方程组(\ref{eq:lin_eq})有非零解时,它的解空间$W$
的维数$\text{dim}\;W = n - \text{rank}(A)$. 
\end{Theorem}

\begin{Note}
把$W$的一个基称为齐次线性方程组(\ref{eq:lin_eq})
的一个\textbf{基础解系}. 
若$\eta_1, \cdots, \eta_{n-r}$
是
$n$元其次线性方程组(\ref{eq:lin_eq})的一个基础解系, 则(\ref{eq:lin_eq})的全部解为
\begin{tighteq*}
 k_1 \eta_1 + \cdots + k_{n-r} \eta_{n-r}  \;
(k_1, \cdots, k_{n-r} \in K)
\end{tighteq*}
\end{Note}

\subsection{非其次线性方程组的解集的结构}

\begin{Note}
数域$K$上的$n$元非其次线性方程组
\begin{equation} \label{eq:lin_eq2}
x_1 \alpha_1 + \cdots + x_n \alpha_n = \beta
\end{equation}
的解集记作$U$, 相应的非其次线性方程组
\begin{tighteq*}
x_1 \alpha_1 + \cdots + x_n \alpha_n = \mathbf{0}
\end{tighteq*}
的解集记作$W$, 则满足性质
\begin{enumerate}[(1)]
\item $\gamma, \delta \in U \Rightarrow \gamma - \delta \in W$
\item $\gamma \in U, \eta \in W \Rightarrow \gamma +\eta \in U$
\end{enumerate}
\end{Note}

\begin{Theorem}
如果数域$K$上的非其次线性方程组有解, 设$\gamma_0 \in U$, 称为$\gamma_0$是非其次线性方程组(\ref{eq:lin_eq})的一个\textbf{特解}, 则
$$
\gamma_0 + W := \{ \gamma_0 + \eta \mid \eta \in W \} = U
$$
\end{Theorem}

\begin{Definition}
$\gamma_0 + W$称为$W$形的\textbf{线性流形}, 或子空间$W$的一个\textbf{陪集}.
\end{Definition}

\subsection{子空间的运算}

\begin{Theorem}
设$V$是域$F$上的一个线性空间, $V_1$与$V_2$都是$V$的子空间, 则$V_1 \cap V_2$也是$V$的子空间.
\end{Theorem}

\begin{Theorem}
设$V_1$与$V_2$都是域$F$上的一个线性空间$V$的子空间, 则$V_1 + V_2$也是$V$的子空间, 且是包含$V_1 \cup V_2$的最小的子空间.
\end{Theorem}

\begin{Proposition}
$ 
\langle \alpha_1, \cdots, \alpha_s \rangle + 
\langle \beta_1, \cdots, \beta_r \rangle = 
\langle \alpha_1, \cdots, \alpha_s, \beta_1, \cdots, \beta_r \rangle
$
\end{Proposition}

\begin{Theorem}[!, 子空间的维数公式]
设$V_1, V_2$都是$V$的\;\fbox{有限维}\;子空间, 则$V_1 \cap V_2$和$V_1 + V_2$也是有限维的, 并且
$$
\text{dim} V_1 + \text{dim} V_2 = \text{dim} (V_1 + V_2) +\text{dim}(V_1 \cap V_2)
$$
\end{Theorem}

\begin{Corollary}
设$V_1, V_2$都是$V$的\;\fbox{有限维}\;子空间, 则 
\begin{tighteq*}
V_1 \cap V_2 = 0 \Leftrightarrow \text{dim} V_1 + \text{dim} V_2 = \text{dim} (V_1 + V_2) 
\end{tighteq*}
\end{Corollary}

%\subsubsection{子空间的直和}

\begin{Definition}[直和]
设$V_1, V_2$都是$V$的子空间, 如果$V_1 +V_2$中每个向量$\alpha$表示成$\alpha_1 + \alpha_2\;(\alpha_1 \in V_1, \alpha_2 \in V_2)$的表法唯一, 那么称$V_1 + V_2$是\textbf{直和},
记作$V_1 \oplus V_2$.
\end{Definition}

\begin{Theorem}
设$V_1$与$V_2$都是$V$的子空间, 则下列命题互相等价
\begin{enumerate}[(1)]
\item $V_1 + V_2$是直和
\item $V_1 + V_2$ 中零向量表法唯一, 即$0 = \alpha_1 + \alpha_2\;(\alpha_1 \in V_1, \alpha_2 \in V_2) \Rightarrow \alpha_1 = \alpha_2 = 0$
\item $V_1 \cap V_2 = 0$
\item $V_1$和$V_2$的一个基分别是多重集合$\mathfrak{S}_1$和$\mathfrak{S}_2$, 则合起来$\mathfrak{S}_1 \cup \mathfrak{S}_2$是$V_1 + V_2$的一个基
\end{enumerate}
\end{Theorem}

\begin{Theorem}
设$V_1, V_2$都是$V$有限维的子空间, 则
\begin{tightcenter}
$V_1 + V_2$是直和 $\Leftrightarrow$ $\text{dim}(V_1 + V_2) = \text{dim}V_1 + \text{dim}V_2$
\end{tightcenter}
\end{Theorem}

\begin{Definition}[补空间]
若$V = V_1 \oplus V_2$, 则称$V_2$是$V_1$的一个\textbf{补空间}, 也称$V_1$是$V_2$的一个补空间.
\end{Definition}

\begin{Proposition}
设$\text{dim}V = n$, 则$V$的每一个子空间$U$都在$V$中有一个补空间.
\end{Proposition}

\begin{Remark}
$U$的补空间不唯一.
\end{Remark}

\begin{Definition}
设$V_1, \cdots, V_m$都是$V$的子空间, 如果$V_1 + \cdots + V_m$中每个向量表示成$\alpha_1 + \cdots + \alpha_m$, 其中$\alpha_i \in V_i$~($i = 1, \cdots, m$), 且
表法唯一, 那么称$V_1 + \cdots + V_m$是\textbf{直和}\index{子空间!直和}, 记作
$ %\overset{n}{\underset{i=1}\bigoplus} V_i
\bigoplus_{i=1}^n V_i
$.
\end{Definition}

\begin{Theorem}
设$V_1, \cdots, V_m$都是$V$的子空间, 则下列命题等价:
\begin{enumerate}[(1)]
\item $V_1 + \cdots + V_m$是直和
\item $V_1 + \cdots + V_m$中$0$的表法唯一
\item $\displaystyle V_i \cap (\sum_{j \neq i} V_j) = 0, i = 1, \cdots, m$ % 这是一个例外
\item 设$V_i$的一个基为多重集合$\mathfrak{S}_i$\;($i = 1, \cdots, m$), 则合起来$\mathfrak{S}_1 \cup \cdots \cup \mathfrak{S}_m$是$V_1 + \cdots + V_m$的一个基
\end{enumerate}
\end{Theorem}

\begin{Theorem}
设$V_1, \cdots, V_m$都是$V$的有限维子空间, 则下列命题等价:
\begin{enumerate}[(1)]
\item $V_1 + \cdots + V_m$是直和
\item $\dim {(V_1 + \cdots + V_m)} = \dim V_1 + \cdots + \dim V_m$
\end{enumerate}
\end{Theorem}

\subsection{线性空间的同构}

\begin{Definition}[线性空间的同构]
设$V$和$V'$都是数域$K$上的线性空间, 如果$V$到$V'$有一个双射$\sigma$, 并且
\begin{tightcenter}
$\sigma(\alpha + \beta) = \sigma(\alpha) + \sigma(\beta), \forall \alpha, \beta \in V$(称为保持加法运算) \\
$\sigma(k \alpha) = k \sigma(\alpha), \forall \alpha \in V, k \in K$ (称为保持数乘运算)
\end{tightcenter}
则称$\sigma$是一个$V$到$V'$的\textbf{同构映射}, 此时称$V$与$V'$是同构的, 记作$V \cong V'$.
\end{Definition}

\begin{Property}
$\sigma(k_1 \alpha_1 + \cdots + k_s \alpha_s) = k_1 \sigma(\alpha_1) + \cdots + k_s \sigma (\alpha_s)$
\end{Property}

\begin{Property}
$V$中$\alphas{s}$线性相关 $\iff$ $V'$中$\sigma(\alpha_1), \cdots, \sigma(\alpha_s)$线性相关
\end{Property}

\begin{Property}
$V$中$\alphas{s}$线性无关 $\iff$ $V'$中$\sigma(\alpha_1), \cdots, \sigma(\alpha_s)$线性无关
\end{Property}

\begin{Property}
$\alphas{n}$是$V$的一个基 $\iff$ $\sigma(\alpha_1), \cdots, \sigma(\alpha_s)$是$V'$的一个基
\end{Property}

\begin{thm}
$V$和$V'$是数域$K$上两个有限维的线性空间
\begin{tightcenter}
$V$和$V'$同构 $\iff$ $\dim V = \dim V'$
\end{tightcenter}
\end{thm}

\begin{Note}
若有限维的线性空间$V \cong V'$, 则$\forall a \in V$, 与它在同态满射下的象$a'$,两者的坐标相同.
\end{Note}

\begin{Corollary}
数域$K$上任一$n$维线性空间$V$与$K^n$同构.
\end{Corollary}

\begin{Proposition}
设$\sigma$是数域$K$上线性空间$V$到$V'$的一个同构映射, $U$是$V$的一个子空间, 令
$$
\sigma(U) \triangleq \{ \sigma(\alpha) \mid \alpha \in U \}
$$
则$\sigma(U)$是$V'$的一个子空间. 若$\dim U = m$, 则$\dim \sigma(U) = m$. 
\end{Proposition}

\begin{Note}
设$\Omega = \{$ 数域$K$的线性空间 $\}$, 线性空间的同构是一个等价关系.
此时等价类称为\textbf{同构类}. 所有同构类组成的集合是$\Omega$的一个划分.
\end{Note}

\begin{Note}
设$\Omega_1 = \{$ 数域$K$的有限维线性空间 $\}$, 同构类有$\overline{0}, \overline{K}, \overline{K^2}, \cdots, \overline{K^n}, \cdots$.
同构类完全被维数决定.
\end{Note}

\subsection{商空间}

\begin{Note}
设$V$是数域$K$上的线性空间, $W$是$V$的一个子空间, 定义$V$上的等价关系$\sim$: 
\[ \beta \sim \alpha \triangleq \beta - \alpha \in W \]
任给$\alpha \in V$, 可以推出$\bar{\alpha} = \alpha + W$, 称为$W$的一个\textbf{陪集}. 把$\alpha$称为陪集$\alpha + W$的一个\textbf{代表}. 陪集$\alpha + W$的代表不唯一.
\end{Note}

\begin{Note}
$\alpha + W = \beta + W \iff \alpha - \beta \in W$
\end{Note}

\begin{Note}
$\eta \in W \iff \eta + W = W$
\end{Note}

\begin{Note}
$V / W \triangleq \{ \alpha + W \mid \alpha \in V\}$
是$V$的一个划分, 也称为$V$的对于子空间$W$的一个商集.
\end{Note}

\begin{Note}规定
\[
\begin{aligned}
(\alpha + W) + (\beta + W) &\triangleq (\alpha + \beta) + W \\
k(\alpha + W) &\triangleq k \alpha +W 
\end{aligned}
\]
可证明这两个定义与陪集的代表选择无关, 从而定义是合理的. 
易验证, $V/W$称为数域$K$上的一个线性空间, 把它称为\textbf{商空间}.
\end{Note}

\begin{Theorem}
设$V$是$n$维线性空间, $W$是$V$的一个子空间, 则$ \dim(V/W) = \dim(V) - \dim(W) $
\end{Theorem}

\begin{Theorem}
如果$V/W$的一个基为$(\beta_1 + W), \cdots, (\beta_t + W)$, 令$U \triangleq \langle \beta_1, \cdots, \beta_t \rangle$, 则$V = W \oplus U$, 并且$\beta_1, \cdots, \beta_t$是$U$的一个基.
\end{Theorem}

\begin{Note}
$V = W \oplus ~ \langle \beta_1, \cdots, \beta_t \rangle$ $\iff$
$(\beta_{1} +W), \cdots, (\beta_{t}+W)$是$V/W$的一个基.
\end{Note}

\begin{Proposition}
域$F$上的线性空间$V$的任一子空间$W$都有补空间.
\end{Proposition}
\section{矩阵的运算}

\subsection{矩阵的加法, 数量乘法, 乘法}

\begin{Definition}
$M_{s \times n}(K) \triangleq \{$ 数域$K$上$s \times n$矩阵 $\}$. $s = n$时, $M_{n}(K) \triangleq M_{n \times n}(K)$.
\end{Definition}

\begin{Definition}
两个矩阵相等 $\stackrel{\text{def}}{\iff}$ 行数相等, 列数相等, 对应元素相等.
\end{Definition}

\begin{Definition} 在$M_{s \times n}(K)$中, 
\[
\begin{pmatrix}
a_{11} & \cdots & a_{1n} \\
\vdots &        & \vdots \\
a_{s1} & \cdots & a_{sn} \\
\end{pmatrix}
+
\begin{pmatrix}
b_{11} & \cdots & b_{1n} \\
\vdots &        & \vdots \\
b_{s1} & \cdots & b_{sn} \\
\end{pmatrix}
\triangleq
\begin{pmatrix}
a_{11} + b_{11} & \cdots & a_{1n} + b_{1n} \\
\vdots          &        & \vdots \\
a_{s1} + b_{s1} & \cdots & a_{sn} + b_{sn} \\
\end{pmatrix}
\]
\end{Definition}


\begin{Definition} 在$M_{s \times n}(K)$中, 
\[
k
\begin{pmatrix}
a_{11} & \cdots & a_{1n} \\
\vdots &        & \vdots \\
a_{s1} & \cdots & a_{sn} \\
\end{pmatrix}
\triangleq
\begin{pmatrix}
k a_{11} & \cdots & k a_{1n} \\
\vdots &        & \vdots \\
k a_{s1} & \cdots & k a_{sn} \\
\end{pmatrix}
\]
\end{Definition}

\begin{Definition} 在$M_{s \times n}(K)$中, 
元素全为$0$的矩阵称为\textbf{零矩阵}, 记作$0$.
\end{Definition}

\begin{Note}
易验证, $M_{s \times n}(K)$成为数域$K$上的一个线性空间.
\end{Note}

\begin{Definition}
设$A = (a_{ij})_{s \times n}$, 
$B = (b_{ij})_{n \times n}$
, 定义矩阵$A$与$B$的乘积$AB \mapsto C = (c_{ij})_{s \times m}$, 其中
\[
c_{ij} = a_{i1}b_{1j} + \cdots + a_{in} b_{nj} = 
\sum_{k=1}^n a_{ik} b_{kj}, i = 1, \cdots, s; j = 1, \cdots, m
\]
\end{Definition}

\begin{Note}
矩阵的乘法不满足交换律.
\end{Note}

\begin{Note}
矩阵的乘法满足
\begin{enumerate}[(1)]
\item 结合律: $(AB)C = A(BC)$
\item 左分配率: $A(B + C) = AB + AC$
\item 右分配率: $(B + C)D = BD + CD$
\item 主对角线上的元素都是$1$, 其余元素都是$0$的$n$级矩阵称为\textbf{$n$级单位矩阵}, 记作$I_n$, 
或简记作$I$. 容易计算得$I_s A_{s \times n} = A_{s \times n}$, $A_{s \times n} I_n = A_{s \times n}$.
特别的, 若$A \in M_n(K)$, 则$IA = AI = A$.
\item $k(AB) = (kA)B = A(kB)$
\item 设$A \in M_n(K)$, 则$A^m \triangleq A A \cdots A$ ($m$个), $m \in N^*$; $A^0 \triangleq I$. 易得$A^{k + \ell} = A^k A^\ell, (A^k)^\ell = A^{k\ell}$, $k, \ell \in N$.
\item $(A + B)' = A' + B'$
\item $(kA)' = kA'$
\item $(AB)' = B'A'$
\end{enumerate}
\end{Note}

\begin{Note}
对于数域$K$上的$n$元线性方程组
$$
\left\{
\begin{matrix}
a_{11} x_1 +\cdots + a_{1n} x_n = b_1 \\
\vdots \\
a_{s1} x_1 +\cdots + a_{sn} x_n = b_n \\
\end{matrix}
\right.
$$%
可以写成
\[
\begin{pmatrix} 
a_{11} & \cdots & a_{1n} \\ 
\vdots &        & \vdots \\ 
a_{s1} & \cdots & a_{sn} \\
\end{pmatrix}
\begin{pmatrix} x_1 \\ \vdots \\ x_{n} \end{pmatrix}
= 
\begin{pmatrix} b_{1} \\ \vdots \\ b_{n} \end{pmatrix}
\]
记作$AX = \beta$. 相应的其次线性方程组为$AX = \mathbf{0}$
\end{Note}

\begin{Note}
\[
A B = 
\begin{blockarray}{ccc}
\alpha_1 & \cdots & \alpha_n \\
\begin{block}{(ccc)}
a_{11} & \cdots & a_{1n} \\
\vdots & \vdots & \vdots \\
a_{s1} & \cdots & a_{sn} \\
\end{block}
\end{blockarray}
\begin{pmatrix}
b_{11} & \cdots & b_{1j} & \cdots&  b_{1m} \\
\vdots &        & \vdots &       & \vdots \\
b_{n1} & \cdots & b_{nj} & \cdots & b_{nm} \\
\end{pmatrix}
\]
的第$j$列是
\[
\begin{pmatrix}
a_{11} b_{1j} + \cdots + a_{1n} b_{nj} \\
\vdots \\
a_{s1} b_{1j} + \cdots + a_{sn} b_{nj} \\
\end{pmatrix} = b_{1j} \alpha_1 + \cdots + b_{nj} \alpha_n
\]
于是
\[ 
AB = (\alpha_1, \cdots, \alpha_n) 
\begin{pmatrix}
b_{11} & \cdots & b_{1m} \\
\vdots &        & \vdots \\
b_{n1} & \cdots & b_{nm} \\
\end{pmatrix}
= 
\left(
\sum_{k=1}^n b_{k1} \alpha_k, \cdots, \sum_{k=1}^n b_{km} \alpha_k
\right)
\]
\end{Note}


\begin{Note}
\[
A B = 
\begin{pmatrix}
a_{11} & \cdots & a_{1n} \\
\vdots &        & \vdots \\
a_{i1} & \cdots & a_{in} \\
\vdots &        & \vdots \\
a_{s1} & \cdots & a_{sn} \\
\end{pmatrix}
\begin{blockarray}{cccc}
\begin{block}{(ccc)c}
b_{11} & \cdots & b_{1m} & \gamma_1 \\
\vdots & \vdots & \vdots & \vdots \\
b_{n1} & \cdots & b_{nm} & \gamma_n\\
\end{block}
\end{blockarray}
\]
的第$i$行是
\[
\begin{pmatrix}
\sum\limits_{j=1}^n a_{ij} b_{j1}, \ldots, \sum\limits_{j=1}^n a_{ij} b_{jm} \end{pmatrix} = 
\sum\limits_{j=1}^n a_{ij} \gamma_j =
a_{i1} \gamma_1 + \cdots + a_{in} \gamma_n
\]
于是
\[ 
AB = \begin{pmatrix}
a_{11} & \cdots & a_{1n} \\
\vdots &        & \vdots \\
a_{s1} & \cdots & a_{sn} \\
\end{pmatrix}
\begin{pmatrix}
\gamma_1 \\
\vdots \\
\gamma_n \\
\end{pmatrix}
= 
\begin{pmatrix}
a_{11} \gamma_1 + \cdots + a_{1n} \gamma_n \\
\vdots \\
a_{s1} \gamma_1 + \cdots + a_{sn} \gamma_n \\
\end{pmatrix}
\]
\end{Note}

%\begin{Note}
%从而$AB$的列向量组,可由$A$的列向量组线性表出. 于是$\rank(AB) \le \rank(A)$
%\end{Note}

\begin{Theorem}
$ \rank(AB) \le \min \{ \rank(A), \rank(B) \} $
\end{Theorem}

\subsection{特殊矩阵}

\begin{Definition}[基本矩阵]
只有第$i$行第$j$列为$1$, 其余元素都为$0$的矩阵,记作$E_{ij}$, 称为一个\textbf{基本矩阵}. 
\end{Definition}

\begin{Note}
矩阵组$E_{11}, \cdots, E_{1n}, \cdots, E_{s1}, \cdots, E_{sn}$是$M_{s \times n}(K)$的一个基, $\dim M_{s \times n}(K) = sn$.
\end{Note}

\begin{Definition}[对角矩阵]
$M_n(K)$中,主对角线上的元素依次为$d_1, \cdots, d_n$, 其余元素都为$0$的矩阵
\[
\begin{pmatrix}
d_1 &   &    \\
    & \ddots & \\
    &        &d_n
\end{pmatrix} 
\]
称为\textbf{对角矩阵}.
\end{Definition}

\begin{Note}
$\{$ 数域$K$上$n$级对角矩阵 $\}$ 是 $M_n(K)$的一个子空间. 
\end{Note}

\begin{Note}
\[
\begin{aligned}
\begin{pmatrix}
d_1 &   &    \\
    & \ddots & \\
    &        &d_s
\end{pmatrix} 
\begin{pmatrix}
\gamma_1 \\
\vdots \\
\gamma_s \\
\end{pmatrix} 
&= 
\begin{pmatrix}
d_1 \gamma_1 \\
\vdots \\
d_s \gamma_s \\
\end{pmatrix} \\
\begin{pmatrix}
\alpha_1, \cdots, \alpha_n \\
\end{pmatrix} 
\begin{pmatrix}
d_1 &   &    \\
    & \ddots & \\
    &        &d_n
\end{pmatrix} 
&=
\begin{pmatrix}
d_1 \alpha_1, \cdots, d_n \alpha_n \\
\end{pmatrix} \\
\begin{pmatrix}
d_1 &   &    \\
    & \ddots & \\
    &        &d_n
\end{pmatrix} 
\begin{pmatrix}
d'_1 &   &    \\
    & \ddots & \\
    &        &d'_n
\end{pmatrix} 
&=
\begin{pmatrix}
d_1 d'_1 &   &    \\
    & \ddots & \\
    &        &d_n d'_n
\end{pmatrix} 
\end{aligned} 
\]
\end{Note}

\subsubsection{数量矩阵}

\begin{Definition}[数量矩阵]
\[
kI = \begin{pmatrix}
k &   &    \\
    & \ddots & \\
    &        &k \\
\end{pmatrix} 
\]
称为\textbf{数量矩阵}.
\end{Definition}

\begin{Note}
$ KI \triangleq \{ kI \mid k \in K \} $ 是 $M_n(K)$的一个子空间. 
\end{Note}

\begin{Note} \ \\
\begin{enumerate}[(1)]
\item $ (kI) (\ell I) = (k\ell) I $
\item $A(kI) = (kI)A = kA$
\end{enumerate}
\end{Note}

\subsubsection{上(下)三角矩阵}

\begin{Definition}[上三角矩阵]
\[
\begin{pmatrix}
a_{11} & a_{12} & \cdots & a_{1n} \\
       & a_{22} & \cdots & a_{2n} \\
       &        & \ddots & \vdots \\
       &        &        & a_{nn} \\
\end{pmatrix}
\]
称为\textbf{$n$级上三角矩阵}.

\end{Definition}

\begin{Note}
$\{$ 数域$K$上$n$级上三角矩阵 $\}$ 是 $M_n(K)$的一个子空间. 
\end{Note}

\begin{Note}
两个上三角矩阵的乘积还是上三角矩阵
\end{Note}

\begin{Definition}[下三角矩阵]
\[
\begin{pmatrix}
a_{11} &        &        &        \\
a_{21} & a_{22} &        & \\
\vdots & \vdots & \ddots &  \\
a_{n1} & a_{n2} &        & a_{nn} \\
\end{pmatrix}
\]
称为\textbf{$n$级下三角矩阵}.
\end{Definition}

\subsubsection{初等矩阵}

\begin{Definition} 把$I$的第$i$行的$k$倍加到第$j$行,(或者第$j$列$k$倍加到第$i$列),得到
\[
P(j, i(k)) \triangleq 
\begin{blockarray}{cccccccc}
\begin{block}{(ccccccc)c}
1 &        &   &        &   &        &   & \\
  & \ddots &   &        &   &        &   & \text{row $i$} \\
  &        & 1 &        &   &        &   & \\
  &        &   & \ddots &   &        &   & \\
  &        & k &        & 1 &        &   & \text{row $j$} \\
  &        &   &        &   & \ddots &   & \\
  &        &   &        &   &        & 1 & \\
\end{block}
\end{blockarray}
\]
第$i$行和第$j$行互换,(或者第$i$列和第$j$列互换), 得到
\[
P(i, j) \triangleq 
\begin{blockarray}{cccccccc}
\begin{block}{(ccccccc)c}
1 &        &   &        &   &        &   & \\
  & \ddots &   &        &   &        &   & \text{row $i$} \\
  &        & 0 &        & 1 &        &   & \\
  &        &   & \ddots &   &        &   & \\
  &        & 1 &        & 0 &        &   & \text{row $j$} \\
  &        &   &        &   & \ddots &   & \\
  &        &   &        &   &        & 1 & \\
\end{block}
\end{blockarray}
\]
第$i$行乘以非零数$c$,(或者第$i$列乘以非零数$c$), 得到
\[
P(i(c)) \triangleq 
\begin{blockarray}{cccccccc}
\begin{block}{(ccccccc)c}
1 &        &   &        &   &        &   & \\
  & \ddots &   &        &   &        &   & \text{row $i$} \\
  &        & c &        & 0 &        &   & \\
  &        &   & \ddots &   &        &   & \\
  &        & 0 &        & 1 &        &   & \text{row $j$} \\
  &        &   &        &   & \ddots &   & \\
  &        &   &        &   &        & 1 & \\
\end{block}
\end{blockarray}
\]
\end{Definition}

\begin{Note} 用初等矩阵左乘一个矩阵相当于对这个矩阵做相应的初等行变换; 用初等矩阵右乘一个矩阵相当于对这个矩阵做相应的初等列变换.

\begin{comment}

\[
%%%%%%%%%%%%%%%%
\begin{aligned}
P(j, i(k))\;A &=
\begin{pmatrix}
\gamma_1 \\
\vdots \\
\gamma_i \\
\vdots \\
\gamma_j \\
\vdots \\
\gamma_s \\
\end{pmatrix} =
\begin{pmatrix}
1 &        &   &        &   &        &   \\
  & \ddots &   &        &   &        &   \\
  &        & 1 &        &   &        &   \\
  &        &   & \ddots &   &        &   \\
  &        & k &        & 1 &        &   \\
  &        &   &        &   & \ddots &   \\
  &        &   &        &   &        & 1 \\
\end{pmatrix}
\begin{pmatrix}
\gamma_1 \\
\vdots \\
\gamma_i \\
\vdots \\
\gamma_j \\
\vdots \\
\gamma_s \\
\end{pmatrix} 
= 
\begin{pmatrix}
\gamma_1 \\
\vdots \\
\gamma_i \\
\vdots \\
k \gamma_i +\gamma_j \\
\vdots \\
\gamma_s \\
\end{pmatrix} \\
%%%%%%%%%%%%%%%%%%
A\;P(j, i(k)) &= (\alpha_1, \cdots, \alpha_i, \cdots, \alpha_j, \cdots, \alpha_n)
\begin{pmatrix}
1 &        &   &        &   &        &   \\
  & \ddots &   &        &   &        &   \\
  &        & 1 &        &   &        &   \\
  &        &   & \ddots &   &        &   \\
  &        & k &        & 1 &        &   \\
  &        &   &        &   & \ddots &   \\
  &        &   &        &   &        & 1 \\
\end{pmatrix} \\
&= (\alpha_1, \cdots, \alpha_i + k \alpha_j, \cdots, \alpha_j, \cdots, \alpha_n)  \\
%%%%%%%%%%%%%%%%%%%%
P(i,j)\;A &=
\begin{pmatrix}
\gamma_1 \\
\vdots \\
\gamma_i \\
\vdots \\
\gamma_j \\
\vdots \\
\gamma_s \\
\end{pmatrix} =
\begin{pmatrix}
1 &        &   &        &   &        &   \\
  & \ddots &   &        &   &        &   \\
  &        & 0 &        & 1 &        &   \\
  &        &   & \ddots &   &        &   \\
  &        & 1 &        & 0 &        &   \\
  &        &   &        &   & \ddots &   \\
  &        &   &        &   &        & 1 \\
\end{pmatrix}
\begin{pmatrix}
\gamma_1 \\
\vdots \\
\gamma_i \\
\vdots \\
\gamma_j \\
\vdots \\
\gamma_s \\
\end{pmatrix} 
= 
\begin{pmatrix}
\gamma_1 \\
\vdots \\
\gamma_j \\
\vdots \\
\gamma_i \\
\vdots \\
\gamma_s \\
\end{pmatrix} \\
%%%%%%%%%%%%%%%%%%%%%%%%%%
A\;P(i,j) &= (\alpha_1, \cdots, \alpha_i, \cdots, \alpha_j, \cdots, \alpha_n)
\begin{pmatrix}
1 &        &   &        &   &        &   \\
  & \ddots &   &        &   &        &   \\
  &        & 0 &        & 1 &        &   \\
  &        &   & \ddots &   &        &   \\
  &        & 1 &        & 0 &        &   \\
  &        &   &        &   & \ddots &   \\
  &        &   &        &   &        & 1 \\
\end{pmatrix} \\
&= (\alpha_1, \cdots, \alpha_i + k \alpha_j, \cdots, \alpha_i, \cdots, \alpha_n)  \\
\end{aligned} 
]\]
\end{comment}
\end{Note}

\subsection{x}

\begin{Definition}[对称矩阵]
数域上$K$上的$n$级矩阵是\textbf{对称矩阵} $\stackrel{\text{def}}{\iff}$ $A' = A$.
\end{Definition}

\begin{Note}
$\{$数域$K$上$n$级对称矩阵$\}$ 是 $M_n(K)$ 的一个子空间.
\end{Note}

\begin{Definition}[斜对称矩阵]
数域上$K$上的$n$级矩阵是\textbf{斜对称矩阵} $\stackrel{\text{def}}{\iff}$ $A' = -A$.
\end{Definition}

\begin{Note}
斜对称矩阵的形状
\[
\begin{pmatrix}
0       & a_{12}   & \cdots & a_{1n} \\
-a_{12} & 0        & \cdots & a_{2n} \\
\vdots  & \vdots   & \ddots & \vdots \\
-a_{1n} & -a_{2n}  & \cdots & 0      \\
\end{pmatrix}
\]
\end{Note}

\begin{Note}
$\{$数域$K$上$n$级斜对称矩阵$\}$ 是 $M_n(K)$ 的一个子空间.
\end{Note}

\subsection{可逆矩阵}

\begin{Definition}[可逆矩阵]
设$A \in M_n(K)$, 如果存在$B \in M_n(K)$, 使得$AB = BA = I$, 那么称$A$是\textbf{可逆矩阵}, $B$称为$A$的\textbf{逆矩阵}. 把$A$的逆矩阵记作$A^{-1}$.
\end{Definition}

\begin{Definition}[伴随矩阵]
\[
A^* \triangleq
\begin{pmatrix}
A_{11} & A_{21} & \cdots & A_{n1} \\
A_{12} & A_{22} & \cdots & A_{n2} \\
\vdots & \vdots &        & \vdots \\
A_{1n} & A_{2n} & \cdots & A_{nn} \\
\end{pmatrix}
\]
称为$A$的\textbf{伴随矩阵}.
\end{Definition}

\begin{Note}
数域$K$上的$n$级矩阵$A$可逆 $\iff$ $|A| \neq 0$ $\iff$ $A$满秩; 且$A$可逆时, $A^{-1} = \dfrac{1}{|A|}A^{*}$.
\end{Note}

\begin{Proposition}
设$A, B \in M_n(K)$, 若$AB = I$, 则$A$, $B$都可逆, 且$A^{-1} = B$, $B^{-1} = A$.
\end{Proposition}

\begin{Proposition}
初等矩阵都是可逆矩阵, 且$P(j, i(k))^{-1} = P(j, i(-k))$, $P(i, j)^{-1} = P(i, j)$, $P(i(c))^{-1} = P(i(c^{-1}))$.
\end{Proposition}

\begin{Proposition}
若$A$, $B$都是$n$级可逆矩阵, 则$AB$可逆, 且$(AB)^{-1} = B^{-1} A^{-1}$.
\end{Proposition}

\begin{Note}
若$A_1$, $\cdots$, $A_s$都是$n$级可逆矩阵, 则$A_1 \cdots A_s$也是可逆矩阵, 且$\left( A_1 \cdots A_s \right)^{-1} = A_s^{-1} \cdots A_1^{-1}$.
\end{Note}

\begin{Proposition}
若$A$可逆, 则$A'$可逆, 且$\left(A'\right)^{-1} = \left(A^{-1}\right)'$.
\end{Proposition}

\begin{Proposition}
$n$级矩阵$A$可逆, 则它可以通过初等行变换, 化成$I$.
\end{Proposition}

\begin{Proposition}
$n$级矩阵$A$可逆 $\iff$ $A$等于一些初等矩阵的乘积.
\end{Proposition}

\begin{Proposition}
用可逆矩阵左(右)乘矩阵$A$, 不改变$A$的秩.
\end{Proposition}

\begin{Note}
求$A^{-1}$的基本方法, 称为\textbf{初等变换法}
\[
\begin{pmatrix}
A~I
\end{pmatrix}
\xrightarrow{\text{初等行变换}}
\begin{pmatrix}
I~A^{-1}
\end{pmatrix}
\]
\end{Note}

\subsection{矩阵的分块}

\begin{Note}
分块矩阵的乘法可以和普通矩阵的乘法一样做.
\end{Note}

\begin{Note}
分块矩阵的初等行变换
\begin{enumerate}[(1)]
\item 把一个块行的左$P$倍加到另一个块行中
\item 交换两个块行的位置
\item 用一个可逆矩阵左乘某一块行
\end{enumerate}
\end{Note}

\begin{Note}
分块矩阵的初等列变换
\begin{enumerate}[(1)]
\item 把一个块列的右$P$倍加到另一个块列中
\item 交换两个块列的位置
\item 用一个可逆矩阵右乘某一块列
\end{enumerate}
\end{Note}

\begin{Note} \ \\
\begin{itemize}
\item 分块初等矩阵都是可逆矩阵. 
\item 用分块初等矩阵左乘一个矩阵相当于对这个矩阵做相应的初等行变换; 用分块初等矩阵右乘一个矩阵相
当于对这个矩阵做相应的初等列变换.
\item 分块矩阵的初等行变换不改变矩阵的秩. 
\end{itemize}
\end{Note}

\begin{Theorem}[Binet-Cauchy公式]
设$A = (a_{ij})_{s \times n}$, $B = (b_{ij})_{n \times s}$
\[
\norm{AB} = \begin{cases}
0 & s > n \\
\norm{A} \norm{B} & s = n \\
\mathlarger{\mathlarger{\mathlarger{\sum}}}\limits_{1 \le v_1 \le v_2 \le \cdots \le v_s \le n} A \begin{pmatrix}
1,   & 2,   & \cdots, & s \\
v_1, & v_2, & \cdots, & v_s
\end{pmatrix} B \begin{pmatrix}
v_1, & v_2, & \cdots, & v_s \\
1,   & 2,   & \cdots, & s
\end{pmatrix} & s \le n \\
\end{cases}
\]
\end{Theorem}
\section{线性映射}

\begin{Definition}[线性映射]
设$V$和$V'$都是域$F$上的线性空间, $V$到$V'$的一个映射$\underline{A}$如果满足
\[
\begin{aligned}
  \underline{A}(\alpha + \beta) = \underline{A}(\alpha) + \underline{A}(\beta), \forall \alpha, \beta \in V \\
  \underline{A}{(k \alpha)} = k \underline{A}(\alpha), \forall \alpha \in V, \forall k \in F \\
\end{aligned}
\]
那么称$\underline{A}$是$V$到$V'$的一个\textbf{线性映射}.
\end{Definition}

\printindex

\end{document}
