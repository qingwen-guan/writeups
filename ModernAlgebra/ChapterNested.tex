\chapter{套娃}

\section{数域$K$上的一元多项式环$K[x]$} %%%%%%%%%%%%%%%%%

\begin{Note}
数域$K$上的一元多项式环$K[x]$是一个欧式环、主理想环、唯一分解环、整环.
\end{Note}

\begin{Property}
设$f(x), g(x) \in K[x]$, 则
\[
\begin{aligned}
\deg \Big( f(x) + g(x) \Big) &\le \max \Big( \deg f(x), \deg g(x) \Big) \\
\deg \Big( f(x) g(x) \Big) &= \deg f(x) + \deg g(x) 
\end{aligned}
\]
\end{Property}

\begin{Corollary}
设$f(x), g(x) \in K[x]$, 则
\[
f(x) \neq 0, g(x) \neq 0 \implies f(x) g(x) \neq 0
\]
\end{Corollary}

\begin{Corollary}[消去律]
设$f(x), g(x), h(x) \in K[x]$, 则
\[
f(x) g(x) = f(x) h(x) \And f(x) \neq 0 \implies g(x) = h(x)
\]
\end{Corollary}


\begin{Note}
任给$A \in M_n(K)$, 矩阵$A$的多项式组成的集合$K[A]$. 容易验证非空集合$K[A]$对于矩阵的减法和乘法封闭, 从而$K[A]$是环$M_n(K)$的一个子环. 且$K[A]$是有单位元的交换环.
\end{Note}

\begin{Note}
$KI$是$K[A]$的一个子环.
\end{Note}

\begin{Theorem}[一元多项式环的通用性质]
设$K$是一个数域, $R$是一个有单位元$1'$的交换环. 且$K$到$R$的一个子环$R_1$(含有$1'$)
有一个同构映射$\tau$. 任给$t \in R$, 令
\[
\begin{aligned}
\sigma_t: K[X] &\rightarrow R \\
          f(x) = \sum\limits_{i=0}^n a_i x^i &\mapsto \sum\limits_{i=0}^n \tau(a_i) t^i \triangleq f(t)
\end{aligned}
\]
则$\sigma_t$是$K[x]$到$R$的一个映射, 且$\sigma_t(x) = t$, 且$\sigma_t$保持加法、乘法运算, 即
\[
f(x) +g(x) = h(x), f(x)g(x) = p(x) \implies f(t) + g(t) = h(t), f(t) g(t) = p(t)
\]
称$\sigma_t$是$x$用$t$带入.
\end{Theorem}

\begin{Note}定理回答了为什么多项式环的未定元可以带入特定的值.
\begin{center}
\begin{forest}
for tree={grow=west, parent anchor=west, child anchor=east, anchor=east}
[, s sep=2em,phantom
	[ 数域$K$, name=FieldK
		[{$K[x]$} , tier=x, name=Kx, edge label={node [midway, above, font=\scriptsize] {子环}} 	
		]
	]		
	[ {含幺子环${R_1}$}, name=R1
		[ {含幺交换环$R$} ,tier=x, name=R, edge label={node [midway, below, font=\scriptsize] {子环}} 
		]
	]
] {
	\draw[->%, out=-130,in=-85
	] ([xshift=-1em]FieldK.south east)
	--([xshift=-1em]R1.north east) node [midway, left] {$\tau$}  node [midway, right] {$\cong$}
	;
	\draw[->
	] ([xshift=-1em]Kx.south east)
	--([xshift=-1em]R.north east) node [midway, left] {$\sigma$}
	;
	\draw[decorate, decoration={brace, amplitude=1.5em}]
     ([yshift=-.5em]current bounding box.north east) --
      node[right=2em%, font=\sffamily\bfseries\Large
      ]{{
      	$\implies  \left\{ \begin{aligned}
      		& \begin{aligned}
      		\forall t \in R, \exists \sigma_t: K[X] &\rightarrow R \\
          		f(x) = \sum\limits_{i=0}^n a_i x^i &\mapsto \sum\limits_{i=0}^n \tau(a_i) t^i \triangleq f_\tau(t) \\
          	\end{aligned} \\
          	& \text{满足} \\
          	& 
          		\quad \sigma_t(x) = t \\
          	& \quad
          		f(x) + g(x) = h(x) \Rightarrow f_\tau(t) + g_\tau(t) = h_\tau(t) \\
          	& \quad
          		f(x) g(x) = p(x) \Rightarrow f_\tau(t) g_\tau(t) = p_\tau(t) \\
        \end{aligned} \right.$
      }}
      ([yshift=-.5em]current bounding box.south east)
   ;	
}
%\BraceForest[red,ultra thick]%
  % inside environment forest
\end{forest}
\end{center}
\end{Note}


\subsection{整除关系, 带余除法}

\begin{Proposition}[!]
在$K[x]$中, $g(x) \mid f(x) \And f(x) \neq 0 \implies \deg g(x) \le \deg f(x)$.
\end{Proposition}

\begin{Proposition}
若$f(x), g(x) \in K[x]$, 那么
\begin{tightcenter}
$f(x) \sim g(x) \iff f(x) = c g(x)~(c \in K^*)$
\end{tightcenter}
\end{Proposition}

\begin{Theorem}[带余除法]
设$f(x), g(x) \in K[x]$, 且$g(x) \neq 0$, 则\;\fbox{唯一}\;的存在$K[x]$中一对多项式$h(x), r(x)$,
使得
\[
f(x) = h(x) g(x) + r(x), \deg r(x) < \deg g(x)
\]
我们把$h(x)$叫做\textbf{商式},
$r(x)$叫做\textbf{余式}.
\end{Theorem}

\begin{Note}
因为证明用到逆元, 所以要是至少是除环(我还没确定够不够), 普通环肯定不行.
\end{Note}

\begin{Corollary}
设$f(x), g(x) \in K[x]$, 且$g(x) \neq 0$, 则
\begin{tightcenter}
$g(x) \mid f(x)$ $\iff$ $g(x)$除$f(x)$的余式是$0$.
\end{tightcenter}
\end{Corollary}

\begin{Proposition}[整除性不随数域的扩大而改变]
设$f(x), g(x) \in K[x]$, $g(x) \neq 0$, 数域$E$包含$K$, 则
\begin{tightcenter}
在$K[x]$中$g(x) \mid f(x)$ $\iff$ 在$E[x]$中$g(x) \mid f(x)$ 
\end{tightcenter}
\end{Proposition}

%\begin{Proposition}
%在$K[x]$中, 若$g(x) \mid f_i (x)$~($i=1,\cdots, s$), 则对任意$u_1(x), \cdots, u_s(x)$, 都有
%$g(x) \mid u_1(x) f_1(x) + \cdots + u_s(x) f_s(x)$.
%\end{Proposition}

\subsection{最大公因式}

\begin{Definition}[因式]
如果$g(x) \mid f(x)$, 则$g(x)$称为$f(x)$的一个\textbf{因式}, $f(x)$称为$g(x)$的一个\textbf{倍式}.
\end{Definition}

\begin{Definition}[公因式]
$K[x]$中,若$c(x) \mid f(x)$且$c(x) \mid g(x)$, 则称$c(x)$是$f(x)$和$g(x)$的一个\textbf{公因式}.
\end{Definition}

\begin{Definition}[最大公因式]
设$f(x), g(x) \in K[x]$, 如果有$K[x]$中的一个多项式$d(x)$满足
\begin{enumerate}[(1)]
\item $d(x) \mid f(x)$且$d(x) \mid g(x)$
\item $f(x)$与$g(x)$的任一公因式$c(x) \mid d(x)$
\end{enumerate}
那么称$d(x)$是$f(x)$与$g(x)$的一个\textbf{最大公因式}.
\end{Definition}

\begin{Note}
$\forall f(x) \in K[x]$, $f(x)$与$0$的一个最大公因式是$f(x)$. 特别的$0$与$0$的一个最大公因式是$0$.
\end{Note}

\begin{Note}
接下来探索$f(x) \neq 0$与$g(x) \neq 0$的最大公因式是否存在? 如果存在, 如何求? 有多少个? 
\end{Note}

\begin{Lemma}
设$f(x), g(x) \in K[x]$, 且$g(x) \neq 0$, 如果有下式成立
\[
	f(x) = h(x) g(x) + r(x)
\]
那么
\[
c(x) \mid f(x) \And c(x) \mid g(x) \iff c(x) \mid g(x) \And c(x) \mid r(x)
\]
从而
\begin{center}
$d(x)$是$f(x)$与$g(x)$的一个最大公因式 $\iff$ $d(x)$是$g(x)$与$r(x)$的一个最大公因式
\end{center}
\end{Lemma}

\begin{comment}
\begin{Note}[辗转相除法]
设$f(x), g(x) \in K[x]$, 且$g(x) \neq 0$, 作带余除法
\[
	f(x) = h_1(x) g(x) + r_1(x), \deg r_1(x) < \deg g(x)
\]
若$r_1(x) \neq 0$, 则
\[
	g(x) = h_2(x) r_1(x) + r_2(x), \deg r_2(x) < \deg r_1(x)
\]
若$r_2(x) \neq 0$, 则
\[
	r_1(x) = h_3(x) r_2(x) + r_3(x), \deg r_3(x) < \deg r_2(x)
\]
\[
\vdots \quad \text{有限步必终止, 不妨设$r_{s+1}$是第一个$=0$的多项式} \quad \vdots
\]
若$r_{s-1}(x) \neq 0$, 则
\[
	r_{s-2}(x) = h_s(x) r_{s-1}(x) + r_s(x), \deg r_s(x) < \deg r_{s-1}(x)
\]
若$r_{s}(x) \neq 0$, 则
\[
	r_{s-1}(x) = h_{s+1}(x) r_{s}(x) + 0
\]
于是
\[
\begin{aligned}
& \phantom{\implies} \text{$ r_s(x) $ 是 $r_s(x)$与$0$的一个最大公因式} \\
& \implies           \text{$ r_s(x) $ 是 $r_{s-1}(x)$与$r_s(x)$的一个最大公因式} \\
& \implies           \text{$ r_s(x) $ 是 $r_{s-2}(x)$与$r_{s-1}(x)$的一个最大公因式} \\
& \phantom{\implies} \vdots \\
& \implies           \text{$ r_s(x) $ 是 $r_{1}(x)$与$r_{2}(x)$的一个最大公因式} \\
& \implies           \text{$ r_s(x) $ 是 $g(x)$与$r_{1}(x)$的一个最大公因式} \\
& \implies           \text{$ r_s(x) $ 是 $f(x)$与$g(x)$的一个最大公因式} \\
\end{aligned}
\]
\end{Note}
\end{comment}

\begin{Theorem}[!]
$K[x]$中任意一对多项式$f(x)$与$g(x)$都有一个最大公因式$d(x)$, 并且存在$u(x), v(x) \in K[x]$使得
\[
	u(x) f(x) + v(x) g(x) = d(x)
\]
\end{Theorem}

\begin{Note}
设$d_1(x)$, $d_2(x)$都是$f(x)$与$g(x)$的最大公因式, 则$d_1(x) \mid d_2(x)$且$d_2(x) \mid d_1(x)$, 从而$d_1(x) \sim d_2(x)$.
\end{Note}

\begin{Note}
当$f(x)$与$g(x)$不全为$0$时, 它们的最大公因式是非零多项式. 用$\Big( f(x), g(x) \Big)$表示$f(x)$与$g(x)$的首项系数为$1$的最大公因式,
简称\textbf{首一最大公因式}.
\end{Note}

\begin{Proposition}[首一最大公因式不随数域的扩大而改变]
设$f(x), g(x) \in K[x]$, 且$g(x) \neq 0$, 数域$E \supseteq K$, 则
\begin{center}
$f(x)$与$g(x)$在$K[x]$中的首一最大公因式 $=$ $f(x)$与$g(x)$在$E[x]$中的首一最大公因式.
\end{center}
\end{Proposition}

\begin{Definition}[互素]
在$K[x]$中, 若$\Big( f(x), g(x) \Big) = 1$, 则称$f(x)$和$g(x)$\textbf{互素}.
\end{Definition}

\begin{Note}
$f(x)$与$g(x)$互素 $\iff$ $f(x)$与$g(x)$的任一公因式$c(x)$是零次多项式.
\end{Note}

\begin{Theorem}[!]
在$K[x]$中$f(x)$与$g(x)$互素 $\iff$ 存在$u(x), v(x) \in K[x]$使得$u(x) f(x) +v(x) g(x) = 1$.
\end{Theorem}

\begin{Proposition}[互素性不随数域的扩大而改变]
设$f(x), g(x) \in K[x]$, 且$g(x) \neq 0$, 数域$E \supseteq K$, 则
\begin{center}
$f(x)$与$g(x)$在$K[x]$中互素 $\iff$ $f(x)$与$g(x)$在$E[x]$中互素 
\end{center}
\end{Proposition}

\begin{Property}[!]
在$K[x]$中, 若$f(x) \mid g(x) h(x)$, 且$\Big( f(x), g(x) \Big) = 1$, 则 $f(x) \mid h(x)$.
\end{Property}

\begin{Property}
在$K[x]$中, 若$f(x) \mid h(x)$, $g(x) \mid h(x)$, 且$\Big( f(x), g(x) \Big) = 1$, 则$f(x) g(x) \mid h(x)$.
\end{Property}

\begin{Property}
在$K[x]$中, 若$\Big( f(x), h(x) \Big) = 1$, 且$\Big( g(x), h(x) \Big) = 1$, 则$\Big( f(x) g(x), h(x) \Big) = 1$.
\end{Property}

\begin{Corollary}
在$K[x]$中, 若$\Big( f_i(x), h(x) \Big) = 1$, $i = 1, \dots, s$, 则$\Big( f_1(x)\cdots f_s(x), h(x) \Big) = 1$.
\end{Corollary}

\subsection{不可约多项式}

\begin{Definition}
设$f(x) \in K[x]$, $\deg f(x) > 0$, 如果$f(x)$的因式只有零次多项式和$f(x)$的相伴元, 那么称$f(x)$是在$K$上\textbf{不可约}的, 否则称$f(x)$在$K$上\textbf{可约}.
\end{Definition}

\begin{Theorem}
设$p(x) \in K[x]$, 且$\deg p(x) > 0$, 则下列命题等价
\begin{enumerate}[(1)]
\item $p(x)$在$K$上不可约
\item $\forall f(x) \in K[x]$, 有$\Big( p(x), f(x) \Big) = 1$ 或 $p(x) \mid f(x)$
\item 在$K[x]$中, 从$p(x) \mid f(x) g(x)$ 可推出 $p(x) \mid f(x)$ 或 $p(x) \mid g(x)$
\item 在$K[x]$中, $p(x)$不能分解成两个次数比$p(x)$低的多项式的乘积
\end{enumerate}
\end{Theorem}

\begin{Corollary}
$K[x]$中, 若不可约多项式$p(x) \mid f_1(x) \cdots f_s(x)$, 则$p(x) \mid f_j(x)$, 其中$j \in \{ 1, \dots, s \}$.
\end{Corollary}

\begin{Corollary}
$K[x]$中, 一次多项式是不可约的.
\end{Corollary}

\begin{Corollary}
$f(x) \in K[x]$, $\deg f(x) > 0$
\begin{center}
$f(x)$可约 $\iff$ $f(x)$能够分解成两个次数比$f(x)$低的多项式的乘积.
\end{center}
\end{Corollary}

\begin{Theorem}[唯一因式分解定理]
$K[x]$中每一个次数大于$0$的多项式$f(x)$都能\;\fbox{唯一}\;分解成$K$上有限多个不可约多项式的乘积. 其中唯一性是指若
$f(x) = p_1(x)\cdots p_s(x) = q_1(x) \cdots q_t(x)$, 则$s=t$, 且适当排列因式的次序, 有$p_i(x) \sim q_i(x)$.
\end{Theorem}

\begin{Note}[标准分解式]
\[
f(x) = a p_1(x) ^{\ell_1} p_2(x)^{\ell_2} \cdots p_m(x)^{\ell_m}
\]
称为$f(x)$的\textbf{标准分解式}, 其中$p_1(x), p_2(x), \dots, p_m(x)$是两两不等的
首一不可约多项式, $\ell_i > 0$, $i = 1, 2, \dots, m$.
\end{Note}

\begin{Definition}
设$f(x) \in K[x]$, $\deg f(x) > 0$, 若不可约多项式$p(x)$满足$p^k(x) \mid f(x)$, 且$p^{k+1}(x) \nmid f(x)$,
则称$p(x)$是$f(x)$的\textbf{$k$重因式}. 
\begin{itemize}
\item 当$k = 0$时, $p(x)$不是$f(x)$的因式. 
\item 当$k = 1$时, $p(x)$称为$f(x)$的\textbf{单因式}.
\item 当$k \ge 2$时, $p(x)$称为$f(x)$的\textbf{重因式}.
\end{itemize}
\end{Definition}