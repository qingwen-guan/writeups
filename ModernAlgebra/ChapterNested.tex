\chapter{套娃}

\section{数域$K$上的一元多项式环$K[x]$} %%%%%%%%%%%%%%%%%

\begin{Note}
数域$K$上的一元多项式环$K[x]$是一个欧式环、主理想环、唯一分解环、整环.
\end{Note}

\begin{Property}
设$f(x), g(x) \in K[x]$, 则
\[
\begin{aligned}
\deg \Big( f(x) + g(x) \Big) &\le \max \Big( \deg f(x), \deg g(x) \Big) \\
\deg \Big( f(x) g(x) \Big) &= \deg f(x) + \deg g(x) 
\end{aligned}
\]
\end{Property}

\begin{Corollary}
设$f(x), g(x) \in K[x]$, 则
\[
f(x) \neq 0, g(x) \neq 0 \implies f(x) g(x) \neq 0
\]
\end{Corollary}

\begin{Corollary}[消去律]
设$f(x), g(x), h(x) \in K[x]$, 则
\[
f(x) g(x) = f(x) h(x) \And f(x) \neq 0 \implies g(x) = h(x)
\]
\end{Corollary}


\begin{Note}
任给$A \in M_n(K)$, 矩阵$A$的多项式组成的集合$K[A]$. 容易验证非空集合$K[A]$对于矩阵的减法和乘法封闭, 从而$K[A]$是环$M_n(K)$的一个子环. 且$K[A]$是有单位元的交换环.
\end{Note}

\begin{Note}
$KI$是$K[A]$的一个子环.
\end{Note}

\begin{Theorem}[一元多项式环的通用性质]
设$K$是一个数域, $R$是一个有单位元$1'$的交换环. 且$K$到$R$的一个子环$R_1$(含有$1'$)
有一个同构映射$\tau$. 任给$t \in R$, 令
\[
\begin{aligned}
\sigma_t: K[X] &\rightarrow R \\
          f(x) = \sum\limits_{i=0}^n a_i x^i &\mapsto \sum\limits_{i=0}^n \tau(a_i) t^i \triangleq f(t)
\end{aligned}
\]
则$\sigma_t$是$K[x]$到$R$的一个映射, 且$\sigma_t(x) = t$, 且$\sigma_t$保持加法、乘法运算, 即
\[
f(x) +g(x) = h(x), f(x)g(x) = p(x) \implies f(t) + g(t) = h(t), f(t) g(t) = p(t)
\]
称$\sigma_t$是$x$用$t$带入.
\end{Theorem}

\begin{Note}定理回答了为什么多项式环的未定元可以带入特定的值.
\begin{center}
\begin{forest}
for tree={grow=west, parent anchor=west, child anchor=east, anchor=east}
[, s sep=2em,phantom
	[ 数域$K$, name=FieldK
		[{$K[x]$} , tier=x, name=Kx, edge label={node [midway, above, font=\scriptsize] {子环}} 	
		]
	]		
	[ {含幺子环${R_1}$}, name=R1
		[ {含幺交换环$R$} ,tier=x, name=R, edge label={node [midway, below, font=\scriptsize] {子环}} 
		]
	]
] {
	\draw[->%, out=-130,in=-85
	] ([xshift=-1em]FieldK.south east)
	--([xshift=-1em]R1.north east) node [midway, left] {$\tau$}  node [midway, right] {$\cong$}
	;
	\draw[->
	] ([xshift=-1em]Kx.south east)
	--([xshift=-1em]R.north east) node [midway, left] {$\sigma$}
	;
	\draw[decorate, decoration={brace, amplitude=1.5em}]
     ([yshift=-.5em]current bounding box.north east) --
      node[right=2em%, font=\sffamily\bfseries\Large
      ]{{
      	$\implies  \left\{ \begin{aligned}
      		& \begin{aligned}
      		\forall t \in R, \exists \sigma_t: K[X] &\rightarrow R \\
          		f(x) = \sum\limits_{i=0}^n a_i x^i &\mapsto \sum\limits_{i=0}^n \tau(a_i) t^i \triangleq f_\tau(t) \\
          	\end{aligned} \\
          	& \text{满足} \\
          	& 
          		\quad \sigma_t(x) = t \\
          	& \quad
          		f(x) + g(x) = h(x) \Rightarrow f_\tau(t) + g_\tau(t) = h_\tau(t) \\
          	& \quad
          		f(x) g(x) = p(x) \Rightarrow f_\tau(t) g_\tau(t) = p_\tau(t) \\
        \end{aligned} \right.$
      }}
      ([yshift=-.5em]current bounding box.south east)
   ;	
}
%\BraceForest[red,ultra thick]%
  % inside environment forest
\end{forest}
\end{center}

\subsection{整除关系, 带余除法}

\end{Note}
\begin{Proposition}
在$K[x]$中, 如果$g(x) \mid f(x)$, 其中$f(x) \neq 0$, 则$\deg g(x) \le \deg f(x)$.
\end{Proposition}

\begin{Proposition}
若$f(x), g(x) \in K[x]$, 那么
\begin{tightcenter}
$f(x) \sim g(x) \iff f(x) = c g(x)~(c \in K^*)$
\end{tightcenter}
\end{Proposition}

\begin{Theorem}[带余除法]
设$f(x), g(x) \in K[x]$, 且$g(x) \neq 0$, 则\;\fbox{唯一}\;的存在$K[x]$中一对多项式$h(x), r(x)$,
使得$f(x) = h(x) g(x) + r(x)$, $\deg r(x) < \deg g(x)$. 我们把$h(x)$叫做\textbf{商式},
$r(x)$叫做\textbf{余式}.
\end{Theorem}

\begin{Note}
因为证明用到逆元, 所以要是至少是除环(我还没确定够不够), 普通环肯定不行.
\end{Note}

\begin{Corollary}
设$f(x), g(x) \in K[x]$, 且$g(x) \neq 0$, 则
\begin{tightcenter}
$g(x) \mid f(x)$ $\iff$ $g(x)$除$f(x)$的余式是$0$.
\end{tightcenter}
\end{Corollary}

\begin{Proposition}[整除性不随数域的扩大而改变]
设$f(x), g(x) \in K[x]$, $g(x) \neq 0$, 数域$E$包含$K$, 则
\begin{tightcenter}
在$K[x]$中$g(x) \mid f(x)$ $\iff$ 在$E[x]$中$g(x) \mid f(x)$ 
\end{tightcenter}
\end{Proposition}

%\begin{Proposition}
%在$K[x]$中, 若$g(x) \mid f_i (x)$~($i=1,\cdots, s$), 则对任意$u_1(x), \cdots, u_s(x)$, 都有
%$g(x) \mid u_1(x) f_1(x) + \cdots + u_s(x) f_s(x)$.
%\end{Proposition}

\subsection{最大公因式}

\begin{Definition}[因式]
如果$g(x) \mid f(x)$, 则$g(x)$称为$f(x)$的一个\textbf{因式}, $f(x)$称为$g(x)$的一个\textbf{倍式}.
\end{Definition}

\begin{Definition}[公因式]
$K[x]$中,若$c(x) \mid f(x)$且$c(x) \mid g(x)$, 则称$c(x)$是$f(x)$和$g(x)$的一个\textbf{公因式}.
\end{Definition}

