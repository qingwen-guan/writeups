\documentclass[UTF8]{ctexrep}
\usepackage{amsmath}
\usepackage{amssymb}
\usepackage[dvipsnames, svgnames, x11names]{xcolor}  % 一般放得靠前
%\usepackage{xcolor}
%\usepackage{marginnote}
\usepackage{wasysym}
\usepackage{geometry}
\usepackage{ntheorem}
%\usepackage{amsthm}
\usepackage{enumerate}
\usepackage[shortlabels]{enumitem} %https://tex.stackexchange.com/questions/229540/customized-enumerate-items
\usepackage{forest}
\usepackage{hyperref}
\usepackage[normalem]{ulem} % either use this (simple) or
\usepackage{verbatim} % for comment


\usetikzlibrary{decorations.pathreplacing}

%\geometry{a4paper,left=2cm,right=4cm,top=2cm,bottom=1cm,marginparwidth=3cm}
\geometry{a4paper,left=2cm,right=2cm,top=2cm,bottom=1cm}

\theorembodyfont{\upshape}
\newtheorem{Definition}{定义}%[subsection]
\newtheorem{Theorem}[Definition]{定理}

\newtheorem{Remark}[Definition]{注意}
\newtheorem*{RemarkNon}[Definition]{注意}

\newtheorem{Lemma}[Definition]{引理}
\newtheorem{Corollary}[Definition]{推论}
\newtheorem{Proposition}[Definition]{命题}

\newtheorem{Property}[Definition]{性质}
\newtheorem*{PropertyNon}[Definition]{性质}

\newtheorem{Note}[Definition]{说明}
\newtheorem*{NoteNon}[Definition]{说明}

\setenumerate[1]{itemsep=0pt,partopsep=0pt,parsep=\parskip,topsep=0pt}
\setitemize[1]{itemsep=0pt,partopsep=0pt,parsep=\parskip,topsep=0pt}

\DeclareMathOperator{\OpIm}{Im}
%\DeclareMathOperator{\deg}{deg}

%\renewcommand\marginfont{%
%    \scriptsize
%}

%\setlength{\parskip}{0pt}
%  \setlength{\topskip}{0pt}

\newenvironment{tightcenter}{%
  \setlength\topsep{0pt}
  \setlength\parskip{0pt}
  \begin{center}
}{%
  \end{center}
}

\begin{document}

\title{近世代数(抽象代数)笔记}
\author{管清文}
\maketitle
\tableofcontents
\clearpage

% http://blog.sina.com.cn/s/blog_4b3651ad0101nk4o.html
\begin{PropertyNon}[Property]
结果值得一记, 但是没有定理深刻.
\end{PropertyNon}

\begin{RemarkNon}[Remark]
涉及到一些结论,更像是非正式的定理.
\end{RemarkNon}

\begin{NoteNon}[Note]
就是注解.
\end{NoteNon}

\begin{NoteNon} \ \\
\begin{itemize}
	\item 关于一一映射的说法都被改成了双射(Bijection), 因为在英文资料中, one-to-one表示的是单射(Injection), 容易引起歧义.
	\item 所有的当且仅当的命题(定理、$\cdots$)都被写成以下形式: 
	\begin{Proposition}
	 假定blablabla,那么
	\begin{tightcenter}
		$p \Leftrightarrow q$
	\end{tightcenter}
	\end{Proposition}
\end{itemize}
\end{NoteNon}


\setcounter{Definition}{0}

\section{Highlights}

\section{基本概念}


\begin{Note}
近世代数(或抽象代数)的主要内容就是研究所谓\textbf{代数系统},即带有运算的集合。
\end{Note}

\begin{Note}
规定
$-\infty < n, -\infty + n = -\infty, \forall n \in N$
 规定$-\infty + (-\infty) = -\infty$.
\end{Note}

\subsection{映射}

\begin{Definition}[映射]
$$ \begin{aligned}
\phi: A &\rightarrow D \\
a &\mapsto d = \phi (a) = \overline{a} \end{aligned}$$ 
其中$A$称为\textbf{定义域(Domain)}, $D$称为\textbf{陪域(Codomain)}, $\{ \phi(a) \mid a \in A \}$称为\textbf{值域(Image)}, 记作$f(A)$或者$\mathop{\text{Im}}f$.
\end{Definition}

\begin{Definition}
对于映射$f$: $A \rightarrow B$, $g$: $C \rightarrow D$
\begin{tightcenter}
$f = g  \stackrel{\text{def}}{\iff} A = B \And C = D \And f(a) = g(a)~\forall a \in A$
\end{tightcenter}
\end{Definition}

\begin{Definition}[满射]
映射$\phi: A \rightarrow B$被称为\textbf{满射}, 如果$B = \OpIm{\phi}$, 换句话说
\begin{tightcenter}
$\forall b \in B \implies$ $b \in \OpIm{\phi}$~(即$\exists a \in A$ 使得$\phi{(a)} = b$).
\end{tightcenter}
它对应$\phi^{-1}$都有象.
\end{Definition}

\begin{Definition}[单射]
映射$\phi: A \rightarrow \bar{A}$被称为\textbf{单射}, 如果$A$中不同元素的在$\phi$下的象不同, 换句话说
\begin{tightcenter}
设$a_1, a_2 \in A$, $\phi(a_1) = \phi(a_2) \implies a_1 = a_2$
\end{tightcenter}
它对应$\phi^{-1}$象唯一
\end{Definition}

\begin{Definition}[变换]
从$A$到$A$的映射 $\tau: A \rightarrow A, a \mapsto \tau(a)$ 叫$A$\textbf{变换}, 我们也用$a^\tau$表示$\tau(a)$.
 {{如果$\tau$是满射(单射、双射), 则称为}}\textbf{满变换}(\textbf{单变换}、\textbf{双射变换}).
\end{Definition}

%\subsection{映射的乘法、逆映射}

\begin{Definition}[映射的乘法]
设$f$: $A \rightarrow B$, $g$: $B \rightarrow C$, 令
$$
	(g \circ f)(a) \triangleq g(f(a)), \forall a \in A
$$
则称$g \circ f$是$g$与$f$的\textbf{乘积}.
\end{Definition}

\begin{Note}
映射的乘法满足结合律, 即$h \circ (g \circ f) = (h \circ g) \circ f$.
\end{Note}


\begin{Definition}
若$f$: $A \rightarrow A$, $a \mapsto a$, 则$f$是$A$上的\textbf{恒等变换}, 记作$1_A$.
\end{Definition}

\begin{Proposition}设$f$: $A \rightarrow B$, 则$f \circ 1_A = 1_B \circ f = f$
\end{Proposition}

\begin{Definition}[可逆映射]
设$f$: $A \Rightarrow B$, 如果存在$g$: $B \rightarrow A$使得$g \circ f = 1_A$, 且$f \circ g = 1_B$, 那么称$f$是可逆映射, 把$g$称为$f$的逆映射.
\end{Definition}

\begin{Note}
若$f$可逆, 则$f$的逆映射唯一, 把$f$的逆映射记作$f^{-1}$. $f^{-1}$也是可逆映射, 并且$(f^{-1})^{-1} = f$.
\end{Note}

\begin{Theorem}
$f$: $A \rightarrow B$是可逆映射 $\iff$ $f$是双射.
\end{Theorem}

\subsection{等价关系与集合划分}

\begin{Definition}[集合的划分]
如果集合$A$是他的一些非空子集的并集, 其中每两个不相等的子集的交是空集(称为\textbf{不相交}), 那么把这些子集组成的集合称为$A$的一个\textbf{划分}.
\end{Definition}

\begin{Definition}[二元关系\mbox{[Relation]}]
$S \times S$的一个子集$W$称为$S$上的一个二元关系. 若$(a, b) \in W$, 则称$a$和$b$有$W$关系, 记作$a \sim_W b$.
若$(a, b) \notin W$, 则称$a$和$b$没有$W$关系.
%$R: A \times A \rightarrow D = \{\text{对}, \text{错}\} $, 
%若
%$R(a, b) = \text{对}$
%, 称
%$(a, b)$
%满足关系$R$, 记为$a \, R \, b$.
\end{Definition}

\begin{Note}
整除是一个二元关系.
\end{Note}

\begin{Definition}[等价关系]
如果$\sim$是$A$的元素间的关系,满足 
\begin{enumerate}[(1)]
\item 自反性, $\forall a \in A, a \sim a$.
\item 对称性, $\forall a, b \in A$, 若$a \sim b$, 则$b \sim a$.
\item 传递性, $\forall a, b, c \in A$, 若$a \sim b$, $b\sim c$, 则$a \sim c$.
\end{enumerate}
则称$\sim$为\textbf{等价关系}. \index{等价关系}
\end{Definition}

\begin{Definition}
设$~$是$S$上的一个等价关系, 任给$a \in S$, 令
$$
\bar{a} \triangleq \{ x \in S \mid x \sim a\}
$$
则把$\bar{a}$称为$a$的\textbf{等价类}.
\end{Definition}

\begin{Note}
$x \in \bar{a} \iff x \sim a$
\end{Note}

\begin{Note}(代表)
由于$a \sim a$, 因此$a \in \bar{a}$, 把$a$称为$\bar{a}$的一个\textbf{代表}.
\end{Note}

\begin{Property}
$\bar{a} = \bar{b} \iff a \sim b$
\end{Property}

\begin{Property}
$\bar{a} \neq \bar{b} \implies \bar{a} \cap \bar{b} = \emptyset$
\end{Property}

\begin{Theorem}
如果集合$S$上有一个等价关系$\sim$, 那么所有等价类组成的集合是$S$的一个划分.
\end{Theorem}


\begin{Theorem}
如果集合$S$中有一个划分, 那么可以在$S$上建立一个等价关系, 使得这个划分是由所有等价类组成的.
\end{Theorem}

\begin{Definition}[商集]
集合$S$的一个划分也称为$S$的一个\textbf{商集}, 是$S$的所有等价类组成的集合.
\end{Definition}

\begin{Definition}[$\mathbb{Z}_p$\mbox{[模$n$的剩余类]}]
{\color{gray}{
$ \{ [0], [1], \cdots, [n-1] \} $, $[i] = \{ k n + i \mid k \in \mathbb{Z} \}$
}}
\end{Definition}

\subsection{代数运算}


%\begin{Note}
%判断一个法则$\phi$是映射的充要条件: (1) 都有象 (2) 象唯一.
%\end{Note}

\begin{Definition}[代数运算] 
$$\begin{aligned}
A \times B &\rightarrow D \\ 
(a, b) &\mapsto d = \phi(a, b) = \circ (a, b) = a \circ b \end{aligned}$$

\end{Definition}

%{\color{gray}{
%\begin{Note}
%$A = B$时, 对于代数运算$ A \times A \rightarrow D $, $ a \circ b $ 和 $ b \circ a $ 都有意义,但不一定相等.
%\end{Note}
%}}

\begin{Definition}[$A$的代数运算, 二元运算] 
假如 $ \circ $ 是一个 $ A \times A \rightarrow A$的代数运算(即$A = B = D$),我们说集合$A$对于代数运算$\circ$来说是闭的, 也说, $\circ$是\textbf{$A$的代数运算}或\textbf{二元运算}.
\end{Definition}

\begin{Note}[$A$的代数运算判别] 
 \ \\
\begin{center} \begin{forest}
%for tree={circle,draw,s sep=0.8cm}
for tree={grow'=east, parent anchor=east, child anchor=west, anchor=west}
[ { $A$的代数运算 }
	[ 是映射 
		[ 都有象 ]
		[ {象唯一 $\forall a, a', b, b' \in A: a = a', b = b' \Rightarrow a \circ b = a' \circ b'$ }
		]
	]
	[ {封闭 $\forall a, b \in A: a \circ b \in A, $} ]
]
\end{forest} \end{center}

\end{Note}

\subsection{运算律} %%%%%%%%%%%%%%

\begin{Definition}[结合率]
我们说,一个集合$A$的代数运算$\circ$满足结合律,假如对于$A$的任何三个元素$a, b, c$来说都有$
(a \circ b) \circ c = a \circ (b \circ c)
$
\end{Definition}

%\begin{Definition}
%假如对于$A$的$n$ ($n \ge 2$)个固定的元 $a_1, a_2, \cdots, a_n$来说,所有的加括号方式 $\pi(a_1 \circ a_2 \circ \cdots \circ a_n)$都相等,我们就把这些步骤可以得到的唯一的结果,用$a_1 \circ a_2 \circ \cdots \circ a_n $ 来表示.
%\end{Definition}

\begin{Theorem}
若$A$的代数运算$\circ$满足结合律,则对于$A$的任意$n$($n \ge 2$)个元素 $a_1, a_2, \cdots, a_n$来说,对于任意的加括号的方法$\pi$, $\pi(a_1 \circ a_2 \circ \cdots \circ a_n)$ 都相等,我们用$a_1 \circ a_2 \circ \cdots \circ a_n$ 来表示.
\end{Theorem}

\begin{Definition}[交换律]
如果$A$上的代数运算$\circ$满足$\forall a, b \in A: a \circ b = b \circ a$,则称$\circ$满足\textbf{交换律}. 对于$a, b \in A$, 如果$a \circ b = b \circ a$, 则称$a, b$\textbf{可交换}.
\end{Definition}

\begin{Theorem}
若$A$上的代数运算$\circ$满足结合律与交换律,则$a_1 \circ a_2 \circ \cdots \circ a_n$ 可以任意交换顺序.
\end{Theorem}

\begin{Definition}[分配率]
$ \astrosun $和$ \oplus $ 都是 $A$上的代数运算, 
\begin{enumerate}[(1)]
\item 若
$ a \, \astrosun \, (b \oplus c) = (a \, \astrosun \, b) \oplus (a \, \astrosun \, c), \forall a, b, c $, 则称 $ \astrosun$和$ \oplus $  满足左分配率.
\item 若
$ (a \oplus b) \, \astrosun \, c = (a \, \astrosun \, c) \oplus ( b \, \astrosun \, c), \forall a, b, c$, 则称 $ \astrosun$和$\oplus $  满足右分配率.
\end{enumerate}
\end{Definition}

\begin{Theorem}
若$A$上的二元运算$\oplus$ 满足结合律, $ \astrosun $和$\oplus $ 满足左分配率,则
$$
a \, \astrosun \, ( b_1 \oplus b_2 \oplus \cdots \oplus b_n ) =  ( a \, \astrosun \, b_1) \oplus (a \, \astrosun \, b_2) \oplus \cdots \oplus (a \, \astrosun \, b_n)
$$
\end{Theorem}

\begin{Theorem}
若$A$上的二元运算$\oplus$ 满足结合律, $ \astrosun $和$\oplus $ 满足右分配率,则
$$
( a_1 \oplus a_2 \oplus \cdots \oplus a_n ) \, \astrosun \, b =  ( a_1 \, \astrosun \, b) \oplus ( a_2 \, \astrosun \, b) \oplus \cdots \oplus (  a_n \, \astrosun \, b)
$$
\end{Theorem}

\subsection{同态}


%\begin{Definition}[一一映射]
%{\color{gray}{
%既是满射又是单射.
%}}
%\end{Definition}

\begin{Note}[映射判别] \ \\ \begin{center}
\begin{forest}
%for tree={circle,draw,s sep=0.8cm}
for tree={grow'=east, parent anchor=east, child anchor=west, anchor=west}
[ 同构映射, color=red
	[
		同态满射, color=red, edge={dashed, color=red}, name=IsHomoFullMap, tier=1
		[
			同态映射, color=red, edge={dashed, color=red}, name=IsHomoMap, tier=2
			[,phantom]
			[
			{同态 ${a \circ b} \mapsto \bar{a}\; \bar{\circ} \; \bar{b} $}, name=IsHomo, color=red, edge={color=red}, tier=prop
			]
		] 
		[,phantom]
	]
	[ 双射, edge={color=red},tier=1
		[ 满射, name=IsFullMap, tier=2
			[ 是映射, name=IsMap, tier=prop 
				[ 都有象
				]
				[ {象唯一 $a = b \Rightarrow \bar{a} = \bar{b}$ }
				]
			]
			[ 满的, tier=prop ]
		]
		[ 单射, edge=dashed, tier=2
			[ {单的 $ \overline{a}  = \overline{b} \Rightarrow a = b $
			  }
			  , name=IsSingle
			  , tier=prop
			]
		] {
			\draw[-] (.east) to (IsMap.west); % Don't forget semicolon.
		}
	] {
		\draw[-] (.east) to (IsSingle.west); % Don't forget semicolon.	
	}
] {
	\draw[-, red] (.east) to (IsHomo.west); % Don't forget semicolon.		
	\draw[-, red] (IsHomoFullMap.east) to (IsFullMap.west); % 
	\draw[-, red] (IsHomoFullMap.east) to (IsHomo.west); % 
	\draw[-, red] (IsHomoMap.east) to (IsMap.west); % 
}
\end{forest}
\end{center}
\end{Note}


\begin{Definition}[同态映射]
对于$\phi: A \rightarrow \bar{A}$, $A$上有二元运算$\circ$, $\bar{A}$上有二元运算$\bar{\circ}$. 称 $\phi$是 $A$到 $\bar{A}$的\textbf{同态映射}, 如果
$\forall a, b \in A$, $\bar{a} := \phi(a), \bar{b} := \phi(b)$有 
$ a \circ b \mapsto \bar{a} \, \bar{\circ} \, \bar{b}$.
\end{Definition}

%\begin{Note}[同态映射判别]
%(1) 是映射(都有象、象唯一) (2) $ \overline{a \circ b} = \bar{a} \,{\bar{\circ}}\, \bar{b}$
%\end{Note}

\begin{Definition}[同态满射、同态]
如果$A$到$\bar{A}\;${\fbox{存在}}\;一个同态映射$\phi$, 且它是满射, 则称$A$与$\bar{A}$\;(关于$\circ$与$\bar{\circ}$)\textbf{同态}. 称这个映射是一个\textbf{同态满射}.
\end{Definition}

%\begin{Note}[同态满射判别]
%(1) 是映射(都有象、象唯一) (2) 满 (3) 同态 
%\end{Note}

\begin{Definition}[同构映射、同构]
如果$A$到$\bar{A}\;${\fbox{存在}}\;一个同态映射$\phi$, 且它是双射, 则称$A$与$\bar{A}\;$(关于$\circ$与$\bar{\circ}$)\textbf{同构}, 记为$A \cong
 \bar{A}$. 称这个映射是一个(关于$\circ$与$\bar{\circ}$的)\textbf{同构映射}(简称\text{同构}).
\end{Definition}

\begin{Proposition}
同构关系是一个等价关系.
\end{Proposition}

%\begin{Note}[同构映射判别]
%(1) 是映射(都有象、象唯一) (2) 满 (3) 单 (4) 同态
%\end{Note}

\begin{Theorem}
假定对于代数运算$\circ$和$\bar{\circ}$来说, $A$与$\bar{A}$同态, 那么
\begin{enumerate}[(1)]
\item 若 $\circ$ 满足结合律, $\bar{\circ}$也满足结合律;
\item 若 $\circ$ 满足交换律, $\bar{\circ}$也满足交换律.
\end{enumerate}
\end{Theorem}

\begin{Theorem}
$ \astrosun $和$ \oplus $ 是 $A$的两个代数运算, 
$ \bar{\astrosun} $和$\bar{\oplus} $ 是 $\bar{A}$的两个代数运算,
有
$\phi$既是$A$与$\bar{A}$的关于$ \astrosun $和$\bar{\astrosun}$ 的同态满射,
$\phi$也是$A$与$\bar{A}$的关于$ \oplus $和$\bar{\oplus}$ 的同态满射,
则 
\begin{enumerate}[(1)]
\item 若$ \astrosun$和$ \oplus $ 满足第一分配率, 则 $ \bar{\astrosun} $和$\bar{\oplus} $ 也满足第一分配率.
\item 若$ \astrosun$和$\oplus $ 满足第二分配率, 则 $ \bar{\astrosun} $和$\bar{\oplus} $ 也满足第二分配率.
\end{enumerate}
\end{Theorem}

%\begin{Definition}[自同构]
%对于$\circ$和$\circ$来说的一个$A$与$A$之间的\;\fbox{同构映射}\;叫做一个对于$\circ$来说的$A$的\textbf{自同构}.
%\end{Definition}


\chapter{群}

\section{群论}

\subsection{群的定义和性质}

\begin{Remark}
群是一个代数系统(定义代数运算的集合), 它只有一个代数运算, 被称为乘法. 便利起见$(a, b)$的象写成$a b$ %\marginnote{之前写成$a \circ b$}
\end{Remark}

\begin{Definition}[群\mbox{[Group]}的第一定义]
在集合$G \neq \emptyset$上规定一个叫做乘法的\;\fbox{代数运算}\;%\marginnote{以后简称乘法}.
这个代数系统被称为群, 如果
\begin{itemize}
	\item[\uppercase\expandafter{\romannumeral1}] 乘法封闭, $\forall a, b \in G, ab \in G$ %\marginnote{代数运算要求封闭性}
	\item[\uppercase\expandafter{\romannumeral2}] 乘法结合, $\forall a, b, c \in G, (ab)c = a(bc)$
	\item[\uppercase\expandafter{\romannumeral3}] $ \forall a, b \in G$, $ax = b, ya = b$在$G$中都有解.
\end{itemize}
\end{Definition}

\begin{Theorem}[左单位元]
对于群$G$中至少有一个元$e$, 叫做$G$的一个\textbf{左单位元},使得$\forall a \in G$都有 $ea = a$.
\end{Theorem}

\begin{Theorem}[左逆元]
对于群$G$中的任何一个元素$a$, 在$G$中存在一个元$a^{-1}$,叫做$a$的\textbf{左逆元}, 能让$a^{-1} a = e$.
\end{Theorem}

\begin{Definition}[群\mbox{[Group]}的第二定义]
在集合$G \neq \emptyset$上规定乘法. 这个代数系统被称为\textbf{群}, 如果
\begin{itemize}
	\item[\uppercase\expandafter{\romannumeral1}] 乘法封闭
	\item[\uppercase\expandafter{\romannumeral2}] 乘法结合
	\item[IV] 左单位元: $\exists e \in G$使 $ea =a$ 对 $\forall a \in G$都成立.
	\item[V] 左逆元: $\forall a \in G, \exists \, a^{-1}$使$a^{-1}a = e$.
\end{itemize}
\end{Definition}

\begin{Definition}[群的阶]
如果$|G|$有限, 称其为\textbf{有限群}, 称他的\textbf{阶}是$G$的元素个数. 
如果$G$中有无穷多个元素, 称其为\textbf{无限群}, 称他的\textbf{阶}无限.
\end{Definition}

\begin{Definition}[交换群、Abel群]
群中交换律不一定成立, 如果乘法满足交换律($\forall a, b \in G, ab = ba$), 则称之为\textbf{交换群}(\textbf{Abel群}).
\end{Definition}

\begin{Theorem}[单位元]
在一个群$G$里存在且只存在一个元$e$, 使得$ea = ae = a$对于$\forall a \in G$成立. 这个元素被称为群$G$的\textbf{单位元}.
\end{Theorem}

\begin{Theorem}[逆元]
对于群$G$的任意一个元素$a$来说, 有且只有一个元素$a^{-1}$, 
使 $a^{-1} a = a a^{-1} = e$. 这个元素被称为$a$的\textbf{逆元}, 或者简称\textbf{逆}.
\end{Theorem}

\begin{Note}
证明$a^{-1}$是$a$的逆的方法: $a^{-1}a = e$或者$aa^{-1} = e$ (不用都说明).
\end{Note}

\begin{Property}[乘积的逆等于逆的乘积]
$\forall a, b \in G, {(ab^{-1})}^{-1} = b a^{-1}$
\end{Property}

\begin{Definition}
规定$\forall n \in \mathbb{Z}^{+}: a^n = \underbrace{a a \cdots a}_{n\text{个}}, a^0 = e, a^{-n} = (a^{-1})^n$
\end{Definition}

\begin{Proposition}
$ \forall n, m \in \mathbb{Z}: a^n a^m = a^{n+m}, {(a^n)}^m = a^{mn} $ ({$\Rightarrow {(a^{-1})}^{-1} = a$})
\end{Proposition}

\begin{Definition}[元素的阶]
在一个群$G$中,使得$a^n = e$的最小正整数, 叫做$a$的\textbf{阶}. 若这样的$n$不存在, 称$a$是无穷阶的,或者叫$a$的阶是无穷.
\end{Definition}

%\begin{Theorem}
%假定群的元$a$的阶是$n$, 则$a^r$的阶是$\displaystyle \frac{n}{\text{gcd}(r, n)}$.
%\end{Theorem}

\begin{Theorem}[III'\mbox{[消去律]}]
群的乘法满足: $ax = ax' \Rightarrow x = x', ya = y'a \Rightarrow y = y'$
\end{Theorem}

\begin{Corollary}
在群里, $ax = b$ 和 $ya = b$都有唯一解.
\end{Corollary}

\begin{Theorem}[有限群的另一定义]
一个带有乘法的 \fbox{有限集合}  $G \neq \emptyset$, 若满足I、II、III', 则$G$是一个\textbf{群}.
\end{Theorem}

\subsection{群的同态}

\begin{Theorem}
$G$与$\bar{G}$关于他们的乘法同态, 则 $G$是群$\Rightarrow \bar{G}$也是群.
\end{Theorem}

\begin{Theorem}
假定$G$和$\bar{G}$是两个群, 在$G$到$\bar{G}$的一个同态满射之下, $G$的单位元$e$的象是$\bar{G}$的单位元, $G$的元$a$的逆元$a^{-1}$的象是$a$的象的逆元($\overline{a^{-1}} = \bar{a}^{-1}$).
\end{Theorem}

\begin{Theorem}
$G$与$\bar{G}$关于他们的乘法同构, 则 $G$是群$\Leftrightarrow \bar{G}$是群.
\end{Theorem}

\subsection{变换群} %%%%%%%%%%%%%%%%%%

\begin{Definition}[变换的乘法]
$\tau_1 \tau_2: a \mapsto {\left(a^{\tau_1}\right)}^{\tau_2}$
\end{Definition}

\begin{Theorem}[变换乘法结合]
$(\tau_1 \tau_2) \tau_3 = \tau_1 ( \tau_2 \tau_3 )$
\end{Theorem}

\begin{Theorem}
$G$是集合$A$的若干变换构成的集合, 如果$G$基于变换的乘法做成一个群,
则$G$中的变换一定是双射变换.
\end{Theorem}

\begin{Definition}[变换群]
如果一个集合$A$的若干\;\fbox{双射变换}\;对于变换的乘法能够做成一个群,则称这个群为A的一个\textbf{变换群}.
\end{Definition}

\begin{Theorem}
一个集合$A$上的所有双射变换做成一个变换群$G$.
\end{Theorem}

\begin{Theorem}
任何一个群都与一个变换群同构.
\end{Theorem}

\begin{Theorem}
一个变换群的单位元一定是恒等变换.
\end{Theorem}

\subsection{置换群} %%%%%%%%%%%%%%%%%%%%%%%%%%%%%%%%%%%%%%%%%%%%%%%%

\begin{Definition}[置换]
\fbox{有限集合}\;上的\;\fbox{双射变换}\;叫做\textbf{置换}, 一般用$\pi$表示.
\end{Definition}

\begin{Definition}[置换群]
有限集合上的若干置换做成的群叫\textbf{置换群}.
\end{Definition}

\begin{Definition}[对称群]
一个$n$元集合$A = \{ a_1, a_2, \cdots, a_n \}$上的所有置换
(有$n!$个)做成的群叫做$n$次\textbf{对称群}, 用$S_n$来表示.
\end{Definition}

\begin{Theorem}
$$
\left.
\begin{aligned}
\pi_1 &= \begin{pmatrix} 
j_1       & \cdots & j_k       & j_{k+1} & \cdots & j_n \\
j_1^{(1)} & \cdots & j_k^{(1)} & j_{k+1} & \cdots & j_n \\
\end{pmatrix} \\
\pi_2 &= \begin{pmatrix} 
j_1       & \cdots & j_k       & j_{k+1}       & \cdots & j_n      \\
j_1       & \cdots & j_k       & j_{k+1}^{(2)} & \cdots & j_n^{(2)} \\
\end{pmatrix} 
\end{aligned}
\right\}
\Rightarrow
\pi_1 \pi_2 = \begin{pmatrix} 
j_1       & \cdots & j_k       & j_{k+1}       & \cdots & j_n \\
j_1^{(1)} & \cdots & j_k^{(1)} & j_{k+1}^{(2)} & \cdots & j_n^{(2)} \\
\end{pmatrix}
$$
\end{Theorem}

\begin{Definition}[$k$-循环置换]
如果$S_n$中的置换满足$a_{i_1}$的象是$a_{i_2}$, $a_{i_2}$的象是$a_{i_3}$, $\cdots$, 
$a_{i_{k-1}}$的象是$a_{i_k}$, $a_{i_{k}}$的象是$a_{i_1}$, 其他元素,如果还有的话,象是不变的, 则称之为\textbf{$k$-循环置换}.
用$(i_1 \, i_2 \, i_3 \, \cdots \, i_{k-1} \, i_k)$ 或 
$(i_2 \, i_3 \, \cdots \, i_{k-1} \, i_k \, i_1)$ 或 $\cdots$ 或   
$( i_k \, i_1 \, i_2 \, i_3 \, \cdots \, i_{k-1})$来表示.
\end{Definition}

\begin{Proposition}
$(i_1 \, i_2 \, \cdots i_k)^{-1} = (i_k \, \cdots \, i_2 \, i_1)$.
\end{Proposition}

\begin{Proposition}
$k$-循环置换的阶是$k$.
\end{Proposition}

\begin{Proposition}
任何一个置换都可以写成若干没有共同数字的循环置换的乘积.
\end{Proposition}

\begin{Proposition}
两个没有共同数字的循环置换可以交换.
\end{Proposition}

\begin{Proposition}
任何一个有限群都与一个置换群同构.
\end{Proposition}

\subsection{循环群}

\begin{Definition}[循环群]
若一个群$G$的每一个元都是$G$的某一固定元$a$的乘方, 我们就称$G$是一个\textbf{循环群}, $a$是$G$的一个\textbf{生成元}, 并记$G = (a)$, 且说$G$是由元$a$生成的。
\end{Definition}

\begin{Definition}[$\mathbb{Z}_n$\mbox{[模$n$的剩余类加群]}]
$G$包含所有模$n$的剩余类,$G = \{ [0], [1], \cdots, [n-1] \}$, 定义乘法(叫做加法) $[a] + [b] = [a +b]$,可以证明$(G, +)$做成一个群, 叫做\textbf{模$n$的剩余类加群}.
\end{Definition}

\begin{Theorem}
假定$G$是由$a$生成的循环群, 则$G$的构造可以完全由$a$的阶来决定:
\begin{itemize}
\item 如果$a$的阶无限, 则$G \cong \mathbb{Z}$.
\item 如果$a$的阶为$n$, 则$G \cong \mathbb{Z}_n$.
\end{itemize}
\end{Theorem}

\begin{Note}
于是$|(a)| = n$, 其中$n$为$a$的阶.
\end{Note}

\begin{Proposition}
一个循环群一定是交换群.
\end{Proposition}

\begin{Proposition}
$a$生成一个阶是$n$的循环群$G$, 则$a^r$也生成$G$,如果$\text{gcd}(r, d) = 1$.
\end{Proposition}

\begin{Proposition}
$G$是循环群, 且$G$与$\bar{G}$同态,则$\bar{G}$也是循环群.
\end{Proposition}

\begin{Proposition}
$G$是无限阶循环群, $\bar{G}$是任何循环群, 则$G$与$\bar{G}$同态.
\end{Proposition}

\subsection{子群}

\begin{Definition}[子群]
如果一个群$G$的一个子集$H$关于群$G$的乘法也能做成一个群,则称$H$为$G$的一个\textbf{子群}.
\end{Definition}

\begin{Theorem}
一个群$G$的一个非空子集$H$做成$G$的子群,当且仅当
\begin{enumerate}[(i)]
\item $a, b \in H \Rightarrow ab \in H$
\item $a \in H \Rightarrow a^{-1} \in H$
\end{enumerate}
\end{Theorem}

\begin{Corollary}
若$H$是$G$的子群, 则, $H$的单位元就是$G$的单位元, $a$在$H$中的逆就是$a$的$G$中的逆.
\end{Corollary}

\begin{Theorem}
一个群$G$的一个非空子集$H$做成$G$的子群,当且仅当 (iii) $a, b \in H \Rightarrow ab^{-1} \in H$
\end{Theorem}

\begin{Theorem}
一个群$G$的一个非空\;\fbox{有限}\;子集$H$做成$G$的子群,当且仅当 (i) $a, b \in H \Rightarrow ab \in H$
\end{Theorem}

\begin{Note}[验证非空集合是群的方法]
(1) I, II, III (2) I、II、IV, V (3) 有限集: I, II, III'
(4) 子群: (i), (ii) \mbox{(5) 子群: (iii)} (6) 有限子群: (i)
\end{Note}

\begin{Definition}[生成子群]
对于群$G$的非空子集$S$, 包含$S$的最小子群, 被称为由$S$生成的子群, 记为$(S)$.
\end{Definition}

\begin{Theorem}
$S = \{a\}$时, $(S) = (a)$.
\end{Theorem}

%\begin{Proposition}
%群$G$的两个子群的交集也是$G$的子群.
%\end{Proposition}

\begin{Proposition}
循环群的子群也是循环群.
\end{Proposition}

\begin{Proposition}
$H$是群$G$的一个非空子集, 且$H$的每个元素的阶都有限,
则$H$做成子群的充要条件是 (i) $a, b \in H \Rightarrow ab \in H$.
\end{Proposition}

\subsection{子群的陪集}

\begin{Definition}
群$G$, 子群$H$, 规定$G$上的关系$\sim: a \sim b \Leftrightarrow a b^{-1} \in H$
\end{Definition}

\begin{Theorem}
上面规定的关系$\sim$是等价关系.
\end{Theorem}

\begin{Definition}[右陪集]
由上述等价关系确定集合的分类叫做$H$的\textbf{右陪集}.
\end{Definition}

\begin{Theorem}
包含元$a$的右陪集$ = Ha =  \{ ha \mid h \in H \}$
\end{Theorem}

\begin{Definition}
群$G$, 子群$H$, 规定$G$上的关系$\sim': a \sim 'b \Leftrightarrow b^{-1} a \in H$. 可以证明$\sim'$是等价关系.
\end{Definition}

\begin{Definition}[左陪集]
由上述等价关系$\sim': a \sim' b \Leftrightarrow b^{-1} a \in H$, 确定集合的分类叫做$H$的\textbf{左陪集}, 包含元$a$的左陪集可以用$aH = \{ ah \mid h \in H \}$表示.
\end{Definition}

\begin{Theorem}
一个子群的右陪集与左陪集个数相等: 个数或者都是无穷大, 或者都有限且相等.
\end{Theorem}

\begin{Definition}[指数]
一个群$G$的一个子群$H$的右陪集(或左陪集)的个数叫做$H$在$G$里的指数.
\end{Definition}

\begin{Theorem}
右陪集所含元素的个数等于子群$H$所含元素的个数.
\end{Theorem}

\begin{Theorem}
$H$是一个有限群$G$的子群, 那么$H$的阶$n$和他在$G$中的指数$j$都能整除$G$的阶$N$, 并且$N = nj$
\end{Theorem}

\begin{Theorem}[元素的阶整除群的阶]
一个有限群$G$的任何一个元$a$的阶能够整除$G$的阶$|G|$.
\end{Theorem}

\begin{Proposition}
阶是素数的群一定是循环群.
\end{Proposition}

\begin{Proposition}
阶是$p^m$的群($p$是素数)一定包含一个阶是$p$的子群.
\end{Proposition}

\begin{Proposition}
若我们把同构的群看做一样的,一共只存在两个阶是4的群,它们都是交换群.
\end{Proposition}

%\begin{Proposition}
%有限非交换群至少有6个元素.
%\end{Proposition}

\subsection{不变子群、商群}

\begin{Definition}[不变子群]
群$G$的子群$N$叫做$G$的\textbf{不变子群}, 如果$\forall a \in G$, 有$Na = aN$. 一个不变子群$N$的一个左(或右)陪集叫做$N$的一个\textbf{陪集}.
\end{Definition}

\begin{Definition}
$S_1, S_2 \subseteq $群$G$, 规定子集的乘法
$S_1 S_2  = \{s_1 s_2 \mid s_1 \in S_1, s_2 \in S_2 \}$. 显然这个乘法满足结合律.
\end{Definition}

\begin{Theorem}
已知一个群$G$有一个子群$N$, $N$是不变子群的充要条件是$aNa^{-1} = N, \forall a \in G$.
\end{Theorem}

\begin{Theorem}
已知一个群$G$有一个子群$N$, $N$是不变子群的充要条件是
$a \in G, n \in N \Rightarrow ana^{-1} \in N $.
\end{Theorem}

\begin{Theorem}
如果$N$刚好包含$G$的所有具有以下性质的元$n$,
$$
	na = an, \forall a \in G
$$
则$N$是$G$的不变子群. 我们称这个不变子群是G的\textbf{中心}.
\end{Theorem}

\begin{Theorem}
$N$是群$G$的不变子群, 在其陪集$\{ aN, bN, cN, \cdots \}$上定义的乘法$(xN, yN) \mapsto (xy)N$, 则这个乘法是此陪集的二元运算,且此陪集对于上面规定的乘法来说构成一个群.
\end{Theorem}

\begin{Definition}[商群]
一个群$G$的一个不变子群$N$的所有陪集关于陪集的乘法做成的群叫做$G$的\textbf{商群},用$G/N$表示.
\end{Definition}

\begin{Theorem}
对于有限群, $ \displaystyle | G/N | $ = $\frac{|G|}{|N|}$.
\end{Theorem}

%\begin{Proposition}
%两个不变子群的交集还是不变子群.
%\end{Proposition}

\begin{Proposition}
$H$是$G$的子群, $N$是$G$的不变子群, 则$HN$是$G$的子群.
\end{Proposition}

\subsection{同态与不变子群}

\begin{Theorem}
一个群$G$与它的商群$G/N$同态.
\end{Theorem}

\begin{Definition}[核]
$\phi$是群$G$到群$\bar{G}$的一个同态满射, $\bar{G}$的单位元$\bar{e}$在$\phi$之下的所有原象做成的$G$的子集叫做$\phi$的\textbf{核}.
\end{Definition}

\begin{Theorem}
$G$和$\bar{G}$是两个群,且$G$与$\bar{G}$同态,则这个同态满射的核$N$是$G$的一个不变子群,且$G/N \cong \bar{G}$.
\end{Theorem}

\begin{Remark}
一个群只和``相当于''它的商群同态
\end{Remark}

\begin{Definition}
$\phi$是$A \rightarrow \bar{A}$的满射, 取$S \subseteq A$, 定义$S$的象是$S$中所有元素的象做成的集合. 取$\bar{S} \subseteq \bar{A}$, 定义$\bar{S}$的原象是$\bar{S}$中所有元素的原象做成的集合.
\end{Definition}

\begin{Theorem}
$G$和$\bar{G}$是两个群,且$G$与$\bar{G}$同态,则在这个同态满射之下:
\begin{itemize}
\item[(1)] $G$的一个子群$H$的象$\bar{H}$也是$\bar{G}$的一个子群.
\item[(2)] $G$的一个不变子群$N$的象$\bar{N}$也是$\bar{G}$的一个不变子群.
\item[(1')] $\bar{G}$的一个子群$\bar{H}$的原象$H$也是$G$的一个子群.
\item[(2')] $\bar{G}$的一个不变子群$\bar{N}$的原象$N$也是$G$的一个不变子群.
\end{itemize}
\end{Theorem}

\begin{Remark}
这也体现了同态的性质,前面有的后面也有!
\end{Remark}

\begin{Proposition}
假定群$G$与群$\bar{G}$同态, $\bar{N}$是$\bar{G}$的不变子群, $N$是$\bar{N}$的逆象,则
$ G/N \sim \bar{G}/\bar{N} $.
\end{Proposition}

\begin{Proposition}
假定群$G$与$\bar{G}$是两个有限循环群,他们的阶各是$m$和$n$,则$G$与$\bar{G}$同态$\Leftrightarrow n \mid m$
\end{Proposition}

\begin{Proposition}
假定群$G$是一个循环群, $N$是$G$的一个子群, 则$G/N$也是循环群.
\end{Proposition}

\section{特殊的群}

\subsection{加群}

\begin{Definition}[加群]
一个交换群叫做一个的\textbf{加群}, 如果我们把这个群的代数运算称为加法, 并且用符号$+$表示.
\end{Definition}

\begin{Definition}[$\Sigma$]
{{
$n$个元的和$a_1 + a_2 + \cdots + a_n$用符号$\displaystyle \sum_{i=1}^{n} a_n$ 来表示.
}
}
\end{Definition}

\begin{Definition}
$n$个$a$的和$\displaystyle \sum_{i=1}^{n} a$我们用$na$表示.
\end{Definition}

\begin{Definition}[零元]
加群唯一的单位元用$\mathfrak{0}$来表示, 并且把它叫做\textbf{零元}.
\end{Definition}

\begin{Definition}[负元]
元$a$的唯一的逆元我们用$-a$来表示,并且把它叫做$a$的\textbf{负元}. $a + (-b)$我们简写成$a - b$.
\end{Definition}

\begin{Theorem}
加群满足以下运算规则
\begin{enumerate}[(1)]
\item $\mathfrak{0} + a = a + \mathfrak{0} = a$
\item $-a + a = a - a = \mathfrak{0}$
\item $-(-a) = a$
\item[(4: 移项)] $a + c = b \Leftrightarrow c = b - a$
\item $-(a +b) = -a - b, -(a-b) = -a +b$
\item $ma + na = (m+n)a, m(na) = (mn)a, n(a+b) = na + nb, \forall m, n \in \mathbb{Z}^+$
\end{enumerate}
\end{Theorem}

\begin{Note}
非空子集$S$做成子群的充要条件变成了 
\begin{itemize}
\item (i) $a, b \in S \Rightarrow a+b \in S$ (ii) $a \in S \Rightarrow -a \in S$
\item 或者 (iii) $a, b \in S \Rightarrow a - b \in S$.
\end{itemize}
\end{Note}


\section{环与域}

\subsection{加群、环的定义}

\begin{Definition}[加群]
一个交换群叫做一个的\textbf{加群}, 如果我们把这个群的代数运算称为加法, 并且用符号$+$表示.
\end{Definition}

\begin{Definition}[$\Sigma$]
{{
$n$个元的和$a_1 + a_2 + \cdots + a_n$用符号$\displaystyle \sum_{i=1}^{n} a_n$ 来表示.
}
}
\end{Definition}

\begin{Definition}
$n$个$a$的和$\displaystyle \sum_{i=1}^{n} a$我们用$na$表示.
\end{Definition}

\begin{Definition}[零元]
加群唯一的单位元用$\mathfrak{0}$来表示, 并且把它叫做\textbf{零元}.
\end{Definition}

\begin{Definition}[负元]
元$a$的唯一的逆元我们用$-a$来表示,并且把它叫做$a$的\textbf{负元}. $a + (-b)$我们简写成$a - b$.
\end{Definition}

\begin{Theorem}
加群满足以下运算规则
\begin{enumerate}[(1)]
\item $\mathfrak{0} + a = a + \mathfrak{0} = a$
\item $-a + a = a - a = \mathfrak{0}$
\item $-(-a) = a$
\item[(4: 移项)] $a + c = b \Leftrightarrow c = b - a$
\item $-(a +b) = -a - b, -(a-b) = -a +b$
\item $ma + na = (m+n)a, m(na) = (mn)a, n(a+b) = na + nb, \forall m, n \in \mathbb{Z}^+$
\end{enumerate}
\end{Theorem}

\begin{Note}
非空子集$S$做成子群的充要条件变成了 
\begin{itemize}
\item (i) $a, b \in S \Rightarrow a+b \in S$ (ii) $a \in S \Rightarrow -a \in S$
\item 或者 (iii) $a, b \in S \Rightarrow a - b \in S$.
\end{itemize}
\end{Note}

%\begin{Definition}[环]
%一个集合$R$叫做一个环, 如果
%\begin{enumerate}
%	\item $R$是一个加群: $R$关于一个叫做加法的代数运算做成一个交换群.
%	\item $R$对于另一个叫做乘法的代数运算是封闭的.
%	\item $R$关于乘法结合
%	\item 分配率: $a(b+c) = bc + ac, (a+b)c = ac + bc$
%\end{enumerate}
%\end{Definition}

\begin{Theorem}
环还满足以下运算规则
\begin{enumerate}[(1)]
\item[(7)] $(a-b)c = ac - bc, c(a - b) = ca - cb$
\item[(8)] $\mathfrak{0}a = a\mathfrak{0} = \mathfrak{0}$
\item[(9)] $(-a)b = a(-b) = -(ab)$
\item[(10)] $(-a)(-b) = ab$
\item[(11)] $a(b_1 + b_2 + \cdots + b_n) = ab_1 +ab_2 + \cdots + ab_n, (b_1 + b_2 + \cdots + b_n)a = b_1a + b_2a + \cdots + b_na$
\item[(12)] 
$ \displaystyle \left( \sum_{i=1}^m a_i \right) \left( \sum_{j=1}^n b_j \right) 
= \sum_{a=1}^m \sum_{b=1}^{n} a_i b_j $
           $$
\begin{aligned}
(a_1 + a_2 + \cdots + a_m) (b_1 + b_2 + \cdots + b_n) 
= 
a_1b_1 + a_1 b_2 + &\cdots + a_1 b_n \\
+ a_2 b_1 + a_2 b_2 + &\cdots + a_2 b_n \\
+ &\cdots \\
+ a_m b_1 + a_m b_2 + &\cdots + a_m b_n
\end{aligned}
 $$
 \item[(13)] $ (na)b = a(nb) = n(ab), n \in \mathbb{Z}^+$ 
 \item[(14)] 规定 $a^n = \underbrace{a a \cdots a}_{n\text{个}}, n \in \mathbb{Z}^{+}$, 则 $ a^m a^n = a^{m+n}, (a^m)^n = a^{mn} $
\end{enumerate}
\end{Theorem}

\subsection{交换律、单位元、零因子、整环}

\begin{Definition}[交换环]
一个环$R$叫做\textbf{交换环}, 如果$ab = ba,  \forall a, b \in R$.
\end{Definition}

\begin{Proposition}
在一个交换环中${(ab)}^n = a^n b^n$.
\end{Proposition}

\begin{Definition}[单位元]
对于环$R$, 如果$ea = ae = a, \forall a \in R$, 则称$e$是环$R$的单位元. 一般,一个环未必有单位元.
\end{Definition}

\begin{Proposition}
一个环如果有单位元, 则唯一. 用$\mathfrak{1}$来表示.
\end{Proposition}

\begin{Definition}[整数环]
整数关于普通加法和乘法构成的环.
\end{Definition}

\begin{Definition}[逆元]
若$ba = \mathfrak{1}$, 则称$b$为$a$的\textbf{左逆元}. 若$ba = ab = \mathfrak{1}$, 则称$b$为$a$的\textbf{逆元}. 
\end{Definition}

\begin{Proposition}
如果$a$有逆元, 则唯一.
\end{Proposition}

\begin{Proposition}
如果$a$有逆元, 则规定$a^{-m} = {(a^{-1})}^m, a^0 = \mathfrak{1}$. 则$a^m a^n = a^{m+n}, {(a^m)}^n = a^{mn}, \forall m, n \in \mathbb{Z}$.
\end{Proposition}

\begin{Proposition}[模$n$的剩余类环]
$R = \{ [0], [1], \cdots, [n-1] \}$, 加法: $[a] + [b] = [a+b]$, 乘法: $[a][b] = [ab]$做成一个交换环, 被称为\textbf{模$n$的剩余类环}, 零元$\mathfrak{0} = [0]$, 单位元$\mathfrak{1} = [1]$.
\end{Proposition}

\begin{Proposition}
$ab = \mathfrak{0} \Rightarrow a = \mathfrak{0} $ 或者 $b = \mathfrak{0}$ 在环里不一定对.
\end{Proposition}

\begin{Definition}[零因子]
在一个环$R$中, 若$a \neq \mathfrak{0}, b \neq \mathfrak{0}$但$ab = \mathfrak{0}$, 则称$a$是$R$的\textbf{左零因子}, $b$是$R$的\textbf{右零因子}.
\end{Definition}

\begin{Remark}
左零因子不一定是右零因子. 但是如果有左零因子, 就一定有右零因子. 如果$R$是交换环, 则左零因子一定是右零因子.
\end{Remark}

\begin{Theorem}
在一个没有零因子的环里, 两个消去律都成立.
\begin{enumerate}
	\item $a \neq \mathfrak{0}, ab = ac \Rightarrow b = c$
	\item $a \neq \mathfrak{0}, ba = ca \Rightarrow b = c$
\end{enumerate}
反过来, 在一个环里如果\;\fbox{有一个}\;消去律成立,那么这个环没有零因子.
\end{Theorem}

\begin{Corollary}
在一个环$R$中如果有一个消去律成立,那么另一个消去律也成立.
\end{Corollary}

\begin{Definition}[整环]
一个环$R$叫做一个\textbf{整环}, 如果
\begin{enumerate}
	\item 乘法适合交换律: $ab = ba$.
	\item $R$有单位元1: $\mathfrak{1}a = a\mathfrak{1} = a$.
	\item $R$没有零因子: $ab = \mathfrak{0} \Rightarrow a = \mathfrak{0}$或$b = \mathfrak{0}$
\end{enumerate}
\end{Definition}

\begin{Note}[!整环的判别] \ \\ \begin{tightcenter}
\begin{forest}
%for tree={circle,draw,s sep=0.8cm}
for tree={grow'=east, parent anchor=east, child anchor=west, anchor=west}
	[ 整环, name=IntRing, tier=2
		[ 环, name={Ring}, tier=property
			[ 加群
				[ 加法成群: {I II IV V} ]
				[ 加法交换 ]
			]
			[ 乘法: I II
			]
			[ 两个分配律
			]
		]
		[乘法交换, name={ProductCommutative}, tier=property]
		[有$\mathfrak{1}$, name={Has1}, tier=property]
		[无零因子, name=NonZeroFact, tier=property]
	]
\end{forest}
\end{tightcenter}
\end{Note}

\begin{Proposition}
整数环是一个整环.
\end{Proposition}

\begin{Proposition}
对于有单位元的环来说, 加法适合交换律是环定义里其他条件的结果.
\end{Proposition}

%\begin{Proposition}
%二项式定理$\displaystyle (a + b)^n = \binom{n}{0} a^n + \binom{n}{1} a^{n-1}b + \binom{n}{2} a^{n-2}b^2 + \cdots + \binom{n}{n} b^n$在交换环中成立.
%\end{Proposition}

%\begin{Proposition}
%$\{ a + b \sqrt{2} \mid a, b \in \mathbb{Z}\}$对于普通加法和乘法来说是一个整环.
%\end{Proposition}

\subsection{除环、域} %%%%%%%%%%%%%%%%%%%%%

\begin{Proposition}
对于元素个数$\ge 2$的环, $\mathfrak{1} \neq \mathfrak{0}$, 且$\mathfrak{0}$没有逆元.
\end{Proposition}

\begin{Definition}[除环]
一个环$R$叫做一个\textbf{除环}, 如果
\begin{enumerate}
	\item $R$至少含有一个不等于零的元.
	\item $R$有单位元.
	\item $R$的任何一个非零元都有逆.
\end{enumerate}
\end{Definition}

\begin{Definition}[域]
一个交换除环叫做一个\textbf{域}.
\end{Definition}

\begin{Property}
除环没有零因子.
\end{Property}

\begin{Property}
除环$R$的所有非零元对于乘法来说做成一个群$R^*$, 我们把$R^*$叫做\textbf{除环$R$的乘群}.
\end{Property}

\begin{Note}
对于一个环$R$来说, 从$R^*$是对于乘法做成一个群, 也能推出$R$是除环.
\end{Note}

\begin{Note}
在除环$R$中, 方程$ax = b, ya = b (a \neq \mathfrak{0})$都有唯一解, 分别是$a^{-1}b$和$ba^{-1}$, 他们未必相等. 在一个域里$a^{-1}b = ba^{-1}$, 用符号
$\displaystyle \frac{b}{a}$表示.
\end{Note}

\begin{Property}
域满足以下计算法则
\begin{enumerate}
	\item $\displaystyle \frac{a}{b} = \frac{c}{d} \Leftrightarrow ad = bc$.
	\item $\displaystyle \frac{a}{b} + \frac{c}{d} = \frac{ad + bc}{bd}$.
	\item $\displaystyle \frac{a}{b} \frac{c}{d} = \frac{ac}{bd}$.
\end{enumerate}
\end{Property}

\begin{Proposition}
%$R = 
%\left\{ (\alpha, \beta) \mid \alpha, \beta \in \mathbb{C} \right\}, 
%(\alpha_1, \beta_1) + (\alpha_2, \beta_2) = 
%	(\alpha_1 + \alpha_2, \beta_1 + \beta_2), 
%(\alpha_1, \beta_1)(\alpha_2, \beta_2) = 
%	(\alpha_1 \alpha_2 - \beta_1 \overline{\beta_2})
%	(\alpha_1 \beta_2 + \beta_1 \overline{\alpha_2})
%$ 做成一个除环, 叫做\textbf{四元数除环}, 它不是交换环(所以不是域).
存在不是域的除环, 例如\textbf{四元数除环}.
\end{Proposition}

\begin{Note} !环、整环、域之间的关系: \begin{center}
\begin{forest}
%for tree={circle,draw,s sep=0.8cm}
for tree={grow'=east, parent anchor=east, child anchor=west, anchor=west}
[,phantom
	[ 整环, name=IntRing, tier=2
		[ 环, name={Ring}, tier=property
			[ 加群
				[ 加法成群: {I II IV V} ]
				[ 加法交换 ]
			]
			[ 乘法: I II
			]
			[ 两个分配律
			]
		]
		[乘法交换, name={ProductCommutative}, tier=property]
		[有$\mathfrak{1}$, name={Has1}, tier=property]
		[无零因子, name=NonZeroFact, tier=property]
	]
	[ 域
		[ 除环, for tree={edge={color=Blue}}, tier=2
			[有非零元, tier=property, for tree={edge={color=red}}]
			[非零元有逆, tier=property, for tree={edge={color=red}}] {
				\draw[->, color=DarkGreen] () to[out=east, in=east] (NonZeroFact);
			}
		] {
			\draw[-, color=red] (.east) to (Ring.west);	
			\draw[-, color=red] (.east) to (Has1.west);
		}
	] {
		\draw[-, color=Blue] (.east) to (ProductCommutative.west); % Don't forget semicolon.
		\draw[-, color=Blue, dotted] (.east) to (IntRing.west); % Don't forget semicolon.
	}
]
\end{forest}
\end{center}
\end{Note}

\begin{Proposition}
一个至少有两个元且没有零因子的有限环,是一个除环.
\end{Proposition}

\subsection{无零因子环的特征}  %%%%%%%%%%%%%%%%%%%%%

\begin{Proposition}
对于模$p$的剩余类环$\mathbb{Z}_p$, $p$是素数 $\Leftrightarrow $ $\mathbb{Z}_p$做成一个域.
\end{Proposition}

\begin{Proposition}
在一个环$R$里, 对于加法的阶, 可能有的元素是无限的,有的元素是有限的.
\end{Proposition}

\begin{Theorem}
在一个无零因子环中, 所有非零元素$R$对于加法的阶都相同: 要么都无限大, 要么都有限且相等.
\end{Theorem}

\begin{Definition}[无零因子环的特征]
在一个无零因子环$R$中, 所有非零元关于加法的阶,叫做$R$的\textbf{特征}.
\end{Definition}

\begin{Theorem}
如果无零因子环$R$的特征是一个有限整数$n$, 则$n$一定是素数.
\end{Theorem}

\begin{Corollary}
整环、除环以及域的特征或者是无限大,或者是一个素数.
\end{Corollary}

%\begin{Proposition}
%{\color{gray}{一个的特征是$p$的交换环里, $(a+b)^p = a^p +b^p$.}}
%\end{Proposition}

\subsection{子环、环的同态} %%%%%%%%%%%%

\begin{Definition}[子环]
一个环$R$的非空子集$S$如果对于R的代数运算来说也是环(整环、除环、域), 则称$S$是$R$的一个\textbf{子环}(\textbf{子整环}、\textbf{子除环}、\textbf{子域}).
\end{Definition}

\begin{Theorem}
若$S$是环$R$的一个非空子集, 则$S$是$R$的子环 的充要条件是 $a, b \in S \Rightarrow a-b \in S, ab \in S$.
\end{Theorem}

\begin{Theorem}
若$S$是整环$R$的一个非空子集, 则$S$是$R$的子整环 的充要条件是 (1)$a, b \in S \Rightarrow a-b \in S, ab \in S; (2) \mathfrak{1} \in S$.
\end{Theorem}

\begin{Theorem}
若$S$是除环$R$的一个非空子集, 则$S$是$R$的子除环 的充要条件是 (1) $S$有非零元; (2) $a, b \in S \Rightarrow a-b \in S$; (3) $\forall a, b \in S, b \neq 0 \Rightarrow ab^{-1} \in S$.
\end{Theorem}

\begin{Theorem}
若$S$是域$R$的一个非空子集, 则$S$是$R$的子域 的充要条件是 (1) $S$有非零元; (2) $a, b \in S \Rightarrow a-b \in S$; (3) $\forall a, b \in S, b \neq 0 \Rightarrow ab^{-1} \in S$.
\end{Theorem}

\begin{Proposition}
环$R$的可以同每个元交换的元做成一个j交换子环$N = \{ n \mid a n = na, \forall a \in R\}$, 这个子环称为$R$的中心.
\end{Proposition}

\begin{Theorem}
若$R$是环, $R$到$\bar{R}$有一个满射使得对于两个运算都同态, 则$\bar{R}$也是一个环.
\end{Theorem}

\begin{Remark}
总结下来, 如果$A$与$\bar{A}$同态,那么前面有什么后面就也有什么:
\begin{itemize}
	\item 前面有结合,后面就也有结合
	\item 前面有交换,后面就也有交换
	\item 前面有分配,后面就也有分配
	\item 前面是群,后面就也是群
	\item 前面是环,后面就也是环
\end{itemize}
\end{Remark}

\begin{Theorem}
若$R$和$\bar{R}$都是环, 且$R$与$\bar{R}$同态, 则
\begin{itemize}
	\item $R$的零元的象是$\bar{R}$的零元.
	\item $R$的元$a$的负元的象是$a$的象的负元{($\overline{-a} = -\overline{a}$)}
	\item $R$是交换环 $\Rightarrow$ $\bar{R}$也是交换环
	\item $R$有单位元$\mathfrak{1}$ $\Rightarrow$ $\bar{R}$也有单位元$\bar{\mathfrak{1}}$, 且$\bar{\mathfrak{1}}$是$\mathfrak{1}$的象.
	\item $R$无零因子 $\not\Rightarrow$ $\bar{R}$无零因子
	\item $R$有零因子 $\not\Rightarrow$ $\bar{R}$有零因子
	\item $R$是整环(除环、域) $\not\Rightarrow$ $\bar{R}$是整环(除环、域)
\end{itemize}
\end{Theorem}

\begin{Proposition}
若$R$和$\bar{R}$都是环, 且$R$与$\bar{R}$同态, 则
\begin{itemize}
	\item $R$无零因子 $\not\Rightarrow$ $\bar{R}$无零因子
	\item $R$有零因子 $\not\Rightarrow$ $\bar{R}$有零因子
	\item $R$是整环(除环、域) $\not\Rightarrow$ $\bar{R}$是整环(除环、域)
\end{itemize}
\end{Proposition}

\begin{Proposition}
$R$与$\bar{R}$都是环, 且$R \cong \bar{R}$, 则
\begin{itemize}
	\item $R$无零因子 $\Leftrightarrow$ $\bar{R}$无零因子.
	\item $R$有非零元 $\Leftrightarrow$ $\bar{R}$有非零元.
	\item  $R$非零元有逆 $\Leftrightarrow$ $\bar{R}$非零元有逆
\end{itemize}
\end{Proposition}

\begin{Theorem}
$R$与$\bar{R}$都是环, 且$R \cong \bar{R}$, 则
\begin{itemize}
	\item $R$是整环 $\Leftrightarrow$ $\bar{R}$是整环.
	\item  $R$是除环 $\Leftrightarrow$ $\bar{R}$是除环.
	\item  $R$是域 $\Leftrightarrow$ $\bar{R}$是域.
\end{itemize}
\end{Theorem}

\begin{Lemma}
集合$A$和$\bar{A}$之间有一个双射$\phi$, 并且$A$有加法和乘法, 于是我们可以在$\bar{A}$中规定加法和乘法,使得$A$与$\bar{A}$关于一对加法和一对乘法来说都同构.
\end{Lemma}

\begin{Theorem}
假定$S$是环$R$的一个子环, $S$在$R$中的补集($R$ - $S$)与另一个环$\bar{S}$没有公共元,并且$S \cong \bar{S}$, 那么
存在一个与$R$同构的环$\bar{R}$,且$\bar{S}$是$\bar{R}$的子环.
\end{Theorem}

\newcommand{\StubSize}{0.5em}%
%\newcommand{\BraceForest}[1][]
%\usetikzlibrary{decorations, decorations.pathmorphing} % in the preamble


\begin{Note}[!]
\ \\
\begin{center}
\begin{forest}
for tree={grow=west, parent anchor=west, child anchor=east, anchor=east}
[, s sep=2em,phantom
	[ 环$S$,
		  name=RingS 
		[环R	 ,edge label={node [midway, above, font=\scriptsize] {子环}} 	
		]
	]		
	[ {环$\bar{S}$}, name=RingBarS
		[ $?$ ,edge label={node [midway, below, font=\scriptsize] {子环}} 
		]	
	]
	[
		{$ (R-S) \cap \bar{S} = \emptyset $}
	]
] {
	\draw[<->%, out=-130,in=-85
	] (RingS.south) %.. controls +(south west: 0.5em) and +(north west: 0.5em) .. 
	--(RingBarS.north) node [midway, right] {$\cong_\phi$}
	;
	\draw[decorate, decoration={brace, amplitude=1.5em}]
     ([yshift=-.5em]current bounding box.north east) --
      node[right=2em%, font=\sffamily\bfseries\Large
      ]
        {
        %\begin{matrix}
         {$\Rightarrow \exists$ 环$? = \bar{R}: \left\{ \begin{aligned}
        	&\bar{R} = {(R-S)} \cup \bar{S} \\ 
       	    &
       	    	\forall \bar{x}, \bar{y} \in \bar{R}: 
       	    	\begin{aligned}
       	    	& \bar{x} + \bar{y} = \psi(x + y), 
       	    	\bar{x} \bar{y} = \psi(x y), \\
       	    	& x = \psi^{-1}(\bar{x}), y = \psi^{-1}(\bar{y})
       	    	\end{aligned}
        	 \\ 
       	   % }
        	&R \cong \bar{R}, \psi: x \mapsto \begin{cases} x & x \in R-S \\ \phi{(x)} & x \in S \end{cases}
        \end{aligned} \right.$} 
%        {subject to $R \cong \bar{R}, \psi: x \mapsto \begin{cases} a \\ b\end{cases}$}
        %\end{matrix}
        }
      ([yshift=-.5em]current bounding box.south east)
   ;	
}
%\BraceForest[red,ultra thick]%
  % inside environment forest
\end{forest}
\end{center}
\end{Note}

\begin{Proposition}
一个除环的中心是一个域.
\end{Proposition}

\subsection{多项式环} %%%%%%%%%%%%%%%
\begin{Note}
假定$R_0$是一个有单位元的交换环, $R$是$R_0$的子环, 并且包含$R_0$的单位元. 取$x \in R_0$, 则
$ \displaystyle \sum_{i=0}^n a_i x^i = a_0 + a_1 x + a_2 x^2 + \cdots + a_n x^n\; (a_i \in R) $有意义, 且$\in R_0$.
\end{Note}

\begin{Definition}[多项式]
一个可以写成 $ a_0 + a_1 x + a_2 x^2 \cdots + a_n x^n, a_i \in R, n \in \mathbb{Z}^+ $形式的$R_0$的元叫做$R$上的关于$x$的一个\textbf{多项式}, $a_i$叫做多项式的\textbf{系数}. 我们把所有$R$上的$x$的多项式放在一起, 做成一个集合,用$R[x]$来表示.
\end{Definition}

\begin{Note}[环上的多项式构成一个环]
在$R[x]$上定义
$$
\begin{aligned}
\text{加法: }& \displaystyle \sum a_i x^i + \sum b_i x^i = \sum (a_i + b_i) x^i \\
\text{乘法: }& \displaystyle \left(\sum_{i=0}^m a_i x^i \right) \left( \sum_{j=0}^{n} b_j x^j \right) 
= \sum_{i=k}^{mn} \left(\sum_{i+j=k} a_i b_{j} \right) x^k
\end{aligned}
$$
 都为初等代数里的计算方法, 则$R[x]$构成一个交换环.
\end{Note}

\begin{Definition}[未定元]
$R_0$里得一个元$x$叫做$R$上的一个\textbf{未定元}, 如果在$R$里找不到不都等于零的元
$a_0, a_1, a_2, \cdots, a_n$, 使得$a_0 + a_1 x + a_2 x^2 \cdots + a_n x^n = 0$
\end{Definition}

\begin{Proposition}
$R$上的一个未定元x的多项式(简称\textbf{一元多项式}),如果不计入系数是零的项, 只能用一种方式写成
$a_0 + a_1 x + a_2 x^2 + \cdots + a_n x^n \; (a_i \in R)$
\end{Proposition}

\begin{Definition}[多项式的次数]
令$a_0 + a_1 x + a_2 x^2 \cdots + a_n x^n = 0, a_n \neq \mathfrak{0}$是环$R$上的一个一元多项式,
那么非负整数$n$叫做这个多项式的\textbf{次数}, 多项式$\mathfrak{0}$没有次数.
\end{Definition}

\begin{Proposition}
对于给定的$R_0$来说, $R_0$未必含有$R$上的未定元.
\end{Proposition}

\begin{Theorem}
给了一个有单位元的交换环$R$, 一定有一个环$R_0$, $R$上的未定元$x \in R_0$存在,因此也就有$R$上的多项式环$R[x]$存在.
\end{Theorem}

\begin{Note}
对于一个有单位元的交换环$R_0$, 和它的一个子环$R$, 其中$R$包含$R_0$的单位元. 我们从$R_0$里任意
取出$n$个元$x_1, x_2, \cdots, x_n$来, 那么我们可以做$R$上的$x_1$的多项式环$R[x_1]$,然后做$R[x_1]$上的$x_2$的多项式环$R[x_1][x_2]$. 这样下去,可以得到$R[x_1][x_2]\cdots[x_n]$. 这个环包括所有可以写成 
$\displaystyle \sum_{i_1 i_2 \cdots i_n} a_{i_1 i_2 \cdots i_n} x_1^{i_1} x_2^{i_2} \cdots x_n^{i_n}$ ($a_{i_1 i_2 \cdots i_n} \in R$
, 但只有有限个$a_{i_1 i_2 \cdots i_n} \neq 0$)形式的元.
\end{Note}

\begin{Definition}
一个有上述形式的元叫做$R$上的$x_1, x_2, \cdots, x_n$的一个多项式,$a_{i_1 i_2 \cdots i_n}$叫做多项式的系数. 环$R[x_1][x_2]\cdots[x_n]$叫做$R$上的$x_1, x_2, \cdots, x_n$的多项式环. 这个环我们也用符号$R[x_1, x_2, \cdots, x_n]$来表示.
\end{Definition}

\begin{Proposition}
假定$R$是一个整环, 那么$R$上的一元多项式环也是一个整环.
\end{Proposition}

\subsection{理想} %%%%%%%%%%%%%%%%%%%%%%%%%%%%

\begin{Definition}[!理想]
环$R$的一个非空子集$I$叫做一个\textbf{理想子环}(简称\textbf{理想}), 如果
\begin{enumerate}
	\item $a, b \in I \Rightarrow a - b \in I$
	\item $a \in I, r \in R \Rightarrow ra, ar \in I$.
\end{enumerate}
\end{Definition}

\begin{Proposition}
一个环至少有两个理想 (1) $I = \{\mathfrak{0}\}$, 叫做$R$的\textbf{零理想}.
(2) I = R, 叫做$R$的\textbf{单位理想}.
\end{Proposition}

\begin{Theorem}
一个除环$R$只有两个理想,就是零理想和单位理想.
\end{Theorem}

\begin{Note}
因此,理想这个概念对于除环或者域来说没有多大用处.
\end{Note}

\begin{Note}
一个环除了以上两个理想之外,可能有其他理想.
\end{Note}

\begin{Proposition}
给定一个环$R$,$a$是$R$中的任意一个元素,考虑最小的理想$I$使得$a \in I$. 作集合
$I = \{ \left( x_1 a y_1 + x_2 a y_2 + \cdots \right) + s a + a t + na \mid
x_i, y_i, s, t \in R, n \in \mathbb{Z} \}$, 则$I$是包含$a$的最小理想.
\end{Proposition}

\begin{Definition}[主理想]
上面的这样的$I$叫做元$a$生成的\textbf{主理想}, 用符号$(a)$来表示.
\end{Definition}

\begin{Note}
一个主理想$(a)$的元的形式并不是永远像上面那样复杂.
\begin{enumerate}
	\item 当$R$满足交换律时, 可以写成$ra + na, r \in R, n \in \mathbb{Z}$.
	\item 当$R$有单位元时, 可以写成$ \displaystyle \sum x_i a y_i, x_i, y_i \in R $.

	\item 当$R$既满足交换律又有单位元时, 可以写成$ \displaystyle ra, r \in R $.
\end{enumerate}
\end{Note}

\begin{Proposition}
给定一个环$R$,$a_1, a_2, \cdots, a_m \in R$,考虑最小的理想$I$使得$a_1, a_2, \cdots, a_m \in I$.
做集合$I = \{ s_1 + s_2 + \cdots + s_m \mid s_i \in (a_i) \}$, 
则$I$是包含$a_1, a_2, \dots, a_m$的最小理想.
\end{Proposition}

\begin{Definition}
上面的这样的$I$叫做$a_1, a_2, \cdots, a_m$生成的理想, 用符号$(a_1, a_2, \cdots, a_m)$来表示.
\end{Definition}

\begin{Note}
两个元素生成的理想, 可能是主理想,也可能不是.
\end{Note}

\begin{Proposition} \ \\
\begin{itemize}
\item 群$G$的两个子群的交集还是$G$的子群.
\item 两个不变子群的交集还是不变子群.
\item 两个子环的交集还是子环.
\item 两个子整环的交集还是子整环.
\item 两个子除环的交集还是子除环.
\item 两个子域的交集还是子域.
\item 两个理想的交集还是一个理想.
\end{itemize}
\end{Proposition}

\subsection{剩余类环、同态与理想} %%%%

\begin{Note}
给定一个环$R$和$R$的一个理想$I$, 则我们就加法来说, $R$做成一个群, $I$做成$R$的一个不变子群, 从而$I$的陪集$[a], [b], [c], \cdots$做成$R$的一个分类, 叫做\textbf{模$I$的剩余类}. 同时这个分类描述$R$的元素之间的等价关系, 用符号$a \equiv b \text{ mod } I$表示(读作$a$同余$b$模$I$), 即$a \equiv b \text{ mod } I \Leftrightarrow a \sim b \Leftrightarrow a - b \in I$. 且类$[a]$所包含的元素可以写成$\{ a + u \mid u \in I \}$
\end{Note}

\begin{Theorem}
假定$R$是一个环, $I$是它的一个理想, $\bar{R}$是所有模$I$的剩余类做成的集合, 如果在$\bar{I}$上规定加法和乘法$[a] + [b] = [a + b], [a][b] = [ab]$.那么$\bar{I}$本身也是一个环, 并且$R$与$\bar{R}$同态.
\end{Theorem}

\begin{Definition}[模$I$的剩余类环]
上面的$\bar{R}$叫做环$R$的\textbf{模$I$的剩余类环}, 用符号$R/I$来表示.
\end{Definition}

\begin{Theorem}[!]
假定$R$与$\bar{R}$是两个环, 并且$R$与$\bar{R}$同态, 那么这个同态满射的核$I$是$R$的一个理想, 并且$R/I \cong \bar{R}$
\end{Theorem}

\begin{Theorem}
在环$R$到环$\bar{R}$的同态满射下:
\begin{enumerate}[(1)]
	\item $R$的一个子环的象$\bar{S}$是$\bar{R}$的一个子环.
	\item $R$的一个理想$I$的象$\bar{I}$是$\bar{R}$的一个理想.
	\item $\bar{R}$的一个子环$\bar{S}$的原象$S$是R的一个子环.
	\item $\bar{R}$的一个理想$\bar{I}$的原象$I$是$R$的一个理想.
\end{enumerate}

\end{Theorem}

\begin{Note}
	环-群, 子环-子群, 理想-不变子群
\end{Note}

\begin{Proposition}
$\phi$是环$R$到环$\bar{R}$的一个同态满射: $\phi$是$R$与$\bar{R}$之间的同构映射 $\Leftrightarrow$ $\phi$的核是零理想.
\end{Proposition}

\subsection{最大理想}

\begin{Definition}[最大理想]
如果一个环$R$的理想$I (\neq R)$, 除了$R$和$I$以外, 无其他包含$I$的理想,称$I$为$R$的\textbf{最大理想}.
\end{Definition}

\begin{Lemma}
假定$I (\neq R)$是环$R$的一个理想: 剩余类环$R/I$除了零理想和单位理想外不再有其他理想 $\Leftrightarrow$ $I$是最大理想.
\end{Lemma}

\begin{Lemma}
若有单位元($\neq \mathfrak{0}$)的交换环$R$除了零理想和单位理想以外没有其他理想, 那么$R$一定是一个域.
\end{Lemma}

\begin{Theorem}[!]
$R$是有单位元的交换环, $I (\neq R)$是$R$的理想: $R/I$是域 $\Leftrightarrow$
$I$是$R$的最大理想.
\end{Theorem}

\begin{Proposition}
$\mathbb{Z}_n$是域$\Leftrightarrow n$是素数. 
\end{Proposition}

\subsection{商域}

\begin{Theorem}
若$R$是无零因子的交换环, 则存在一个包含$R$的域$Q$, 使得$Q$刚好是由所有元$\displaystyle \frac{a}{b} \; (a, b \in R, b \neq \mathfrak{0} )$所做成的,这里$\displaystyle \frac{a}{b} = ab^{-1} = b^{-1}a$.
\end{Theorem}

\begin{Definition}[商域]
一个域$Q$叫做环$R$的一个\textbf{商域}, 如果$Q \supseteq R$, 并且$Q$刚好是由所有元$\displaystyle \frac{a}{b} \; (a, b \in R, b \neq 0)$所做成的.
\end{Definition}

\begin{Theorem}
假定$R$是一个有两个以上的元的环, $F$是一个包含$R$的域,则$F$包含$R$的一个商域.
\end{Theorem}

\begin{Note}
一般来讲, 一个环很可能有两个以上的商域. 不过,同构的环的商域也同构,所以抽象的来讲,一个环最多只有一个商域.
\end{Note}

\section{线性空间}

\chapter{套娃}

\section{数域$K$上的一元多项式环$K[x]$} %%%%%%%%%%%%%%%%%

\begin{Note}
数域$K$上的一元多项式环$K[x]$是一个欧式环、主理想环、唯一分解环、整环.
\end{Note}

\begin{Property}
设$f(x), g(x) \in K[x]$, 则
\[
\begin{aligned}
\deg \Big( f(x) + g(x) \Big) &\le \max \Big( \deg f(x), \deg g(x) \Big) \\
\deg \Big( f(x) g(x) \Big) &= \deg f(x) + \deg g(x) 
\end{aligned}
\]
\end{Property}

\begin{Corollary}
设$f(x), g(x) \in K[x]$, 则
\[
f(x) \neq 0, g(x) \neq 0 \implies f(x) g(x) \neq 0
\]
\end{Corollary}

\begin{Corollary}[消去律]
设$f(x), g(x), h(x) \in K[x]$, 则
\[
f(x) g(x) = f(x) h(x) \And f(x) \neq 0 \implies g(x) = h(x)
\]
\end{Corollary}


\begin{Note}
任给$A \in M_n(K)$, 矩阵$A$的多项式组成的集合$K[A]$. 容易验证非空集合$K[A]$对于矩阵的减法和乘法封闭, 从而$K[A]$是环$M_n(K)$的一个子环. 且$K[A]$是有单位元的交换环.
\end{Note}

\begin{Note}
$KI$是$K[A]$的一个子环.
\end{Note}

\begin{Theorem}[一元多项式环的通用性质]
设$K$是一个数域, $R$是一个有单位元$1'$的交换环. 且$K$到$R$的一个子环$R_1$(含有$1'$)
有一个同构映射$\tau$. 任给$t \in R$, 令
\[
\begin{aligned}
\sigma_t: K[X] &\rightarrow R \\
          f(x) = \sum\limits_{i=0}^n a_i x^i &\mapsto \sum\limits_{i=0}^n \tau(a_i) t^i \triangleq f(t)
\end{aligned}
\]
则$\sigma_t$是$K[x]$到$R$的一个映射, 且$\sigma_t(x) = t$, 且$\sigma_t$保持加法、乘法运算, 即
\[
f(x) +g(x) = h(x), f(x)g(x) = p(x) \implies f(t) + g(t) = h(t), f(t) g(t) = p(t)
\]
称$\sigma_t$是$x$用$t$带入.
\end{Theorem}

\begin{Note}定理回答了为什么多项式环的未定元可以带入特定的值.
\begin{center}
\begin{forest}
for tree={grow=west, parent anchor=west, child anchor=east, anchor=east}
[, s sep=2em,phantom
	[ 数域$K$, name=FieldK
		[{$K[x]$} , tier=x, name=Kx, edge label={node [midway, above, font=\scriptsize] {子环}} 	
		]
	]		
	[ {含幺子环${R_1}$}, name=R1
		[ {含幺交换环$R$} ,tier=x, name=R, edge label={node [midway, below, font=\scriptsize] {子环}} 
		]
	]
] {
	\draw[->%, out=-130,in=-85
	] ([xshift=-1em]FieldK.south east)
	--([xshift=-1em]R1.north east) node [midway, left] {$\tau$}  node [midway, right] {$\cong$}
	;
	\draw[->
	] ([xshift=-1em]Kx.south east)
	--([xshift=-1em]R.north east) node [midway, left] {$\sigma$}
	;
	\draw[decorate, decoration={brace, amplitude=1.5em}]
     ([yshift=-.5em]current bounding box.north east) --
      node[right=2em%, font=\sffamily\bfseries\Large
      ]{{
      	$\implies  \left\{ \begin{aligned}
      		& \begin{aligned}
      		\forall t \in R, \exists \sigma_t: K[X] &\rightarrow R \\
          		f(x) = \sum\limits_{i=0}^n a_i x^i &\mapsto \sum\limits_{i=0}^n \tau(a_i) t^i \triangleq f_\tau(t) \\
          	\end{aligned} \\
          	& \text{满足} \\
          	& 
          		\quad \sigma_t(x) = t \\
          	& \quad
          		f(x) + g(x) = h(x) \Rightarrow f_\tau(t) + g_\tau(t) = h_\tau(t) \\
          	& \quad
          		f(x) g(x) = p(x) \Rightarrow f_\tau(t) g_\tau(t) = p_\tau(t) \\
        \end{aligned} \right.$
      }}
      ([yshift=-.5em]current bounding box.south east)
   ;	
}
%\BraceForest[red,ultra thick]%
  % inside environment forest
\end{forest}
\end{center}

\subsection{整除关系, 带余除法}

\end{Note}
\begin{Proposition}
在$K[x]$中, 如果$g(x) \mid f(x)$, 其中$f(x) \neq 0$, 则$\deg g(x) \le \deg f(x)$.
\end{Proposition}

\begin{Proposition}
若$f(x), g(x) \in K[x]$, 那么
\begin{tightcenter}
$f(x) \sim g(x) \iff f(x) = c g(x)~(c \in K^*)$
\end{tightcenter}
\end{Proposition}

\begin{Theorem}[带余除法]
设$f(x), g(x) \in K[x]$, 且$g(x) \neq 0$, 则\;\fbox{唯一}\;的存在$K[x]$中一对多项式$h(x), r(x)$,
使得$f(x) = h(x) g(x) + r(x)$, $\deg r(x) < \deg g(x)$. 我们把$h(x)$叫做\textbf{商式},
$r(x)$叫做\textbf{余式}.
\end{Theorem}

\begin{Note}
因为证明用到逆元, 所以要是至少是除环(我还没确定够不够), 普通环肯定不行.
\end{Note}

\begin{Corollary}
设$f(x), g(x) \in K[x]$, 且$g(x) \neq 0$, 则
\begin{tightcenter}
$g(x) \mid f(x)$ $\iff$ $g(x)$除$f(x)$的余式是$0$.
\end{tightcenter}
\end{Corollary}

\begin{Proposition}[整除性不随数域的扩大而改变]
设$f(x), g(x) \in K[x]$, $g(x) \neq 0$, 数域$E$包含$K$, 则
\begin{tightcenter}
在$K[x]$中$g(x) \mid f(x)$ $\iff$ 在$E[x]$中$g(x) \mid f(x)$ 
\end{tightcenter}
\end{Proposition}

%\begin{Proposition}
%在$K[x]$中, 若$g(x) \mid f_i (x)$~($i=1,\cdots, s$), 则对任意$u_1(x), \cdots, u_s(x)$, 都有
%$g(x) \mid u_1(x) f_1(x) + \cdots + u_s(x) f_s(x)$.
%\end{Proposition}

\subsection{最大公因式}

\begin{Definition}[因式]
如果$g(x) \mid f(x)$, 则$g(x)$称为$f(x)$的一个\textbf{因式}, $f(x)$称为$g(x)$的一个\textbf{倍式}.
\end{Definition}

\begin{Definition}[公因式]
$K[x]$中,若$c(x) \mid f(x)$且$c(x) \mid g(x)$, 则称$c(x)$是$f(x)$和$g(x)$的一个\textbf{公因式}.
\end{Definition}


\section{整环里的因子分解}

\subsection{素元、唯一分解}

\begin{Definition}[!整除]
对于整环$I$,若$a \in I$, 存在$b, c \in I$使$a=bc$, 则称$b$能\textbf{整除}$a$, 记作$b \mid a$, 称$b$为$a$的\textbf{因子}. 若$b$不是$a$的因子,则记作$b \nmid a$.
\end{Definition}

\begin{Proposition}[整除具有传递性]
$a \mid b, b \mid c \Rightarrow$ $a \mid c$.
\end{Proposition}

\begin{Definition}[单位]
整环$I$的元$\epsilon$叫做$I$的一个\textbf{单位}, 如果$\epsilon$有逆. (整环里面随便一个可逆的元都叫做一个单位)
\end{Definition}


\begin{Theorem}
$\epsilon_1$和$\epsilon_2$是单位$\Rightarrow$ $\epsilon_1 \epsilon_2$是单位; $\epsilon$是单位$\Rightarrow$ $\epsilon^{-1}$ 也是单位.
\end{Theorem}

\begin{Proposition}
$a$和$b$不是单位$\Rightarrow$ $a b$不是单位.
\end{Proposition}

\begin{Definition}[相伴元]
元$b$叫做元$a$的\textbf{相伴元},如果存在一个单位$\epsilon$使得$b = \epsilon a$.
\end{Definition}

\begin{Note}
相伴元对应的关系是一个等价关系.
\end{Note}

\begin{Definition}[平凡因子、真因子]
$\forall a \in $整环$I$, 所有的单位以及$a$的相伴元,叫做$a$的\textbf{平凡因子}. 其余的$a$的因子,如果还有的话,叫做$a$的\textbf{真因子}.
\end{Definition}

\begin{Definition}[素元]
一个整环$I$的一个元$p$叫做一个\textbf{素元}, 如果$p$ (1) 既不是零元,(2)也不是单位,并且 (3) $p$只有平凡因子.
\end{Definition}

\begin{Theorem}
$p$是素元, $\epsilon$是单位$\Rightarrow$ $\epsilon p$也是素元.
\end{Theorem}

\begin{Theorem}[!!]
若$I$是整环, $a \in I$, $a \neq \mathfrak{0}$,则 $a$有真因子 $\Leftrightarrow$ 
$\exists\, b,c$都不是单位使得$a = bc$, .
\end{Theorem}

\begin{Corollary}
$a\neq 0$, $a$有真因子$b$\;($a = bc$) $\Rightarrow$ $c$也是$a$的真因子.
\end{Corollary}

\begin{Note}
$a$有真因子 $\Rightarrow$ $\exists$ $b, c$为$a$的真因子使得$a = bc$
\end{Note}

\begin{Definition}[唯一分解]
我们说,$a \in $整环$I$, 在$I$里有\textbf{唯一分解}, 假如以下条件都能被满足
\begin{enumerate}[(1)]
	\item[(i)] 能分解: $a = p_1 p_2 \cdots p_r$ ($p_i$是$I$的素元).
	\item[(ii)] 若同时 $a = q_1 q_2 \cdots q_s$ ($q_i$是$I$的素元), 则$r=s$, 且我们可以把$q_i$的次序调换,使得$q_i = \epsilon_i p_i$ ($\epsilon$是$I$的单位)。
\end{enumerate}
\end{Definition}

\begin{Note}
若$a$在环$I$中有唯一分解, 则$a \neq \mathfrak{0}$ 且 $a$不是单位.
\end{Note}

\begin{Note}
一个整环的$\neq \mathfrak{0}$也不是单位的元,不一定都有唯一分解.
\end{Note}

\begin{Proposition}
$\mathfrak{0}$不是任何元的真因子.
\end{Proposition}

\begin{Proposition}
定义$I$为所有可以写成 $\displaystyle \frac{m}{2^n}, m \in \mathbb{Z}, n \in \mathbb{N}$形式的有理数, 则$I$是整环, 其单位是所有等于$2^n, n \in \mathbb{Z}$的数.
\end{Proposition}

\subsection{唯一分解环} %%%%%%%%%%%%%

\begin{Definition}[唯一分解环]
一个整环$I$叫做一个\textbf{唯一分解环}, 如果$I$的每一个既$\neq \mathfrak{0}$也不是单位的元,都有唯一分解.
\end{Definition}

\begin{Theorem}
一个唯一分解环有以下性质, 
\begin{itemize}
	\item[(iii)] 素元$p \mid ab$ $\Rightarrow$ $p \mid a$或$p \mid b$.
\end{itemize}
\end{Theorem}

\begin{Theorem}
如果一个整环$I$满足:
\begin{itemize}
	\item[(i)] $\forall \; a \in I$, $a \neq 0$, $a$不是单位, 都可以写成
	$a = p_1 p_2 \cdots p_r$ ($p_i$是素元). 
	\item[(iii)] 素元$p \mid ab$ $\Rightarrow$ $p \mid a$或$p \mid b$. 
\end{itemize}
则整环$I$是一个唯一分解环.
\end{Theorem}

\begin{Definition}[公因子]
元$c$叫做元$a_1, a_2, \cdots, a_n$的\textbf{公因子},如果$c$能同时整除$a_1, a_2, \cdots, a_n$. 元$a_1, a_2, \cdots, a_n$的一个公因子$d$叫做$a_1, a_2, \cdots, a_n$的\textbf{最大公因子}, 如果$d$能被$a_1, a_2, \cdots, a_n$的每个公因子整除.
\end{Definition}

\begin{Theorem}
若$I$是一个唯一分解环, $a, b \in I$, 则$a, b$在$I$里一定有最大公因子. 若$d, d'$都是$a, b$的最大公因子, 则它们只差一个单位因子: $d' = \epsilon d$ ($\epsilon$是单位).
\end{Theorem}

\begin{Corollary}
一个唯一分解环$I$的$n$个元$a_1, a_2, \cdots, a_n$在$I$里一定有最大公因子, $a_1, a_2, \cdots, a_n$的两个最大公因子只能差一个单位因子.
\end{Corollary}

\begin{Definition}[互素]
我们说, 一个唯一分解环的元$a_1, a_2, \cdots, a_n$互素,如果他们的最大公因子是单位.
\end{Definition}

\begin{Proposition}
假定在一个唯一分解环里
$a_1 = d b_1, a_2 = d b_2, \cdots a_n = d b_n$($d \ne 0$), 我们有
\begin{tightcenter}
$d$是$a_1, a_2, \cdots, a_n$的最大公因子 $\Leftrightarrow$ $b_1, b_2, \cdots, b_n$互素.
\end{tightcenter}
\end{Proposition}

%\begin{Proposition}
%假定$I$是一个整环, $(a), (b)$是$I$的两个主理想, 那么
%\begin{tightcenter}
%$(a) = (b)$ $\Leftrightarrow$ $b$是$a$的相伴元.
%\end{tightcenter}
%\end{Proposition}

\subsection{主理想环} %%%%%%%%%%%%

\begin{Definition}[主理想环]
一个整环$I$叫做一个\textbf{主理想环}, 如果$I$的每一个理想都是一个主理想.
\end{Definition}

\begin{Lemma}
假定$I$是一个主理想环, 若存在序列$a_1, a_2, \cdots$ $(a_i \in I)$的每一个元素都是前面一个元素的真因子, 则这个序列一定是一个有限序列.
\end{Lemma}

\begin{Lemma}[!!]
假定$I$是一个主理想环, $p$是$I$的一个素元, 则$p$生成的理想$(p)$一定是$I$的最大理想.
\end{Lemma}

\begin{Theorem}
一个主理想环$I$一定是一个唯一分解环.
\end{Theorem}

\begin{Proposition}
假定$I$是一个主理想环, 并且$(a, b) = (d)$, 那么$d$是$a$和$b$的一个最大公因子, 因此$a$和$b$的任何一个最大公因子$d'$都可以写成$d' = sa +tb$\;($s, t \in I$)的形式.
\end{Proposition}

\begin{Proposition}
一个主理想环的非零最大理想都是由一个素元所生成的.
\end{Proposition}

\begin{Proposition}
两个主理想环$I$和$I_0$, $I_0$是$I$的子环, $a$和$b$是$I_0$的两个元,$d$是这两个元在$I_0$里得一个最大公因子, 则$d$也是这两个元在$I$里的最大公因子.
\end{Proposition}

\subsection{欧氏环} %%%%%%%%%%%%%

\begin{Definition}[欧氏环]
一个整环$I$叫做一个\textbf{欧氏环}, 如果
\begin{itemize}
	\item 有一个从$I$的非零元所做成的集合到$\ge 0$的整数集合的映射$\phi$存在.
	\item 给定一个$I$的非零元$a$, 则$I$的任何元$b$都可以写成$b = aq +r$ ($q, r \in I$)的形式,这里或者$r = 0$, 或者$\phi(r) < \phi(a)$.
\end{itemize}
\end{Definition}

\begin{Theorem}
任何欧氏环$I$一定是是主理想环, 从而一定是一个唯一分解环.
\end{Theorem}

\begin{Note}
整数环是一个欧式环, 从而是一个主理想环,因而是一个唯一分解环.
\end{Note}

\begin{Lemma}
假定$I[x]$是整环$I$上的一个一元多项式环, $I[x]$的元$g(x) = a_n x^n + a_{n-1} x^{n-1} +\cdots + a_0$的最高系数$a_n$是$I$的一个单位. 那么$I[x]$的任意多项式$f(x)$都可以写成
$f(x) = q(x) g(x) + r(x)$ ($q(x), r(x) \in I[x]$)的形式, 这里或者$r(x) = 0$或者$r(x)$的次数小于$g(x)$的次数$n$.
\end{Lemma}

\begin{Theorem}[!]
一个域$F$上的一元多项式环$F[x]$是一个欧式环. ($\Rightarrow F[x]$是主理想环$\Rightarrow F[x]$是唯一分解环).
\end{Theorem}

\subsection{多项式环的因式分解} %%%%%%%%%%%%%%%%%%%%

\begin{Definition}
一个素多项式(多项式环的素元)叫做\textbf{不可约多项式}, 一个有真因子的多项式叫做\textbf{可约多项式}.
\end{Definition}

\begin{Proposition}
$I$的单位是$I[x]$仅有的单位.
\end{Proposition}

\begin{Definition}[本原多项式]
$I[x]$的一个元$f(x)$叫做一个\textbf{本原多项式}, 如果$f(x)$的系数的最大公因子是单位.
\end{Definition}

\begin{Proposition}
一个本原多项式$\neq \mathfrak{0}$.
\end{Proposition}

\begin{Proposition}
若本原多项式$f(x)$可约, 则$f(x) = g(x) h(x)$, 这里$f(x)$和$g(x)$的次数都$> 0$, 因而都$<$
$f(x)$的次数.
\end{Proposition}

\begin{Lemma}
假定$f(x) = g(x) h(x)$, 那么$f(x)$是本原多项式 $\Leftrightarrow$ $g(x)$和$h(x)$都是本原多项式.
\end{Lemma}

\begin{Lemma}
对于一个唯一分解环$I$, 他的商域$Q$做成的一元多项式环$Q[x]$, 
$Q[x]$中的每个不等于零的多项式$f(x)$都可以写成
$\displaystyle f(x) = \frac{b}{a} f_0(x)$的样子.
这里$a, b \in I$, $f_0(x)$是$I[x]$上的本原多项式.
若$g_0(x)$也有$f_0(x)$的性质(即$f(x)$可以写成$\displaystyle \frac{b'}{a'} g_0(x)$的形式),则$g_0(x) = \epsilon f_0(x)$ ($\epsilon$是$I$的单位)
.\end{Lemma}

\begin{Lemma}
$I[x]$的一个本原多项式$f_0(x)$在$I[x]$里可约$\Leftrightarrow$ $f_0(x)$在$Q[x]$里可约.
\end{Lemma}

\begin{Lemma}
$I[x]$中的一个次数$>0$的本原多项式$f_0(x)$在$I[x]$中有唯一分解.
\end{Lemma}

\begin{Theorem}
一个唯一分解环$I$上的多项式环$I[x]$也是唯一分解环.
\end{Theorem}

\begin{Theorem}
若$I$是唯一分解环, 那么$I[x_1, x_2, \cdots, x_n]$也是, 其中$x_1, x_2, \cdots, x_n$是$I$上的未定元.
\end{Theorem}

\subsection{因式分解与多项式的根}

\begin{Definition}
$a $\;($\in $整环$I$)叫做$I[x]$的多项式的\textbf{根}, 如果$f(a) = \mathfrak{0}$.
\end{Definition}

\begin{Theorem}
$a $\;($\in$ 整环$I$)是$f(x)$的一个根 $\Leftrightarrow$ $(x - a) \mid f(x)$.
\end{Theorem}

\begin{Theorem}
给定整环$I$的$k$个不同的元素$a_1, a_2, \cdots, a_k$, 那么
$a_1, a_2, \cdots, a_k$都是$f(x)$的根 $\Leftrightarrow$ 
$(x - a_1)(x - a_2)\cdots(x-a_k) \mid f(x)$.
\end{Theorem}

\begin{Corollary}
$I[x]$中的$n$次多项式$f(x)$, 在$I$中最多有$n$个根.
\end{Corollary}

\begin{Definition}[重根]
$a$\;($\in I$)叫做$f(x)$的一个\textbf{重根}, 如果$(x - a)^k \mid f(x)$, $k$是$\ge 2$的整数.
\end{Definition}

\begin{Definition}[导数]
对于$I[x]$中的多项式$f(x) = a_n x^n +a_{n-1} x^{n-1} + \cdots +a_1 x +a_0$,
定义它的\textbf{导数}$f'(x) = n a_n x^{n-1} + (n-1) a_{n-1} x^{n-1} + \cdots + a_1$
\end{Definition}

\begin{Note}
这只是一个形式上的定义,不能从极限的角度来理解.
\end{Note}

\begin{Proposition}
导数适合以下计算规则
\begin{enumerate}[(i)]
	\item $[f(x) +g(x)]' = f'(x) + g'(x)$.
	\item $[f(x)g(x)]' = f(x) g'(x) +f'(x) g(x)$
	\item $[f(x)^t]' = t f(x)^{t-1}f'(x)$, $t (\in \mathbb{Z})\ge 2$
\end{enumerate}
\end{Proposition}

\begin{Theorem}
假定$a$是$f(x)$的一个根, 我们有 %\vspace*{\fill} 
 \begin{tightcenter}{ $a$是一个重根 $\Leftrightarrow (x - a) \mid f'(x)$}
 \end{tightcenter}
%\vspace*{\fill}
%} \\
\end{Theorem}

\begin{Corollary}
假定$I[x]$是一个唯一分解环, $a \in I[x], f(x) \in I[x]$, 我们有
\begin{tightcenter}
$a$是$f(x)$的一个重根 $\Leftrightarrow$ $(x - a)  $ 能够整除 $f(x)$和$f'(x)$的最大公共因子.
\end{tightcenter}
\end{Corollary}

\begin{Definition}
如果$(x-a)^k \mid f(x)$, 但是$(x-a)^{k+1} \nmid f(x)$, $k \in \mathbb{Z}^+$, 则称$a$是
$f(x)$的$k$重根.
\end{Definition}

\begin{Theorem}
$a$是$f(x)$的$k$重根 $\Rightarrow$ $(x-a)^{k-1} \mid f'(x)$.
\end{Theorem}

\begin{Theorem}
假定整环$I$的特征是无穷的, 我们有
\begin{tightcenter}
$a$是$f(x)$的$k$重根 $\Rightarrow$ $a$是$f'(x)$的$k-1$重根.
\end{tightcenter}
\end{Theorem}
\section{扩域}

\begin{Definition}[扩域]
一个域$E$叫做一个域$F$的\textbf{扩域(扩张)}, 如果$F$是$E$的子域.
\end{Definition}

\begin{Theorem}
令$E$是一个域. 
\begin{itemize}
\item 若$E$的特征是$\infty$, 则$E$含有一个与有理数同构的子域; 
\item 若$E$的特征是素数$p$, 则$E$含有一个域$\mathbb{Z}/(p)$同构的子域, 其中$\mathbb{Z}$是整数环.
\end{itemize}
\end{Theorem}

\begin{Definition}[素域]
一个域叫做一个\textbf{素域}, 假如他不包含真子域.
\end{Definition}

\begin{Note}
一个素域或者与有理数域$\mathbb{Q}$同构, 或者与$\mathbb{Z}_p = \mathbb{Z}/(p)$同构.
\end{Note}

\begin{Note}
令域$E$是与$F$的扩域. 我们从$E$中取一个子集$S$. 我们用$F(S)$表示包含$F$和$S$中的所有元素的$E$的最小子域, 把它叫做添加几个$S$于$F$所得的扩域.
\end{Note}

\begin{Note}
$F(S)$刚好包含$E$的一切可以写成
$\displaystyle \frac
{f_1(\alpha_1, \alpha_2, \cdots, \alpha_n)}
{f_2(\alpha_1, \alpha_2, \cdots, \alpha_n)}
$
形式的元,其中$\alpha_i$是$S$中的任意有限个元素, $f_1$和 $f_2$ ($ f_2 \neq 0$)是这些$\alpha$的多项式.
\end{Note}

\begin{Note}
若$S$是一个有限子集, $S = \{ \alpha_1, \alpha_2, \cdots, \alpha_n \}$, 那么我们也把$F(S)$记作$F(\alpha_1, \alpha_2, \cdots, \alpha_n)$.
\end{Note}

\begin{Theorem}
令$E$是$F$的一个扩域, $S_1, S_2$是$E$的两个子集, 那么$F(S_1)(S_2) = F(S_1 \cup S_2) = F(S_2)(S_1)$.
\end{Theorem}

\begin{Note}
于是我们可以把添加有限集归结为陆续添加单个元素: $F(\alpha_1, \alpha_2, \cdots, \alpha_n)
= F(\alpha_1)(\alpha_2)\cdots(\alpha_n)$.
\end{Note}

\begin{Definition}[单扩域]
添加一个元素$\alpha$于域$F$所得的扩域$F(\alpha)$叫做域$F$的\textbf{单扩域}(\textbf{单扩张}).
\end{Definition}

\subsection{单扩域}

\begin{Definition}[代数元、超越元]
假定$E$是$F$的扩域, $\alpha \in E$. $\alpha$叫做域$F$上的一个\textbf{代数元}, 若$\exists a_0, a_1, \cdots, a_n$不都等于零,使得$a_0 +a_1 \alpha + \cdots + a_n \alpha^n = 0$. 如果这样的$a_0, a_1, \cdots, a_n$不存在, $\alpha$叫做$F$上的一个\textbf{超越元}.
\end{Definition}

\begin{Definition}[单代数扩域、单超越扩域]
若$\alpha$是$F$上的一个代数元, $F(\alpha)$叫做$F$的一个\textbf{单代数扩域}; 
若$\alpha$是$F$上的一个超越元, $F(\alpha)$就叫做$F$的一个\textbf{单超越扩域}.
\end{Definition}

\begin{Theorem}
若$\alpha$是$F$上的一个超越元, 那么$F(\alpha) \cong F[x]$的商域, 其中$F[x]$是$F$上的一个未定元$x$的多项式环.
\end{Theorem}

\begin{Theorem}
若$\alpha$是$F$上的一个代数元, 那么
$F(\alpha) \cong F[x]/\left(p(x)\right)$, 其中$p(x)$是$F[x]$的一个\;\fbox{唯一}\;确定的、最高系数为${1}$的不可约多项式, 并且$p(\alpha) = 0$.
\end{Theorem}

\begin{Theorem}
令$\alpha$是域$F$上的一个代数元, 并且$F(\alpha) \cong F[x]/\left( p(x) \right)$, 
那么$F(\alpha)$的每一个元都可以唯一的表达成
$\displaystyle \sum_{i=0}^{n-1} c_i \alpha_i$
 ($c_i \in F$)的形式, 这里$n$是$p(x)$的系数. 要把两个多项式$f(\alpha)$和$g(\alpha)$相加,只需把相应的系数相加; $f(\alpha)$与$g(\alpha)$的乘积等于$r(\alpha)$, 这里$r(x)$是用$p(x)$除$f(x)g(x)$所得的余式.
\end{Theorem}

\begin{Definition}[极小多项式]
$F[x]$中满足条件$p(\alpha) = 0$的次数最低的多项式
$p(x) = x_n + c_{n-1} x^{n-1} + \cdots + c_0$ ($c_i \in F$), 叫做元$\alpha$的在$F$上的\textbf{极小多项式}, $n$叫做$\alpha$的在$F$上的\textbf{次数}.
\end{Definition}

\begin{Note}
$F$的单超越扩域是存在的, 且它们相互同构.
\end{Note}

\begin{Theorem}
对于任一给定的域$F$以及$F$上的一元多项式环$F[x]$的给定不可约多项式
$p(x) = x^n + c_{n-1} x^{n-1} + \cdots + c_0$, 总存在$F$的单代数扩域$F(\alpha)$, 
其中$\alpha$在$F$上的极小多项式是$p(x)$.
\end{Theorem}

\begin{Theorem}
令$F(\alpha)$和$F(\beta)$是域$F$的两个单代数扩域, 并且$\alpha$和$\beta$在$F$上有相同的极小多项式$p(x)$. 那么$F(\alpha) \cong F(\beta)$.
\end{Theorem}

\begin{Theorem}
在同构的意义下,存在且仅存在域$F$的一个单扩域$F(\alpha)$,其中$\alpha$的极小多项式是$F[x]$的给定的,最高次数为$1$的不可约多项式.
\end{Theorem}

\subsection{代数扩域}

\begin{Theorem}[代数扩域]
若域$F$的扩域$E$的每一个元都是$F$上的一个代数元, 那么$E$叫做$F$的一个\textbf{代数扩域}(\textbf{代数扩张}).
\end{Theorem}

\begin{Definition}
若是域$F$的一个扩域$E$作为$F$上的向量空间有维数$n$, 那么$n$叫做\textbf{扩域$E$在$F$上的次数}, 记作$(E:F)$.
这时$E$叫做$F$的一个\textbf{有限扩域}, 否则$E$叫做域$F$的一个\textbf{无限扩域}.
\end{Definition}

\begin{Theorem}
令$I$是域$F$的有限扩域, 而$E$是$I$的有限扩域. 那么$E$也是$F$的有限扩域, 并且$(E:F) = (E:I)(I:F)$.
\end{Theorem}

\begin{Note}
已经听不懂了!!! 感觉需要补高等代数中的向量空间的知识...
\end{Note}

\end{document}
