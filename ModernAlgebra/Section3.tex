\chapter{环}

\section{环与域}

\subsection{环的定义}

\begin{Definition}[环]
一个集合$R$叫做一个环, 如果
\begin{enumerate}[(1)]
	\item $R$是一个加群: $R$关于一个叫做加法的代数运算做成一个交换群.
	\item $R$对于另一个叫做乘法的代数运算是封闭的.
	\item $R$关于乘法结合
	\item 分配率: $a(b+c) = bc + ac, (a+b)c = ac + bc$
\end{enumerate}
\end{Definition}

\begin{Theorem}
环还满足以下运算规则
\begin{enumerate}[(1)]
\item[(7)] $(a-b)c = ac - bc, c(a - b) = ca - cb$
\item[(8)] $\mathfrak{0}a = a\mathfrak{0} = \mathfrak{0}$
\item[(9)] $(-a)b = a(-b) = -(ab)$
\item[(10)] $(-a)(-b) = ab$
\item[(11)] $a(b_1 + b_2 + \cdots + b_n) = ab_1 +ab_2 + \cdots + ab_n, (b_1 + b_2 + \cdots + b_n)a = b_1a + b_2a + \cdots + b_na$
\item[(12)] 
$ \displaystyle \left( \sum_{i=1}^m a_i \right) \left( \sum_{j=1}^n b_j \right) 
= \sum_{a=1}^m \sum_{b=1}^{n} a_i b_j $
           $$
\begin{aligned}
(a_1 + a_2 + \cdots + a_m) (b_1 + b_2 + \cdots + b_n) 
= 
a_1b_1 + a_1 b_2 + &\cdots + a_1 b_n \\
+ a_2 b_1 + a_2 b_2 + &\cdots + a_2 b_n \\
+ &\cdots \\
+ a_m b_1 + a_m b_2 + &\cdots + a_m b_n
\end{aligned}
 $$
 \item[(13)] $ (na)b = a(nb) = n(ab), n \in \mathbb{Z}^+$ 
 \item[(14)] 规定 $a^n = \underbrace{a a \cdots a}_{n\text{个}}, n \in \mathbb{Z}^{+}$, 则 $ a^m a^n = a^{m+n}, (a^m)^n = a^{mn} $
\end{enumerate}
\end{Theorem}


\subsection{子环、环的同态} %%%%%%%%%%%%

\begin{Definition}[子环]
一个环$R$的非空子集$S$如果对于R的代数运算来说也是环(整环、除环、域), 则称$S$是$R$的一个\textbf{子环}(\textbf{子整环}、\textbf{子除环}、\textbf{子域}).
\end{Definition}

\begin{Theorem}
若$S$是环$R$的一个非空子集, 则$S$是$R$的子环 $\iff$ $a, b \in S \Rightarrow a-b \in S, ab \in S$.
\end{Theorem}

\begin{Proposition}
环$R$的可以同每个元交换的元做成一个j交换子环$N = \{ n \mid a n = na, \forall a \in R\}$, 这个子环称为$R$的中心.
\end{Proposition}

\begin{Theorem}
若$R$是环, $R$到$\bar{R}$有一个满射使得对于两个运算都同态, 则$\bar{R}$也是一个环.
\end{Theorem}

\begin{Note}[环的同态、环的同构]
我们说两个环$R$和$\bar{R}$\textbf{同态}(\textbf{同构}), 如果存在一个$R$到$\bar{R}$的满射(双射), 使得$R$与$\bar{R}$对于两个环的一对加法以及一对乘法来说都同态.
\end{Note}

\begin{Remark}
总结下来, 如果$A$与$\bar{A}$同态,那么前面有什么后面就也有什么:
\begin{itemize}
	\item 前面有结合,后面就也有结合
	\item 前面有交换,后面就也有交换
	\item 前面有分配,后面就也有分配
	\item 前面是群,后面就也是群
	\item 前面是环,后面就也是环
\end{itemize}
\end{Remark}

\begin{Theorem}
若$R$和$\bar{R}$都是环, 且$R$与$\bar{R}$同态, 则
\begin{itemize}
	\item $R$的零元的象是$\bar{R}$的零元.
	\item $R$的元$a$的负元的象是$a$的象的负元{($\overline{-a} = -\overline{a}$)}
	\item $R$是交换环 $\Rightarrow$ $\bar{R}$也是交换环
	\item[!] $R$有单位元$\mathfrak{1}$ $\Rightarrow$ $\bar{R}$也有单位元$\bar{\mathfrak{1}}$, 且$\bar{\mathfrak{1}}$是$\mathfrak{1}$的象.
	\item $R$无零因子 $\not\Rightarrow$ $\bar{R}$无零因子
	\item $R$有零因子 $\not\Rightarrow$ $\bar{R}$有零因子
	\item $R$是整环(除环、域) $\not\Rightarrow$ $\bar{R}$是整环(除环、域)
\end{itemize}
\end{Theorem}

\begin{Proposition}
若$R$和$\bar{R}$都是环, 且$R$与$\bar{R}$同态, 则
\begin{itemize}
	\item $R$无零因子 $\not\Rightarrow$ $\bar{R}$无零因子
	\item $R$有零因子 $\not\Rightarrow$ $\bar{R}$有零因子
	\item $R$是整环(除环、域) $\not\Rightarrow$ $\bar{R}$是整环(除环、域)
\end{itemize}
\end{Proposition}

\begin{Proposition}
$R$与$\bar{R}$都是环, 且$R \cong \bar{R}$, 则
\begin{itemize}
	\item $R$无零因子 $\Leftrightarrow$ $\bar{R}$无零因子.
	\item $R$有非零元 $\Leftrightarrow$ $\bar{R}$有非零元.
	\item  $R$非零元有逆 $\Leftrightarrow$ $\bar{R}$非零元有逆
\end{itemize}
\end{Proposition}

\begin{Theorem}
$R$与$\bar{R}$都是环, 且$R \cong \bar{R}$, 则
\begin{itemize}
	\item $R$是整环 $\Leftrightarrow$ $\bar{R}$是整环.
	\item  $R$是除环 $\Leftrightarrow$ $\bar{R}$是除环.
	\item  $R$是域 $\Leftrightarrow$ $\bar{R}$是域.
\end{itemize}
\end{Theorem}


\begin{Lemma}
集合$A$和$\bar{A}$之间有一个双射$\phi$, 并且$A$有加法和乘法, 于是我们可以在$\bar{A}$中规定加法和乘法,使得$A$与$\bar{A}$关于一对加法和一对乘法来说都同构.
\end{Lemma}

\begin{Theorem}
假定$S$是环$R$的一个子环, $S$在$R$中的补集($R$ - $S$)与另一个环$\bar{S}$没有公共元,并且$S \cong \bar{S}$, 那么
存在一个与$R$同构的环$\bar{R}$,且$\bar{S}$是$\bar{R}$的子环.
\end{Theorem}

\newcommand{\StubSize}{0.5em}%
%\newcommand{\BraceForest}[1][]
%\usetikzlibrary{decorations, decorations.pathmorphing} % in the preamble


\begin{Note}[!]
\ \\
\begin{center}
\begin{forest}
for tree={grow=west, parent anchor=west, child anchor=east, anchor=east}
[, s sep=2em,phantom
	[ 环$S$,
		  name=RingS 
		[环R	 ,edge label={node [midway, above, font=\scriptsize] {子环}} 	
		]
	]		
	[ {环$\bar{S}$}, name=RingBarS
		[ $?$ ,edge label={node [midway, below, font=\scriptsize] {子环}} 
		]	
	]
	[
		{$ (R-S) \cap \bar{S} = \emptyset $}
	]
] {
	\draw[<->%, out=-130,in=-85
	] (RingS.south) %.. controls +(south west: 0.5em) and +(north west: 0.5em) .. 
	--(RingBarS.north) node [midway, right] {$\cong_\phi$}
	;
	\draw[decorate, decoration={brace, amplitude=1.5em}]
     ([yshift=-.5em]current bounding box.north east) --
      node[right=2em%, font=\sffamily\bfseries\Large
      ]
        {
        %\begin{matrix}
         {$\Rightarrow \exists$ 环$? = \bar{R}: \left\{ \begin{aligned}
        	&\bar{R} = {(R-S)} \cup \bar{S} \\ 
       	    &
       	    	\forall \bar{x}, \bar{y} \in \bar{R}: 
       	    	\begin{aligned}
       	    	& \bar{x} + \bar{y} = \psi(x + y), 
       	    	\bar{x} \bar{y} = \psi(x y), \\
       	    	& x = \psi^{-1}(\bar{x}), y = \psi^{-1}(\bar{y})
       	    	\end{aligned}
        	 \\ 
       	   % }
        	&R \cong \bar{R}, \psi: x \mapsto \begin{cases} x & x \in R-S \\ \phi{(x)} & x \in S \end{cases}
        \end{aligned} \right.$} 
%        {subject to $R \cong \bar{R}, \psi: x \mapsto \begin{cases} a \\ b\end{cases}$}
        %\end{matrix}
        }
      ([yshift=-.5em]current bounding box.south east)
   ;	
}
%\BraceForest[red,ultra thick]%
  % inside environment forest
\end{forest}
\end{center}
\end{Note}

\begin{Proposition}
一个除环的中心是一个域.
\end{Proposition}

\subsection{理想} %%%%%%%%%%%%%%%%%%%%%%%%%%%%

\begin{Definition}[!理想]
环$R$的一个非空子集$I$叫做一个\textbf{理想子环}(简称\textbf{理想}), 如果
\begin{enumerate}
	\item $a, b \in I \Rightarrow a - b \in I$
	\item $a \in I, r \in R \Rightarrow ra, ar \in I$.
\end{enumerate}
\end{Definition}

\begin{Proposition}
一个环至少有两个理想 (1) $I = \{\mathfrak{0}\}$, 叫做$R$的\textbf{零理想}.
(2) I = R, 叫做$R$的\textbf{单位理想}.
\end{Proposition}

\begin{Theorem}
一个除环$R$只有两个理想,就是零理想和单位理想.
\end{Theorem}

\begin{Note}
因此,理想这个概念对于除环或者域来说没有多大用处.
\end{Note}

\begin{Note}
一个环除了以上两个理想之外,可能有其他理想.
\end{Note}

\begin{Proposition}
给定一个环$R$,$a$是$R$中的任意一个元素,考虑最小的理想$I$使得$a \in I$. 作集合
$I = \{ \left( x_1 a y_1 + x_2 a y_2 + \cdots \right) + s a + a t + na \mid
x_i, y_i, s, t \in R, n \in \mathbb{Z} \}$, 则$I$是包含$a$的最小理想.
\end{Proposition}

\begin{Definition}[主理想]
上面的这样的$I$叫做元$a$生成的\textbf{主理想}, 用符号$(a)$来表示.
\end{Definition}

\begin{Note}
一个主理想$(a)$的元的形式并不是永远像上面那样复杂.
\begin{enumerate}
	\item 当$R$满足交换律时, 可以写成$ra + na, r \in R, n \in \mathbb{Z}$.
	\item 当$R$有单位元时, 可以写成$ \displaystyle \sum x_i a y_i, x_i, y_i \in R $.

	\item 当$R$既满足交换律又有单位元时, 可以写成$ \displaystyle ra, r \in R $.
\end{enumerate}
\end{Note}

\begin{Proposition}
给定一个环$R$,$a_1, a_2, \cdots, a_m \in R$,考虑最小的理想$I$使得$a_1, a_2, \cdots, a_m \in I$.
做集合$I = \{ s_1 + s_2 + \cdots + s_m \mid s_i \in (a_i) \}$, 
则$I$是包含$a_1, a_2, \dots, a_m$的最小理想.
\end{Proposition}

\begin{Definition}
上面的这样的$I$叫做$a_1, a_2, \cdots, a_m$生成的理想, 用符号$(a_1, a_2, \cdots, a_m)$来表示.
\end{Definition}

\begin{Note}
两个元素生成的理想, 可能是主理想,也可能不是.
\end{Note}

\begin{Proposition} \ \\
\begin{itemize}
\item 群$G$的两个子群的交集还是$G$的子群.
\item 两个不变子群的交集还是不变子群.
\item 两个子环的交集还是子环.
\item 两个子整环的交集还是子整环.
\item 两个子除环的交集还是子除环.
\item 两个子域的交集还是子域.
\item 两个理想的交集还是一个理想.
\end{itemize}
\end{Proposition}

\subsection{剩余类环、同态与理想} %%%%

\begin{Note}
给定一个环$R$和$R$的一个理想$I$, 则我们就加法来说, $R$做成一个群, $I$做成$R$的一个不变子群, 从而$I$的陪集$[a], [b], [c], \cdots$做成$R$的一个分类, 叫做\textbf{模$I$的剩余类}. 同时这个分类描述$R$的元素之间的等价关系, 用符号$a \equiv b \text{ mod } I$表示(读作$a$同余$b$模$I$), 即$a \equiv b \text{ mod } I \Leftrightarrow a \sim b \Leftrightarrow a - b \in I$. 且类$[a]$所包含的元素可以写成$\{ a + u \mid u \in I \}$
\end{Note}

\begin{Theorem}
假定$R$是一个环, $I$是它的一个理想, $\bar{R}$是所有模$I$的剩余类做成的集合, 如果在$\bar{I}$上规定加法和乘法$[a] + [b] = [a + b], [a][b] = [ab]$.那么$\bar{I}$本身也是一个环, 并且$R$与$\bar{R}$同态.
\end{Theorem}

\begin{Definition}[模$I$的剩余类环]
上面的$\bar{R}$叫做环$R$的\textbf{模$I$的剩余类环}, 用符号$R/I$来表示.
\end{Definition}

\begin{Theorem}[!]
假定$R$与$\bar{R}$是两个环, 并且$R$与$\bar{R}$同态, 那么这个同态满射的核$I$是$R$的一个理想, 并且$R/I \cong \bar{R}$
\end{Theorem}

\begin{Theorem}
在环$R$到环$\bar{R}$的同态满射下:
\begin{enumerate}[(1)]
	\item $R$的一个子环的象$\bar{S}$是$\bar{R}$的一个子环.
	\item $R$的一个理想$I$的象$\bar{I}$是$\bar{R}$的一个理想.
	\item $\bar{R}$的一个子环$\bar{S}$的原象$S$是R的一个子环.
	\item $\bar{R}$的一个理想$\bar{I}$的原象$I$是$R$的一个理想.
\end{enumerate}

\end{Theorem}

\begin{Note}
	环-群, 子环-子群, 理想-不变子群
\end{Note}

\begin{Proposition}
$\phi$是环$R$到环$\bar{R}$的一个同态满射: $\phi$是$R$与$\bar{R}$之间的同构映射 $\Leftrightarrow$ $\phi$的核是零理想.
\end{Proposition}

\subsection{最大理想}

\begin{Definition}[最大理想]
如果一个环$R$的理想$I (\neq R)$, 除了$R$和$I$以外, 无其他包含$I$的理想,称$I$为$R$的\textbf{最大理想}.
\end{Definition}

\begin{Lemma}
假定$I (\neq R)$是环$R$的一个理想: 剩余类环$R/I$除了零理想和单位理想外不再有其他理想 $\Leftrightarrow$ $I$是最大理想.
\end{Lemma}

\begin{Lemma}
若有单位元($\neq \mathfrak{0}$)的交换环$R$除了零理想和单位理想以外没有其他理想, 那么$R$一定是一个域.
\end{Lemma}

\begin{Theorem}[!]
$R$是有单位元的交换环, $I (\neq R)$是$R$的理想: $R/I$是域 $\Leftrightarrow$
$I$是$R$的最大理想.
\end{Theorem}

\begin{Proposition}
$\mathbb{Z}_n$是域$\Leftrightarrow n$是素数. 
\end{Proposition}

\subsection{商域}

\begin{Theorem}
若$R$是无零因子的交换环, 则存在一个包含$R$的域$Q$, 使得$Q$刚好是由所有元$\displaystyle \frac{a}{b} \; (a, b \in R, b \neq \mathfrak{0} )$所做成的,这里$\displaystyle \frac{a}{b} = ab^{-1} = b^{-1}a$.
\end{Theorem}

\begin{Definition}[商域]
一个域$Q$叫做环$R$的一个\textbf{商域}, 如果$Q \supseteq R$, 并且$Q$刚好是由所有元$\displaystyle \frac{a}{b} \; (a, b \in R, b \neq 0)$所做成的.
\end{Definition}

\begin{Theorem}
假定$R$是一个有两个以上的元的环, $F$是一个包含$R$的域,则$F$包含$R$的一个商域.
\end{Theorem}

\begin{Note}
一般来讲, 一个环很可能有两个以上的商域. 不过,同构的环的商域也同构,所以抽象的来讲,一个环最多只有一个商域.
\end{Note}

\section{特殊的环}

\subsection{交换环}

\begin{Definition}[交换环]
一个环$R$叫做\textbf{交换环}, 如果$ab = ba,  \forall a, b \in R$.
\end{Definition}

\begin{Proposition}
在一个交换环中${(ab)}^n = a^n b^n$.
\end{Proposition}

\subsection{无零因子环}  %%%%%%%%%%%%%%%%%%%%%

\begin{Note}
$ab = 0 \Rightarrow a = 0 $ 或者 $b = 0$ 在环里不一定成立.
\end{Note}

\begin{Definition}[零因子]
在一个环$R$中, 若$a \neq 0, b \neq 0$但$ab = 0$, 则称$a$是$R$的\textbf{左零因子}, $b$是$R$的\textbf{右零因子}. 左零因子和右零因子统称为\textbf{零因子}.
\end{Definition}

\begin{Remark}
左零因子不一定是右零因子. 但是如果有左零因子, 就一定有右零因子. 如果$R$是交换环, 则左零因子一定是右零因子.
\end{Remark}

\begin{Theorem}[!]
在一个没有零因子的环里, 左右消去律都成立.
\begin{enumerate}
	\item $a \neq 0, ab = ac \implies b = c$
	\item $d \neq 0, bd = cd \implies b = c$
\end{enumerate}
反过来, 在一个环里如果\;\fbox{有一个}\;消去律成立,那么这个环没有零因子.
\end{Theorem}

\begin{Corollary}
在一个环$R$中如果有一个消去律成立,那么另一个消去律也成立.
\end{Corollary}

\begin{Proposition}
对于模$p$的剩余类环$\mathbb{Z}_p$, $p$是素数 $\Leftrightarrow $ $\mathbb{Z}_p$做成一个域.
\end{Proposition}

\begin{Proposition}
在一个环$R$里, 对于加法的阶, 可能有的元素是无限的,有的元素是有限的.
\end{Proposition}

\begin{Theorem}
在一个无零因子环中, 所有非零元素$R$对于加法的阶都相同: 要么都无限大, 要么都有限且相等.
\end{Theorem}

\begin{Definition}[无零因子环的特征]
在一个无零因子环$R$中, 所有非零元关于加法的阶,叫做$R$的\textbf{特征}.
\end{Definition}

\begin{Theorem}
如果无零因子环$R$的特征是一个有限整数$n$, 则$n$一定是素数.
\end{Theorem}

\begin{Corollary}
整环、除环以及域的特征或者是无限大,或者是一个素数.
\end{Corollary}

%\begin{Proposition}
%{\color{gray}{一个的特征是$p$的交换环里, $(a+b)^p = a^p +b^p$.}}
%\end{Proposition}

\subsection{整环}

\begin{Note}[!!整环的判别] \ \\ \begin{tightcenter}
\begin{forest}
%for tree={circle,draw,s sep=0.8cm}
for tree={grow'=east, parent anchor=east, child anchor=west, anchor=west}
	[ 整环, name=IntRing, tier=2
		[ 环, name={Ring}, tier=property
			[ 加群
				[ 加法成群: {I II IV V} ]
				[ 加法交换 ]
			]
			[ 乘法: I II
			]
			[ 两个分配律
			]
		]
		[乘法交换, name={ProductCommutative}, tier=property]
		[有$\mathfrak{1}$, name={Has1}, tier=property]
		[无零因子, name=NonZeroFact, tier=property]
	]
\end{forest}
\end{tightcenter}
\end{Note}

\begin{Definition}[单位元]
对于环$R$, 如果$ea = ae = a$~($\forall a \in R$), 则称$e$是环$R$的单位元. 一般,一个环未必有单位元.
\end{Definition}

\begin{Proposition}
一个环如果有单位元, 则唯一. 用$\mathfrak{1}$来表示.
\end{Proposition}

\begin{Definition}[逆元]
若$ba = \mathfrak{1}$, 则称$b$为$a$的\textbf{左逆元}. 若$ba = ab = \mathfrak{1}$, 则称$b$为$a$的\textbf{逆元}. 
\end{Definition}

\begin{Proposition}
如果$a$有逆元, 则唯一.
\end{Proposition}

\begin{Proposition}
如果$a$有逆元, 则规定$a^{-m} = {(a^{-1})}^m, a^0 = \mathfrak{1}$. 则$a^m a^n = a^{m+n}, {(a^m)}^n = a^{mn}, \forall m, n \in \mathbb{Z}$.
\end{Proposition}

\begin{Proposition}[模$n$的剩余类环]
$R = \{ [0], [1], \cdots, [n-1] \}$, 加法: $[a] + [b] = [a+b]$, 乘法: $[a][b] = [ab]$做成一个交换环, 被称为\textbf{模$n$的剩余类环}, 零元$\mathfrak{0} = [0]$, 单位元$\mathfrak{1} = [1]$.
\end{Proposition}

\begin{Definition}[整环]
一个环$R$叫做一个\textbf{整环}, 如果
\begin{enumerate}
	\item 乘法适合交换律: $ab = ba$.
	\item $R$有单位元1: $\mathfrak{1}a = a\mathfrak{1} = a$.
	\item $R$没有零因子: $ab = \mathfrak{0} \Rightarrow a = \mathfrak{0}$或$b = \mathfrak{0}$
\end{enumerate}
\end{Definition}

\begin{Proposition}
对于有单位元的环来说, 加法适合交换律是环定义里其他条件的结果.
\end{Proposition}


\begin{Theorem}
若$S$是整环$R$的一个非空子集, 则$S$是$R$的子整环的充要条件是 (1)$a, b \in S \implies a-b \in S, ab \in S$; (2) $1 \in S$.
\end{Theorem}

%\begin{Proposition}
%二项式定理$\displaystyle (a + b)^n = \binom{n}{0} a^n + \binom{n}{1} a^{n-1}b + \binom{n}{2} a^{n-2}b^2 + \cdots + \binom{n}{n} b^n$在交换环中成立.
%\end{Proposition}

%\begin{Proposition}
%$\{ a + b \sqrt{2} \mid a, b \in \mathbb{Z}\}$对于普通加法和乘法来说是一个整环.
%\end{Proposition}

\begin{Definition}[!整除]
对于整环$I$,若$a \in I$, 存在$b, c \in I$使$a=bc$, 则称$b$能\textbf{整除}$a$, 记作$b \mid a$, 称$b$为$a$的\textbf{因子}. 若$b$不是$a$的因子,则记作$b \nmid a$.
\end{Definition}

\begin{Proposition}
整除关系是一个二元关系, 满足
\begin{enumerate}[(1)]
\item 反身性, $a \mid a$ 
\item \xout{对称性}
\item 传递性, $a \mid b, b \mid c \implies a \mid c$
\end{enumerate}
\end{Proposition}

\begin{Definition}[相伴]
如果$a \mid b$且$b \mid $a, 则称$a$与$b$\textbf{相伴}, 记作$a \sim b$. 
\end{Definition}

\begin{Definition}[相伴等价定义]
{元$b$叫做元$a$的\textbf{相伴元},如果存在一个单位$\epsilon$使得$b = \epsilon a$}.
\end{Definition}

\subsection{除环、域} %%%%%%%%%%%%%%%%%%%%%

\begin{Proposition}
对于元素个数$\ge 2$的环, $\mathfrak{1} \neq \mathfrak{0}$, 且$\mathfrak{0}$没有逆元.
\end{Proposition}

\begin{Definition}[除环]
一个环$R$叫做一个\textbf{除环}, 如果
\begin{enumerate}
	\item $R$至少含有一个不等于零的元.
	\item $R$有单位元.
	\item $R$的任何一个非零元都有逆.
\end{enumerate}
\end{Definition}

\begin{Definition}[域]
一个交换除环叫做一个\textbf{域}.
\end{Definition}

\begin{Property}
除环没有零因子.
\end{Property}

\begin{Property}
除环$R$的所有非零元对于乘法来说做成一个群$R^*$, 我们把$R^*$叫做\textbf{除环$R$的乘群}.
\end{Property}

\begin{Note}
对于一个环$R$来说, 从$R^*$是对于乘法做成一个群, 也能推出$R$是除环.
\end{Note}

\begin{Note}
在除环$R$中, 方程$ax = b, ya = b (a \neq \mathfrak{0})$都有唯一解, 分别是$a^{-1}b$和$ba^{-1}$, 他们未必相等. 在一个域里$a^{-1}b = ba^{-1}$, 用符号
$\displaystyle \frac{b}{a}$表示.
\end{Note}

\begin{Property}
域满足以下计算法则
\begin{enumerate}
	\item $\displaystyle \frac{a}{b} = \frac{c}{d} \Leftrightarrow ad = bc$.
	\item $\displaystyle \frac{a}{b} + \frac{c}{d} = \frac{ad + bc}{bd}$.
	\item $\displaystyle \frac{a}{b} \frac{c}{d} = \frac{ac}{bd}$.
\end{enumerate}
\end{Property}

\begin{Proposition}
%$R = 
%\left\{ (\alpha, \beta) \mid \alpha, \beta \in \mathbb{C} \right\}, 
%(\alpha_1, \beta_1) + (\alpha_2, \beta_2) = 
%	(\alpha_1 + \alpha_2, \beta_1 + \beta_2), 
%(\alpha_1, \beta_1)(\alpha_2, \beta_2) = 
%	(\alpha_1 \alpha_2 - \beta_1 \overline{\beta_2})
%	(\alpha_1 \beta_2 + \beta_1 \overline{\alpha_2})
%$ 做成一个除环, 叫做\textbf{四元数除环}, 它不是交换环(所以不是域).
存在不是域的除环, 例如\textbf{四元数除环}.
\end{Proposition}

\begin{Note} !环、整环、域之间的关系: \begin{center}
\begin{forest}
%for tree={circle,draw,s sep=0.8cm}
for tree={grow'=east, parent anchor=east, child anchor=west, anchor=west}
[,phantom
	[ 整环, name=IntRing, tier=2
		[ 环, name={Ring}, tier=property
			[ 加群
				[ 加法成群: {I II IV V} ]
				[ 加法交换 ]
			]
			[ 乘法: I II
			]
			[ 两个分配律
			]
		]
		[乘法交换, name={ProductCommutative}, tier=property]
		[有$\mathfrak{1}$, name={Has1}, tier=property]
		[无零因子, name=NonZeroFact, tier=property]
	]
	[ 域
		[ 除环, for tree={edge={color=Blue}}, tier=2
			[有非零元, tier=property, for tree={edge={color=red}}]
			[非零元有逆, tier=property, for tree={edge={color=red}}] {
				\draw[->, color=DarkGreen] () to[out=east, in=east] (NonZeroFact);
			}
		] {
			\draw[-, color=red] (.east) to (Ring.west);	
			\draw[-, color=red] (.east) to (Has1.west);
		}
	] {
		\draw[-, color=Blue] (.east) to (ProductCommutative.west); % Don't forget semicolon.
		\draw[-, color=Blue, dotted] (.east) to (IntRing.west); % Don't forget semicolon.
	}
]
\end{forest}
\end{center}
\end{Note}

\begin{Proposition}
一个至少有两个元且没有零因子的有限环,是一个除环.
\end{Proposition}

\begin{Theorem}
若$S$是除环$R$的一个非空子集, 则$S$是$R$的子除环 的充要条件是 (1) $S$有非零元; (2) $a, b \in S \Rightarrow a-b \in S$; (3) $\forall a, b \in S, b \neq 0 \Rightarrow ab^{-1} \in S$.
\end{Theorem}

\begin{Theorem}
若$S$是域$R$的一个非空子集, 则$S$是$R$的子域 的充要条件是 (1) $S$有非零元; (2) $a, b \in S \Rightarrow a-b \in S$; (3) $\forall a, b \in S, b \neq 0 \Rightarrow ab^{-1} \in S$.
\end{Theorem}



\section{具体的环}

\subsection{整数环} %%%%%%%%%%%%%%%

\begin{Definition}[整数环]
整数关于普通加法和乘法构成的环.
\end{Definition}

\begin{Proposition}
整数环是一个整环.
\end{Proposition}

\subsection{$M_n[K]$}

\begin{Note}
数域$K$上的所有$n$级矩阵$M_n[K]$, 对于矩阵的加法和乘法构成一个环.
\end{Note}