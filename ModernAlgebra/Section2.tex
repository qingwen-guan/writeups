\section{群论}

\subsection{群的定义和性质}

\begin{Remark}
群是一个代数系统(定义代数运算的集合), 它只有一个代数运算, 被称为乘法. 便利起见$(a, b)$的象写成$a b$ %\marginnote{之前写成$a \circ b$}
\end{Remark}

\begin{Definition}[群\mbox{[Group]}的第一定义]
在集合$G \neq \emptyset$上规定一个叫做乘法的\;\fbox{代数运算}\;%\marginnote{以后简称乘法}.
这个代数系统被称为群, 如果
\begin{itemize}
	\item[\uppercase\expandafter{\romannumeral1}] 乘法封闭, $\forall a, b \in G, ab \in G$ %\marginnote{代数运算要求封闭性}
	\item[\uppercase\expandafter{\romannumeral2}] 乘法结合, $\forall a, b, c \in G, (ab)c = a(bc)$
	\item[\uppercase\expandafter{\romannumeral3}] $ \forall a, b \in G$, $ax = b, ya = b$在$G$中都有解.
\end{itemize}
\end{Definition}

\begin{Theorem}[左单位元]
对于群$G$中至少有一个元$e$, 叫做$G$的一个\textbf{左单位元},使得$\forall a \in G$都有 $ea = a$.
\end{Theorem}

\begin{Theorem}[左逆元]
对于群$G$中的任何一个元素$a$, 在$G$中存在一个元$a^{-1}$,叫做$a$的\textbf{左逆元}, 能让$a^{-1} a = e$.
\end{Theorem}

\begin{Definition}[群\mbox{[Group]}的第二定义]
在集合$G \neq \emptyset$上规定乘法. 这个代数系统被称为\textbf{群}, 如果
\begin{itemize}
	\item[\uppercase\expandafter{\romannumeral1}] 乘法封闭
	\item[\uppercase\expandafter{\romannumeral2}] 乘法结合
	\item[IV] 左单位元: $\exists e \in G$使 $ea =a$ 对 $\forall a \in G$都成立.
	\item[V] 左逆元: $\forall a \in G, \exists \, a^{-1}$使$a^{-1}a = e$.
\end{itemize}
\end{Definition}

\begin{Definition}[群的阶]
如果$|G|$有限, 称其为\textbf{有限群}, 称他的\textbf{阶}是$G$的元素个数. 
如果$G$中有无穷多个元素, 称其为\textbf{无限群}, 称他的\textbf{阶}无限.
\end{Definition}

\begin{Definition}[交换群、Abel群]
群中交换律不一定成立, 如果乘法满足交换律($\forall a, b \in G, ab = ba$), 则称之为\textbf{交换群}(\textbf{Abel群}).
\end{Definition}

\begin{Theorem}[单位元]
在一个群$G$里存在且只存在一个元$e$, 使得$ea = ae = a$对于$\forall a \in G$成立. 这个元素被称为群$G$的\textbf{单位元}.
\end{Theorem}

\begin{Theorem}[逆元]
对于群$G$的任意一个元素$a$来说, 有且只有一个元素$a^{-1}$, 
使 $a^{-1} a = a a^{-1} = e$. 这个元素被称为$a$的\textbf{逆元}, 或者简称\textbf{逆}.
\end{Theorem}

\begin{Note}
证明$a^{-1}$是$a$的逆的方法: $a^{-1}a = e$或者$aa^{-1} = e$ (不用都说明).
\end{Note}

\begin{Property}[乘积的逆等于逆的乘积]
$\forall a, b \in G, {(ab^{-1})}^{-1} = b a^{-1}$
\end{Property}

\begin{Definition}
规定$\forall n \in \mathbb{Z}^{+}: a^n = \underbrace{a a \cdots a}_{n\text{个}}, a^0 = e, a^{-n} = (a^{-1})^n$
\end{Definition}

\begin{Proposition}
$ \forall n, m \in \mathbb{Z}: a^n a^m = a^{n+m}, {(a^n)}^m = a^{mn} $ ({$\Rightarrow {(a^{-1})}^{-1} = a$})
\end{Proposition}

\begin{Definition}[元素的阶]
在一个群$G$中,使得$a^n = e$的最小正整数, 叫做$a$的\textbf{阶}. 若这样的$n$不存在, 称$a$是无穷阶的,或者叫$a$的阶是无穷.
\end{Definition}

%\begin{Theorem}
%假定群的元$a$的阶是$n$, 则$a^r$的阶是$\displaystyle \frac{n}{\text{gcd}(r, n)}$.
%\end{Theorem}

\begin{Theorem}[III'\mbox{[消去律]}]
群的乘法满足: $ax = ax' \Rightarrow x = x', ya = y'a \Rightarrow y = y'$
\end{Theorem}

\begin{Corollary}
在群里, $ax = b$ 和 $ya = b$都有唯一解.
\end{Corollary}

\begin{Theorem}[有限群的另一定义]
一个带有乘法的 \fbox{有限集合}  $G \neq \emptyset$, 若满足I、II、III', 则$G$是一个\textbf{群}.
\end{Theorem}

\subsection{群的同态}

\begin{Theorem}
$G$与$\bar{G}$关于他们的乘法同态, 则 $G$是群$\Rightarrow \bar{G}$也是群.
\end{Theorem}

\begin{Theorem}
假定$G$和$\bar{G}$是两个群, 在$G$到$\bar{G}$的一个同态满射之下, $G$的单位元$e$的象是$\bar{G}$的单位元, $G$的元$a$的逆元$a^{-1}$的象是$a$的象的逆元($\overline{a^{-1}} = \bar{a}^{-1}$).
\end{Theorem}

\begin{Theorem}
$G$与$\bar{G}$关于他们的乘法同构, 则 $G$是群$\Leftrightarrow \bar{G}$是群.
\end{Theorem}

\subsection{变换群} %%%%%%%%%%%%%%%%%%

\begin{Definition}[变换的乘法]
$\tau_1 \tau_2: a \mapsto {\left(a^{\tau_1}\right)}^{\tau_2}$
\end{Definition}

\begin{Theorem}[变换乘法结合]
$(\tau_1 \tau_2) \tau_3 = \tau_1 ( \tau_2 \tau_3 )$
\end{Theorem}

\begin{Theorem}
$G$是集合$A$的若干变换构成的集合, 如果$G$基于变换的乘法做成一个群,
则$G$中的变换一定是双射变换.
\end{Theorem}

\begin{Definition}[变换群]
如果一个集合$A$的若干\;\fbox{双射变换}\;对于变换的乘法能够做成一个群,则称这个群为A的一个\textbf{变换群}.
\end{Definition}

\begin{Theorem}
一个集合$A$上的所有双射变换做成一个变换群$G$.
\end{Theorem}

\begin{Theorem}
任何一个群都与一个变换群同构.
\end{Theorem}

\begin{Theorem}
一个变换群的单位元一定是恒等变换.
\end{Theorem}

\subsection{置换群} %%%%%%%%%%%%%%%%%%%%%%%%%%%%%%%%%%%%%%%%%%%%%%%%

\begin{Definition}[置换]
\fbox{有限集合}\;上的\;\fbox{双射变换}\;叫做\textbf{置换}, 一般用$\pi$表示.
\end{Definition}

\begin{Definition}[置换群]
有限集合上的若干置换做成的群叫\textbf{置换群}.
\end{Definition}

\begin{Definition}[对称群]
一个$n$元集合$A = \{ a_1, a_2, \cdots, a_n \}$上的所有置换
(有$n!$个)做成的群叫做$n$次\textbf{对称群}, 用$S_n$来表示.
\end{Definition}

\begin{Theorem}
$$
\left.
\begin{aligned}
\pi_1 &= \begin{pmatrix} 
j_1       & \cdots & j_k       & j_{k+1} & \cdots & j_n \\
j_1^{(1)} & \cdots & j_k^{(1)} & j_{k+1} & \cdots & j_n \\
\end{pmatrix} \\
\pi_2 &= \begin{pmatrix} 
j_1       & \cdots & j_k       & j_{k+1}       & \cdots & j_n      \\
j_1       & \cdots & j_k       & j_{k+1}^{(2)} & \cdots & j_n^{(2)} \\
\end{pmatrix} 
\end{aligned}
\right\}
\Rightarrow
\pi_1 \pi_2 = \begin{pmatrix} 
j_1       & \cdots & j_k       & j_{k+1}       & \cdots & j_n \\
j_1^{(1)} & \cdots & j_k^{(1)} & j_{k+1}^{(2)} & \cdots & j_n^{(2)} \\
\end{pmatrix}
$$
\end{Theorem}

\begin{Definition}[$k$-循环置换]
如果$S_n$中的置换满足$a_{i_1}$的象是$a_{i_2}$, $a_{i_2}$的象是$a_{i_3}$, $\cdots$, 
$a_{i_{k-1}}$的象是$a_{i_k}$, $a_{i_{k}}$的象是$a_{i_1}$, 其他元素,如果还有的话,象是不变的, 则称之为\textbf{$k$-循环置换}.
用$(i_1 \, i_2 \, i_3 \, \cdots \, i_{k-1} \, i_k)$ 或 
$(i_2 \, i_3 \, \cdots \, i_{k-1} \, i_k \, i_1)$ 或 $\cdots$ 或   
$( i_k \, i_1 \, i_2 \, i_3 \, \cdots \, i_{k-1})$来表示.
\end{Definition}

\begin{Proposition}
$(i_1 \, i_2 \, \cdots i_k)^{-1} = (i_k \, \cdots \, i_2 \, i_1)$.
\end{Proposition}

\begin{Proposition}
$k$-循环置换的阶是$k$.
\end{Proposition}

\begin{Proposition}
任何一个置换都可以写成若干没有共同数字的循环置换的乘积.
\end{Proposition}

\begin{Proposition}
两个没有共同数字的循环置换可以交换.
\end{Proposition}

\begin{Proposition}
任何一个有限群都与一个置换群同构.
\end{Proposition}

\subsection{循环群}

\begin{Definition}[循环群]
若一个群$G$的每一个元都是$G$的某一固定元$a$的乘方, 我们就称$G$是一个\textbf{循环群}, $a$是$G$的一个\textbf{生成元}, 并记$G = (a)$, 且说$G$是由元$a$生成的。
\end{Definition}

\begin{Definition}[$\mathbb{Z}_n$\mbox{[模$n$的剩余类加群]}]
$G$包含所有模$n$的剩余类,$G = \{ [0], [1], \cdots, [n-1] \}$, 定义乘法(叫做加法) $[a] + [b] = [a +b]$,可以证明$(G, +)$做成一个群, 叫做\textbf{模$n$的剩余类加群}.
\end{Definition}

\begin{Theorem}
假定$G$是由$a$生成的循环群, 则$G$的构造可以完全由$a$的阶来决定:
\begin{itemize}
\item 如果$a$的阶无限, 则$G \cong \mathbb{Z}$.
\item 如果$a$的阶为$n$, 则$G \cong \mathbb{Z}_n$.
\end{itemize}
\end{Theorem}

\begin{Note}
于是$|(a)| = n$, 其中$n$为$a$的阶.
\end{Note}

\begin{Proposition}
一个循环群一定是交换群.
\end{Proposition}

\begin{Proposition}
$a$生成一个阶是$n$的循环群$G$, 则$a^r$也生成$G$,如果$\text{gcd}(r, d) = 1$.
\end{Proposition}

\begin{Proposition}
$G$是循环群, 且$G$与$\bar{G}$同态,则$\bar{G}$也是循环群.
\end{Proposition}

\begin{Proposition}
$G$是无限阶循环群, $\bar{G}$是任何循环群, 则$G$与$\bar{G}$同态.
\end{Proposition}

\subsection{子群}

\begin{Definition}[子群]
如果一个群$G$的一个子集$H$关于群$G$的乘法也能做成一个群,则称$H$为$G$的一个\textbf{子群}.
\end{Definition}

\begin{Theorem}
一个群$G$的一个非空子集$H$做成$G$的子群,当且仅当
\begin{enumerate}[(i)]
\item $a, b \in H \Rightarrow ab \in H$
\item $a \in H \Rightarrow a^{-1} \in H$
\end{enumerate}
\end{Theorem}

\begin{Corollary}
若$H$是$G$的子群, 则, $H$的单位元就是$G$的单位元, $a$在$H$中的逆就是$a$的$G$中的逆.
\end{Corollary}

\begin{Theorem}
一个群$G$的一个非空子集$H$做成$G$的子群,当且仅当 (iii) $a, b \in H \Rightarrow ab^{-1} \in H$
\end{Theorem}

\begin{Theorem}
一个群$G$的一个非空\;\fbox{有限}\;子集$H$做成$G$的子群,当且仅当 (i) $a, b \in H \Rightarrow ab \in H$
\end{Theorem}

\begin{Note}[验证非空集合是群的方法]
(1) I, II, III (2) I、II、IV, V (3) 有限集: I, II, III'
(4) 子群: (i), (ii) \mbox{(5) 子群: (iii)} (6) 有限子群: (i)
\end{Note}

\begin{Definition}[生成子群]
对于群$G$的非空子集$S$, 包含$S$的最小子群, 被称为由$S$生成的子群, 记为$(S)$.
\end{Definition}

\begin{Theorem}
$S = \{a\}$时, $(S) = (a)$.
\end{Theorem}

%\begin{Proposition}
%群$G$的两个子群的交集也是$G$的子群.
%\end{Proposition}

\begin{Proposition}
循环群的子群也是循环群.
\end{Proposition}

\begin{Proposition}
$H$是群$G$的一个非空子集, 且$H$的每个元素的阶都有限,
则$H$做成子群的充要条件是 (i) $a, b \in H \Rightarrow ab \in H$.
\end{Proposition}

\subsection{子群的陪集}

\begin{Definition}
群$G$, 子群$H$, 规定$G$上的关系$\sim: a \sim b \Leftrightarrow a b^{-1} \in H$
\end{Definition}

\begin{Theorem}
上面规定的关系$\sim$是等价关系.
\end{Theorem}

\begin{Definition}[右陪集]
由上述等价关系确定集合的分类叫做$H$的\textbf{右陪集}.
\end{Definition}

\begin{Theorem}
包含元$a$的右陪集$ = Ha =  \{ ha \mid h \in H \}$
\end{Theorem}

\begin{Definition}
群$G$, 子群$H$, 规定$G$上的关系$\sim': a \sim 'b \Leftrightarrow b^{-1} a \in H$. 可以证明$\sim'$是等价关系.
\end{Definition}

\begin{Definition}[左陪集]
由上述等价关系$\sim': a \sim' b \Leftrightarrow b^{-1} a \in H$, 确定集合的分类叫做$H$的\textbf{左陪集}, 包含元$a$的左陪集可以用$aH = \{ ah \mid h \in H \}$表示.
\end{Definition}

\begin{Theorem}
一个子群的右陪集与左陪集个数相等: 个数或者都是无穷大, 或者都有限且相等.
\end{Theorem}

\begin{Definition}[指数]
一个群$G$的一个子群$H$的右陪集(或左陪集)的个数叫做$H$在$G$里的指数.
\end{Definition}

\begin{Theorem}
右陪集所含元素的个数等于子群$H$所含元素的个数.
\end{Theorem}

\begin{Theorem}
$H$是一个有限群$G$的子群, 那么$H$的阶$n$和他在$G$中的指数$j$都能整除$G$的阶$N$, 并且$N = nj$
\end{Theorem}

\begin{Theorem}[元素的阶整除群的阶]
一个有限群$G$的任何一个元$a$的阶能够整除$G$的阶$|G|$.
\end{Theorem}

\begin{Proposition}
阶是素数的群一定是循环群.
\end{Proposition}

\begin{Proposition}
阶是$p^m$的群($p$是素数)一定包含一个阶是$p$的子群.
\end{Proposition}

\begin{Proposition}
若我们把同构的群看做一样的,一共只存在两个阶是4的群,它们都是交换群.
\end{Proposition}

%\begin{Proposition}
%有限非交换群至少有6个元素.
%\end{Proposition}

\subsection{不变子群、商群}

\begin{Definition}[不变子群]
群$G$的子群$N$叫做$G$的\textbf{不变子群}, 如果$\forall a \in G$, 有$Na = aN$. 一个不变子群$N$的一个左(或右)陪集叫做$N$的一个\textbf{陪集}.
\end{Definition}

\begin{Definition}
$S_1, S_2 \subseteq $群$G$, 规定子集的乘法
$S_1 S_2  = \{s_1 s_2 \mid s_1 \in S_1, s_2 \in S_2 \}$. 显然这个乘法满足结合律.
\end{Definition}

\begin{Theorem}
已知一个群$G$有一个子群$N$, $N$是不变子群的充要条件是$aNa^{-1} = N, \forall a \in G$.
\end{Theorem}

\begin{Theorem}
已知一个群$G$有一个子群$N$, $N$是不变子群的充要条件是
$a \in G, n \in N \Rightarrow ana^{-1} \in N $.
\end{Theorem}

\begin{Theorem}
如果$N$刚好包含$G$的所有具有以下性质的元$n$,
$$
	na = an, \forall a \in G
$$
则$N$是$G$的不变子群. 我们称这个不变子群是G的\textbf{中心}.
\end{Theorem}

\begin{Theorem}
$N$是群$G$的不变子群, 在其陪集$\{ aN, bN, cN, \cdots \}$上定义的乘法$(xN, yN) \mapsto (xy)N$, 则这个乘法是此陪集的二元运算,且此陪集对于上面规定的乘法来说构成一个群.
\end{Theorem}

\begin{Definition}[商群]
一个群$G$的一个不变子群$N$的所有陪集关于陪集的乘法做成的群叫做$G$的\textbf{商群},用$G/N$表示.
\end{Definition}

\begin{Theorem}
对于有限群, $ \displaystyle | G/N | $ = $\frac{|G|}{|N|}$.
\end{Theorem}

%\begin{Proposition}
%两个不变子群的交集还是不变子群.
%\end{Proposition}

\begin{Proposition}
$H$是$G$的子群, $N$是$G$的不变子群, 则$HN$是$G$的子群.
\end{Proposition}

\subsection{同态与不变子群}

\begin{Theorem}
一个群$G$与它的商群$G/N$同态.
\end{Theorem}

\begin{Definition}[核]
$\phi$是群$G$到群$\bar{G}$的一个同态满射, $\bar{G}$的单位元$\bar{e}$在$\phi$之下的所有原象做成的$G$的子集叫做$\phi$的\textbf{核}.
\end{Definition}

\begin{Theorem}
$G$和$\bar{G}$是两个群,且$G$与$\bar{G}$同态,则这个同态满射的核$N$是$G$的一个不变子群,且$G/N \cong \bar{G}$.
\end{Theorem}

\begin{Remark}
一个群只和``相当于''它的商群同态
\end{Remark}

\begin{Definition}
$\phi$是$A \rightarrow \bar{A}$的满射, 取$S \subseteq A$, 定义$S$的象是$S$中所有元素的象做成的集合. 取$\bar{S} \subseteq \bar{A}$, 定义$\bar{S}$的原象是$\bar{S}$中所有元素的原象做成的集合.
\end{Definition}

\begin{Theorem}
$G$和$\bar{G}$是两个群,且$G$与$\bar{G}$同态,则在这个同态满射之下:
\begin{itemize}
\item[(1)] $G$的一个子群$H$的象$\bar{H}$也是$\bar{G}$的一个子群.
\item[(2)] $G$的一个不变子群$N$的象$\bar{N}$也是$\bar{G}$的一个不变子群.
\item[(1')] $\bar{G}$的一个子群$\bar{H}$的原象$H$也是$G$的一个子群.
\item[(2')] $\bar{G}$的一个不变子群$\bar{N}$的原象$N$也是$G$的一个不变子群.
\end{itemize}
\end{Theorem}

\begin{Remark}
这也体现了同态的性质,前面有的后面也有!
\end{Remark}

\begin{Proposition}
假定群$G$与群$\bar{G}$同态, $\bar{N}$是$\bar{G}$的不变子群, $N$是$\bar{N}$的逆象,则
$ G/N \sim \bar{G}/\bar{N} $.
\end{Proposition}

\begin{Proposition}
假定群$G$与$\bar{G}$是两个有限循环群,他们的阶各是$m$和$n$,则$G$与$\bar{G}$同态$\Leftrightarrow n \mid m$
\end{Proposition}

\begin{Proposition}
假定群$G$是一个循环群, $N$是$G$的一个子群, 则$G/N$也是循环群.
\end{Proposition}
