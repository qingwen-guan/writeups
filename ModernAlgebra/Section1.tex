\section{基本概念}


\begin{Note}
近世代数(或抽象代数)的主要内容就是研究所谓\textbf{代数系统},即带有运算的集合。
\end{Note}

\begin{Note}
规定
$-\infty < n, -\infty + n = -\infty, \forall n \in N$
 规定$-\infty + (-\infty) = -\infty$.
\end{Note}

\subsection{映射}

\begin{Definition}[映射]
$$ \begin{aligned}
\phi: A &\rightarrow D \\
a &\mapsto d = \phi (a) = \overline{a} \end{aligned}$$ 
其中$A$称为\textbf{定义域(Domain)}, $D$称为\textbf{陪域(Codomain)}, $\{ \phi(a) \mid a \in A \}$称为\textbf{值域(Image)}, 记作$f(A)$或者$\mathop{\text{Im}}f$.
\end{Definition}

\begin{Definition}
对于映射$f$: $A \rightarrow B$, $g$: $C \rightarrow D$
\begin{tightcenter}
$f = g  \stackrel{\text{def}}{\iff} A = B \And C = D \And f(a) = g(a)~\forall a \in A$
\end{tightcenter}
\end{Definition}

\begin{Definition}[满射]
映射$\phi: A \rightarrow B$被称为\textbf{满射}, 如果$B = \OpIm{\phi}$, 换句话说
\begin{tightcenter}
$\forall b \in B \implies$ $b \in \OpIm{\phi}$~(即$\exists a \in A$ 使得$\phi{(a)} = b$).
\end{tightcenter}
它对应$\phi^{-1}$都有象.
\end{Definition}

\begin{Definition}[单射]
映射$\phi: A \rightarrow \bar{A}$被称为\textbf{单射}, 如果$A$中不同元素的在$\phi$下的象不同, 换句话说
\begin{tightcenter}
设$a_1, a_2 \in A$, $\phi(a_1) = \phi(a_2) \implies a_1 = a_2$
\end{tightcenter}
它对应$\phi^{-1}$象唯一
\end{Definition}

\begin{Definition}[变换]
从$A$到$A$的映射 $\tau: A \rightarrow A, a \mapsto \tau(a)$ 叫$A$\textbf{变换}, 我们也用$a^\tau$表示$\tau(a)$.
 {{如果$\tau$是满射(单射、双射), 则称为}}\textbf{满变换}(\textbf{单变换}、\textbf{双射变换}).
\end{Definition}

%\subsection{映射的乘法、逆映射}

\begin{Definition}[映射的乘法]
设$f$: $A \rightarrow B$, $g$: $B \rightarrow C$, 令
$$
	(g \circ f)(a) \triangleq g(f(a)), \forall a \in A
$$
则称$g \circ f$是$g$与$f$的\textbf{乘积}.
\end{Definition}

\begin{Note}
映射的乘法满足结合律, 即$h \circ (g \circ f) = (h \circ g) \circ f$.
\end{Note}


\begin{Definition}
若$f$: $A \rightarrow A$, $a \mapsto a$, 则$f$是$A$上的\textbf{恒等变换}, 记作$1_A$.
\end{Definition}

\begin{Proposition}设$f$: $A \rightarrow B$, 则$f \circ 1_A = 1_B \circ f = f$
\end{Proposition}

\begin{Definition}[可逆映射]
设$f$: $A \Rightarrow B$, 如果存在$g$: $B \rightarrow A$使得$g \circ f = 1_A$, 且$f \circ g = 1_B$, 那么称$f$是可逆映射, 把$g$称为$f$的逆映射.
\end{Definition}

\begin{Note}
若$f$可逆, 则$f$的逆映射唯一, 把$f$的逆映射记作$f^{-1}$. $f^{-1}$也是可逆映射, 并且$(f^{-1})^{-1} = f$.
\end{Note}

\begin{Theorem}
$f$: $A \rightarrow B$是可逆映射 $\iff$ $f$是双射.
\end{Theorem}

\subsection{等价关系与集合划分}

\begin{Definition}[集合的划分]
如果集合$A$是他的一些非空子集的并集, 其中每两个不相等的子集的交是空集(称为\textbf{不相交}), 那么把这些子集组成的集合称为$A$的一个\textbf{划分}.
\end{Definition}

\begin{Definition}[二元关系\mbox{[Relation]}]
$S \times S$的一个子集$W$称为$S$上的一个二元关系. 若$(a, b) \in W$, 则称$a$和$b$有$W$关系, 记作$a \sim_W b$.
若$(a, b) \notin W$, 则称$a$和$b$没有$W$关系.
%$R: A \times A \rightarrow D = \{\text{对}, \text{错}\} $, 
%若
%$R(a, b) = \text{对}$
%, 称
%$(a, b)$
%满足关系$R$, 记为$a \, R \, b$.
\end{Definition}

\begin{Note}
整除是一个二元关系.
\end{Note}

\begin{Definition}[等价关系]
如果$\sim$是$A$的元素间的关系,满足 
\begin{enumerate}[(1)]
\item 自反性, $\forall a \in A, a \sim a$.
\item 对称性, $\forall a, b \in A$, 若$a \sim b$, 则$b \sim a$.
\item 传递性, $\forall a, b, c \in A$, 若$a \sim b$, $b\sim c$, 则$a \sim c$.
\end{enumerate}
则称$\sim$为\textbf{等价关系}. \index{等价关系}
\end{Definition}

\begin{Definition}
设$~$是$S$上的一个等价关系, 任给$a \in S$, 令
$$
\bar{a} \triangleq \{ x \in S \mid x \sim a\}
$$
则把$\bar{a}$称为$a$的\textbf{等价类}.
\end{Definition}

\begin{Note}
$x \in \bar{a} \iff x \sim a$
\end{Note}

\begin{Note}(代表)
由于$a \sim a$, 因此$a \in \bar{a}$, 把$a$称为$\bar{a}$的一个\textbf{代表}.
\end{Note}

\begin{Property}
$\bar{a} = \bar{b} \iff a \sim b$
\end{Property}

\begin{Property}
$\bar{a} \neq \bar{b} \implies \bar{a} \cap \bar{b} = \emptyset$
\end{Property}

\begin{Theorem}
如果集合$S$上有一个等价关系$\sim$, 那么所有等价类组成的集合是$S$的一个划分.
\end{Theorem}


\begin{Theorem}
如果集合$S$中有一个划分, 那么可以在$S$上建立一个等价关系, 使得这个划分是由所有等价类组成的.
\end{Theorem}

\begin{Definition}[商集]
集合$S$的一个划分也称为$S$的一个\textbf{商集}, 是$S$的所有等价类组成的集合.
\end{Definition}

\begin{Definition}[$\mathbb{Z}_p$\mbox{[模$n$的剩余类]}]
{\color{gray}{
$ \{ [0], [1], \cdots, [n-1] \} $, $[i] = \{ k n + i \mid k \in \mathbb{Z} \}$
}}
\end{Definition}

\subsection{代数运算}


%\begin{Note}
%判断一个法则$\phi$是映射的充要条件: (1) 都有象 (2) 象唯一.
%\end{Note}

\begin{Definition}[代数运算] 
$$\begin{aligned}
A \times B &\rightarrow D \\ 
(a, b) &\mapsto d = \phi(a, b) = \circ (a, b) = a \circ b \end{aligned}$$

\end{Definition}

%{\color{gray}{
%\begin{Note}
%$A = B$时, 对于代数运算$ A \times A \rightarrow D $, $ a \circ b $ 和 $ b \circ a $ 都有意义,但不一定相等.
%\end{Note}
%}}

\begin{Definition}[$A$的代数运算, 二元运算] 
假如 $ \circ $ 是一个 $ A \times A \rightarrow A$的代数运算(即$A = B = D$),我们说集合$A$对于代数运算$\circ$来说是闭的, 也说, $\circ$是\textbf{$A$的代数运算}或\textbf{二元运算}.
\end{Definition}

\begin{Note}[$A$的代数运算判别] 
 \ \\
\begin{center} \begin{forest}
%for tree={circle,draw,s sep=0.8cm}
for tree={grow'=east, parent anchor=east, child anchor=west, anchor=west}
[ { $A$的代数运算 }
	[ 是映射 
		[ 都有象 ]
		[ {象唯一 $\forall a, a', b, b' \in A: a = a', b = b' \Rightarrow a \circ b = a' \circ b'$ }
		]
	]
	[ {封闭 $\forall a, b \in A: a \circ b \in A, $} ]
]
\end{forest} \end{center}

\end{Note}

\subsection{运算律} %%%%%%%%%%%%%%

\begin{Definition}[结合率]
我们说,一个集合$A$的代数运算$\circ$满足结合律,假如对于$A$的任何三个元素$a, b, c$来说都有$
(a \circ b) \circ c = a \circ (b \circ c)
$
\end{Definition}

%\begin{Definition}
%假如对于$A$的$n$ ($n \ge 2$)个固定的元 $a_1, a_2, \cdots, a_n$来说,所有的加括号方式 $\pi(a_1 \circ a_2 \circ \cdots \circ a_n)$都相等,我们就把这些步骤可以得到的唯一的结果,用$a_1 \circ a_2 \circ \cdots \circ a_n $ 来表示.
%\end{Definition}

\begin{Theorem}
若$A$的代数运算$\circ$满足结合律,则对于$A$的任意$n$($n \ge 2$)个元素 $a_1, a_2, \cdots, a_n$来说,对于任意的加括号的方法$\pi$, $\pi(a_1 \circ a_2 \circ \cdots \circ a_n)$ 都相等,我们用$a_1 \circ a_2 \circ \cdots \circ a_n$ 来表示.
\end{Theorem}

\begin{Definition}[交换律]
如果$A$上的代数运算$\circ$满足$\forall a, b \in A: a \circ b = b \circ a$,则称$\circ$满足\textbf{交换律}. 对于$a, b \in A$, 如果$a \circ b = b \circ a$, 则称$a, b$\textbf{可交换}.
\end{Definition}

\begin{Theorem}
若$A$上的代数运算$\circ$满足结合律与交换律,则$a_1 \circ a_2 \circ \cdots \circ a_n$ 可以任意交换顺序.
\end{Theorem}

\begin{Definition}[分配率]
$ \astrosun $和$ \oplus $ 都是 $A$上的代数运算, 
\begin{enumerate}[(1)]
\item 若
$ a \, \astrosun \, (b \oplus c) = (a \, \astrosun \, b) \oplus (a \, \astrosun \, c), \forall a, b, c $, 则称 $ \astrosun$和$ \oplus $  满足左分配率.
\item 若
$ (a \oplus b) \, \astrosun \, c = (a \, \astrosun \, c) \oplus ( b \, \astrosun \, c), \forall a, b, c$, 则称 $ \astrosun$和$\oplus $  满足右分配率.
\end{enumerate}
\end{Definition}

\begin{Theorem}
若$A$上的二元运算$\oplus$ 满足结合律, $ \astrosun $和$\oplus $ 满足左分配率,则
$$
a \, \astrosun \, ( b_1 \oplus b_2 \oplus \cdots \oplus b_n ) =  ( a \, \astrosun \, b_1) \oplus (a \, \astrosun \, b_2) \oplus \cdots \oplus (a \, \astrosun \, b_n)
$$
\end{Theorem}

\begin{Theorem}
若$A$上的二元运算$\oplus$ 满足结合律, $ \astrosun $和$\oplus $ 满足右分配率,则
$$
( a_1 \oplus a_2 \oplus \cdots \oplus a_n ) \, \astrosun \, b =  ( a_1 \, \astrosun \, b) \oplus ( a_2 \, \astrosun \, b) \oplus \cdots \oplus (  a_n \, \astrosun \, b)
$$
\end{Theorem}

\subsection{同态}


%\begin{Definition}[一一映射]
%{\color{gray}{
%既是满射又是单射.
%}}
%\end{Definition}

\begin{Note}[映射判别] \ \\ \begin{center}
\begin{forest}
%for tree={circle,draw,s sep=0.8cm}
for tree={grow'=east, parent anchor=east, child anchor=west, anchor=west}
[ 同构映射, color=red
	[
		同态满射, color=red, edge={dashed, color=red}, name=IsHomoFullMap, tier=1
		[
			同态映射, color=red, edge={dashed, color=red}, name=IsHomoMap, tier=2
			[,phantom]
			[
			{同态 ${a \circ b} \mapsto \bar{a}\; \bar{\circ} \; \bar{b} $}, name=IsHomo, color=red, edge={color=red}, tier=prop
			]
		] 
		[,phantom]
	]
	[ 双射, edge={color=red},tier=1
		[ 满射, name=IsFullMap, tier=2
			[ 是映射, name=IsMap, tier=prop 
				[ 都有象
				]
				[ {象唯一 $a = b \Rightarrow \bar{a} = \bar{b}$ }
				]
			]
			[ 满的, tier=prop ]
		]
		[ 单射, edge=dashed, tier=2
			[ {单的 $ \overline{a}  = \overline{b} \Rightarrow a = b $
			  }
			  , name=IsSingle
			  , tier=prop
			]
		] {
			\draw[-] (.east) to (IsMap.west); % Don't forget semicolon.
		}
	] {
		\draw[-] (.east) to (IsSingle.west); % Don't forget semicolon.	
	}
] {
	\draw[-, red] (.east) to (IsHomo.west); % Don't forget semicolon.		
	\draw[-, red] (IsHomoFullMap.east) to (IsFullMap.west); % 
	\draw[-, red] (IsHomoFullMap.east) to (IsHomo.west); % 
	\draw[-, red] (IsHomoMap.east) to (IsMap.west); % 
}
\end{forest}
\end{center}
\end{Note}


\begin{Definition}[同态映射]
对于$\phi: A \rightarrow \bar{A}$, $A$上有二元运算$\circ$, $\bar{A}$上有二元运算$\bar{\circ}$. 称 $\phi$是 $A$到 $\bar{A}$的\textbf{同态映射}, 如果
$\forall a, b \in A$, $\bar{a} := \phi(a), \bar{b} := \phi(b)$有 
$ a \circ b \mapsto \bar{a} \, \bar{\circ} \, \bar{b}$.
\end{Definition}

%\begin{Note}[同态映射判别]
%(1) 是映射(都有象、象唯一) (2) $ \overline{a \circ b} = \bar{a} \,{\bar{\circ}}\, \bar{b}$
%\end{Note}

\begin{Definition}[同态满射、同态]
如果$A$到$\bar{A}\;${\fbox{存在}}\;一个同态映射$\phi$, 且它是满射, 则称$A$与$\bar{A}$\;(关于$\circ$与$\bar{\circ}$)\textbf{同态}. 称这个映射是一个\textbf{同态满射}.
\end{Definition}

%\begin{Note}[同态满射判别]
%(1) 是映射(都有象、象唯一) (2) 满 (3) 同态 
%\end{Note}

\begin{Definition}[同构映射、同构]
如果$A$到$\bar{A}\;${\fbox{存在}}\;一个同态映射$\phi$, 且它是双射, 则称$A$与$\bar{A}\;$(关于$\circ$与$\bar{\circ}$)\textbf{同构}, 记为$A \cong
 \bar{A}$. 称这个映射是一个(关于$\circ$与$\bar{\circ}$的)\textbf{同构映射}(简称\text{同构}).
\end{Definition}

\begin{Proposition}
同构关系是一个等价关系.
\end{Proposition}

%\begin{Note}[同构映射判别]
%(1) 是映射(都有象、象唯一) (2) 满 (3) 单 (4) 同态
%\end{Note}

\begin{Theorem}
假定对于代数运算$\circ$和$\bar{\circ}$来说, $A$与$\bar{A}$同态, 那么
\begin{enumerate}[(1)]
\item 若 $\circ$ 满足结合律, $\bar{\circ}$也满足结合律;
\item 若 $\circ$ 满足交换律, $\bar{\circ}$也满足交换律.
\end{enumerate}
\end{Theorem}

\begin{Theorem}
$ \astrosun $和$ \oplus $ 是 $A$的两个代数运算, 
$ \bar{\astrosun} $和$\bar{\oplus} $ 是 $\bar{A}$的两个代数运算,
有
$\phi$既是$A$与$\bar{A}$的关于$ \astrosun $和$\bar{\astrosun}$ 的同态满射,
$\phi$也是$A$与$\bar{A}$的关于$ \oplus $和$\bar{\oplus}$ 的同态满射,
则 
\begin{enumerate}[(1)]
\item 若$ \astrosun$和$ \oplus $ 满足第一分配率, 则 $ \bar{\astrosun} $和$\bar{\oplus} $ 也满足第一分配率.
\item 若$ \astrosun$和$\oplus $ 满足第二分配率, 则 $ \bar{\astrosun} $和$\bar{\oplus} $ 也满足第二分配率.
\end{enumerate}
\end{Theorem}

%\begin{Definition}[自同构]
%对于$\circ$和$\circ$来说的一个$A$与$A$之间的\;\fbox{同构映射}\;叫做一个对于$\circ$来说的$A$的\textbf{自同构}.
%\end{Definition}

