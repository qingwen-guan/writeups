\section{基本概念}

\subsection{代数运算}

\begin{Note}
近世代数(或抽象代数)的主要内容就是研究所谓\textbf{代数系统},即带有运算的集合。
\end{Note}

\begin{Definition}[映射]
$$ \begin{aligned}
A &\rightarrow D \\
a &\mapsto d = \phi (a) = \overline{a} \end{aligned}$$ 
\end{Definition}

%\begin{Note}
%判断一个法则$\phi$是映射的充要条件: (1) 都有象 (2) 象唯一.
%\end{Note}

\begin{Definition}[代数运算] 
$$\begin{aligned}
A \times B &\rightarrow D \\ 
(a, b) &\mapsto d = \phi(a, b) = \circ (a, b) = a \circ b \end{aligned}$$
\end{Definition}

%{\color{gray}{
%\begin{Note}
%$A = B$时, 对于代数运算$ A \times A \rightarrow D $, $ a \circ b $ 和 $ b \circ a $ 都有意义,但不一定相等.
%\end{Note}
%}}

\begin{Definition}[$A$的代数运算, 二元运算] 
假如 $ \circ $ 是一个 $ A \times A \rightarrow A$的代数运算(即$A = B = D$),我们说集合$A$对于代数运算$\circ$来说是闭的, 也说, $\circ$是\textbf{$A$的代数运算}或\textbf{二元运算}.
\end{Definition}

\begin{Note}[$A$的代数运算判别] 
 \ \\
\begin{center} \begin{forest}
%for tree={circle,draw,s sep=0.8cm}
for tree={grow'=east, parent anchor=east, child anchor=west, anchor=west}
[ { $A$的代数运算 }
	[ 是映射 
		[ 都有象 ]
		[ {象唯一 $\forall a, a', b, b' \in A: a = a', b = b' \Rightarrow a \circ b = a' \circ b'$ }
		]
	]
	[ {封闭 $\forall a, b \in A: a \circ b \in A, $} ]
]
\end{forest} \end{center}

\end{Note}

\subsection{运算律} %%%%%%%%%%%%%%

\begin{Definition}[结合率]
我们说,一个集合$A$的代数运算$\circ$满足结合律,假如对于$A$的任何三个元素$a, b, c$来说都有$
(a \circ b) \circ c = a \circ (b \circ c)
$
\end{Definition}

%\begin{Definition}
%假如对于$A$的$n$ ($n \ge 2$)个固定的元 $a_1, a_2, \cdots, a_n$来说,所有的加括号方式 $\pi(a_1 \circ a_2 \circ \cdots \circ a_n)$都相等,我们就把这些步骤可以得到的唯一的结果,用$a_1 \circ a_2 \circ \cdots \circ a_n $ 来表示.
%\end{Definition}

\begin{Theorem}
若$A$的代数运算$\circ$满足结合律,则对于$A$的任意$n$($n \ge 2$)个元素 $a_1, a_2, \cdots, a_n$来说,对于任意的加括号的方法$\pi$, $\pi(a_1 \circ a_2 \circ \cdots \circ a_n)$ 都相等,我们用$a_1 \circ a_2 \circ \cdots \circ a_n$ 来表示.
\end{Theorem}

\begin{Definition}[交换律]
如果$A$上的代数运算$\circ$满足$\forall a, b \in A: a \circ b = b \circ a$,则称$\circ$满足\textbf{交换律}. 对于$a, b \in A$, 如果$a \circ b = b \circ a$, 则称$a, b$\textbf{可交换}.
\end{Definition}

\begin{Theorem}
若$A$上的代数运算$\circ$满足结合律与交换律,则$a_1 \circ a_2 \circ \cdots \circ a_n$ 可以任意交换顺序.
\end{Theorem}

\begin{Definition}[分配率]
$ \astrosun $和$ \oplus $ 都是 $A$上的代数运算, 
\begin{enumerate}[(1)]
\item 若
$ a \, \astrosun \, (b \oplus c) = (a \, \astrosun \, b) \oplus (a \, \astrosun \, c), \forall a, b, c $, 则称 $ \astrosun$和$ \oplus $  满足第一分配率.
\item 若
$ (a \oplus b) \, \astrosun \, c = (a \, \astrosun \, c) \oplus ( b \, \astrosun \, c), \forall a, b, c$, 则称 $ \astrosun$和$\oplus $  满足第二分配率.
\end{enumerate}
\end{Definition}

\begin{Theorem}
若$A$上的二元运算$\oplus$ 满足结合律, $ \astrosun $和$\oplus $ 满足第一分配率,则
$$
a \, \astrosun \, ( b_1 \oplus b_2 \oplus \cdots \oplus b_n ) =  ( a \, \astrosun \, b_1) \oplus (a \, \astrosun \, b_2) \oplus \cdots \oplus (a \, \astrosun \, b_n)
$$
\end{Theorem}

\begin{Theorem}
若$A$上的二元运算$\oplus$ 满足结合律, $ \astrosun $和$\oplus $ 满足第二分配率,则
$$
( a_1 \oplus a_2 \oplus \cdots \oplus a_n ) \, \astrosun \, b =  ( a_1 \, \astrosun \, b) \oplus ( a_2 \, \astrosun \, b) \oplus \cdots \oplus (  a_n \, \astrosun \, b)
$$
\end{Theorem}

\subsection{同态}

%\begin{Definition}[满射]
%映射$\phi: A \rightarrow \bar{A}$被称为\textbf{满射}, 如果
%$\forall \hat{a} \in \bar{A}, \exists a \in A \text{ s.t. } \bar{a} = \hat{a}$. (它对应$\phi^{-1}$都有象)
%\end{Definition}

%\begin{Definition}[单射]
%映射$\phi: A \rightarrow \bar{A}$被称为\textbf{单射}, 如果
%$\forall a, b \in A, a \neq b \Rightarrow \bar{a} \neq \bar{b}$. (它对应{$\phi^{-1}$象唯一})
%\end{Definition}

%\begin{Definition}[一一映射]
%{\color{gray}{
%既是满射又是单射.
%}}
%\end{Definition}

\begin{Note}[映射判别] \ \\ \begin{center}
\begin{forest}
%for tree={circle,draw,s sep=0.8cm}
for tree={grow'=east, parent anchor=east, child anchor=west, anchor=west}
[ 同构映射, color=red
	[
		同态满射, color=red, edge={dashed, color=red}, name=IsHomoFullMap, tier=1
		[
			同态映射, color=red, edge={dashed, color=red}, name=IsHomoMap, tier=2
			[,phantom]
			[
			{同态 ${a \circ b} \mapsto \bar{a}\; \bar{\circ} \; \bar{b} $}, name=IsHomo, color=red, edge={color=red}, tier=prop
			]
		] 
		[,phantom]
	]
	[ 双射, edge={color=red},tier=1
		[ 满射, name=IsFullMap, tier=2
			[ 是映射, name=IsMap, tier=prop 
				[ 都有象
				]
				[ {象唯一 $a = b \Rightarrow \bar{a} = \bar{b}$ }
				]
			]
			[ 满的, tier=prop ]
		]
		[ 单射, edge=dashed, tier=2
			[ {单的 $ \overline{a}  = \overline{b} \Rightarrow a = b $
			  }
			  , name=IsSingle
			  , tier=prop
			]
		] {
			\draw[-] (.east) to (IsMap.west); % Don't forget semicolon.
		}
	] {
		\draw[-] (.east) to (IsSingle.west); % Don't forget semicolon.	
	}
] {
	\draw[-, red] (.east) to (IsHomo.west); % Don't forget semicolon.		
	\draw[-, red] (IsHomoFullMap.east) to (IsFullMap.west); % 
	\draw[-, red] (IsHomoFullMap.east) to (IsHomo.west); % 
	\draw[-, red] (IsHomoMap.east) to (IsMap.west); % 
}
\end{forest}
\end{center}
\end{Note}


\begin{Definition}[变换]
从$A$到$A$的映射 $\tau: A \rightarrow A, a \mapsto \tau(a)$ 叫$A$\textbf{变换}, 我们也用$a^\tau$表示$\tau(a)$.
 {{如果$\tau$是满射(单射、双射), 则称为}}\textbf{满变换}(\textbf{单变换}、\textbf{双射变换}).
\end{Definition}

\begin{Definition}[同态映射]
对于$\phi: A \rightarrow \bar{A}$, $A$上有二元运算$\circ$, $\bar{A}$上有二元运算$\bar{\circ}$. 称 $\phi$是 $A$到 $\bar{A}$的\textbf{同态映射}, 如果
$\forall a, b \in A$, $\bar{a} := \phi(a), \bar{b} := \phi(b)$有 
$ a \circ b \mapsto \bar{a} \, \bar{\circ} \, \bar{b}$.
\end{Definition}

%\begin{Note}[同态映射判别]
%(1) 是映射(都有象、象唯一) (2) $ \overline{a \circ b} = \bar{a} \,{\bar{\circ}}\, \bar{b}$
%\end{Note}

\begin{Definition}[同态满射、同态]
如果$A$到$\bar{A}\;${\fbox{存在}}\;一个同态映射$\phi$, 且它是满射, 则称$A$与$\bar{A}$\;(关于$\circ$与$\bar{\circ}$)\textbf{同态}. 称这个映射是一个\textbf{同态满射}.
\end{Definition}

%\begin{Note}[同态满射判别]
%(1) 是映射(都有象、象唯一) (2) 满 (3) 同态 
%\end{Note}

\begin{Definition}[同构映射、同构]
如果$A$到$\bar{A}\;${\fbox{存在}}\;一个同态映射$\phi$, 且它是双射, 则称$A$与$\bar{A}\;$(关于$\circ$与$\bar{\circ}$)\textbf{同构}, 记为$A \cong
 \bar{A}$. 称这个映射是一个(关于$\circ$与$\bar{\circ}$的)\textbf{同构映射}(简称\text{同构}).
\end{Definition}

\begin{Proposition}
同构关系是一个等价关系.
\end{Proposition}

%\begin{Note}[同构映射判别]
%(1) 是映射(都有象、象唯一) (2) 满 (3) 单 (4) 同态
%\end{Note}

\begin{Theorem}
假定对于代数运算$\circ$和$\bar{\circ}$来说, $A$与$\bar{A}$同态, 那么
\begin{enumerate}[(1)]
\item 若 $\circ$ 满足结合律, $\bar{\circ}$也满足结合律;
\item 若 $\circ$ 满足交换律, $\bar{\circ}$也满足交换律.
\end{enumerate}
\end{Theorem}

\begin{Theorem}
$ \astrosun $和$ \oplus $ 是 $A$的两个代数运算, 
$ \bar{\astrosun} $和$\bar{\oplus} $ 是 $\bar{A}$的两个代数运算,
有
$\phi$既是$A$与$\bar{A}$的关于$ \astrosun $和$\bar{\astrosun}$ 的同态满射,
$\phi$也是$A$与$\bar{A}$的关于$ \oplus $和$\bar{\oplus}$ 的同态满射,
则 
\begin{enumerate}[(1)]
\item 若$ \astrosun$和$ \oplus $ 满足第一分配率, 则 $ \bar{\astrosun} $和$\bar{\oplus} $ 也满足第一分配率.
\item 若$ \astrosun$和$\oplus $ 满足第二分配率, 则 $ \bar{\astrosun} $和$\bar{\oplus} $ 也满足第二分配率.
\end{enumerate}
\end{Theorem}

%\begin{Definition}[自同构]
%对于$\circ$和$\circ$来说的一个$A$与$A$之间的\;\fbox{同构映射}\;叫做一个对于$\circ$来说的$A$的\textbf{自同构}.
%\end{Definition}

\subsection{等价关系与集合分类}

\begin{Definition}[关系\mbox{[Relation]}]
$R: A \times A \rightarrow D = \{\text{对}, \text{错}\} $, 
若
$R(a, b) = \text{对}$
, 称
$(a, b)$
满足关系$R$, 记为$a \, R \, b$.
\end{Definition}

\begin{Definition}[等价关系]
如果$\sim$是$A$的元素间的关系,满足 
\begin{enumerate}[(1)]
\item 自反性, $\forall a \in A, a \sim a$.
\item 对称性, $\forall a, b \in A$, 若$a \sim b$, 则$b \sim a$.
\item 传递性, $\forall a, b, c \in A$, 若$a \sim b$, $b\sim c$, 则$a \sim c$.
\end{enumerate}
则称$\sim$为等价关系.
\end{Definition}

\begin{Definition}[集合分类、划分]
集合$A$分成若干子集,满足 (1) 每个元素属于都某子集 (2) 每个元素只属于某子集. 这些类的全体叫做\textbf{集合$A$的一个分类}.
$$ A = A_1 \cup A_2 \cup \cdots \cup A_n, A_i \cap A_j = \emptyset, i \neq j$$
\end{Definition}

\begin{Theorem}
集合上的一个分类,确定一个集合的元素之间的等价关系.
\end{Theorem}

\begin{Theorem}
集合上的一个等价关系,确定一个集合的分类.
\end{Theorem}

\begin{Definition}[$\mathbb{Z}_p$\mbox{[模$n$的剩余类]}]
{\color{gray}{
$ \{ [0], [1], \cdots, [n-1] \} $, $[i] = \{ k n + i \mid k \in \mathbb{Z} \}$
}}
\end{Definition}
