\documentclass[UTF8]{ctexart}
\usepackage{amsmath}
\usepackage{wasysym}
\usepackage{embrac} % 把括号变正
\usepackage{geometry}
%\usepackage{amssymb}

\geometry{a4paper,left=2cm,right=2cm,top=2cm,bottom=1cm}
\newtheorem{Definition}{\hspace{2em}定义}%[subsection]
\newtheorem{Theorem}[Definition]{\hspace{2em}定理}
\newtheorem{Remark}[Definition]{\hspace{2em}注意}


\begin{document}

\title{近世代数(抽象代数)笔记}
\author{管清文}
\maketitle
\tableofcontents

\section{基本概念}

\subsection{代数运算}
%\begin{Definition}[笛卡尔积] 令 $A_1, A_2, \cdots, A_n$是$n$个集合
%$$ A_1 \times A_2 \times \cdots \times A_n = \{ (a_1, a_2, \cdots, a_n) \mid a_i \in A_i, i = 1, 2, \cdots, n \} $$
%\end{Definition}

\begin{Remark}
近世代数(或抽象代数)的主要内容就是研究所谓\textbf{代数系统},即带有运算的集合。
\end{Remark}

\begin{Definition}[映射]
$$ \begin{aligned}
A_1 \times A_2 \times \cdots \times A_n &\rightarrow D \\
 (a_1, a_2, \cdots, a_n) &\mapsto d = \phi (a_1, a_2, \cdots, a_n) = \overline{(a_1, a_2, \cdots, a_n)} \end{aligned}$$ 
% \begin{itemize}
% 	\item 法则 $\phi$ 叫做集合 $A_1 \times A_2 \times \cdots \times A_n$ 到 集合 $D$ 的一个\textbf{映射}
% 	\item 元素 $d$ 叫做 元素  $ (a_1, a_2, \cdots, a_n) $ 在映射 $
% 	\phi$ 下的\textbf{象}
% 	\item 元素 $ (a_1, a_2, \cdots, a_n) $  叫做元素 $d$ 在 $\phi$下的 \textbf{原象}
% \end{itemize}
\end{Definition}

\begin{Remark}
% 	\begin{itemize}
 		%\item 
 		判断一个法则$\phi$是映射的充要条件: (i) 都有象 (ii) 象唯一.
% 	\end{itemize}
\end{Remark}

\begin{Definition}[代数运算] 
$$\begin{aligned}
A \times B &\rightarrow D \\ 
(a, b) &\mapsto d = \phi(a, b) = \circ (a, b) = a \circ b \end{aligned}$$
\end{Definition}

\begin{Remark}
% 	\begin{itemize}
 %		\item 
 $A = B$时, 对于代数运算$ A \times A \rightarrow D $, $ a \circ b $ 和 $ b \circ a $ 都有意义,但不一定相等.
\end{Remark}
% 	\end{itemize}

\begin{Definition}[$A$的代数运算, 二元运算] 
假如 $ \circ $ 是一个 $ A \times A \rightarrow A$的代数运算(即$A = B = D$),我们说集合$A$对于代数运算$\circ$来说是闭的, 也说, $\circ$是\textbf{$A$的代数运算}或\textbf{二元运算}.
\end{Definition}

\begin{Definition}[结合率]
我们说,一个集合$A$的代数运算$\circ$适合结合律,假如对于$A$的任何三个元$a, b, c$来说都有$$
(a \circ b) \circ c = a \circ (b \circ c)
$$
\end{Definition}

\begin{Definition}
假如对于$A$的$n$ ($n \ge 2$)个固定的元素 $a_1, a_2, \cdots, a_n$来说,所有的加括号方式 $\pi(a_1 \circ a_2 \circ \cdots \circ a_n)$都相等,我们就把这些步骤可以得到的唯一的结果,用$a_1 \circ a_2 \circ \cdots \circ a_n $ 来表示.
\end{Definition}

\begin{Theorem}
若$A$的代数运算$\circ$满足结合律,则对于$A$的任意$n$($n \ge 2$)个元素 $a_1, a_2, \cdots, a_n$来说,对于任意的加括号的方法$\pi$, $\pi(a_1 \circ a_2 \circ \cdots \circ a_n)$ 都相等,$a_1 \circ a_2 \circ \cdots \circ a_n$ 也就总有意义.
\end{Theorem}

\begin{Definition}
$A$上的二元运算$\circ$, $a \circ b = b \circ a$(a与b可交换) $\forall a, b \in A$成立,则称$\circ$满足交换律. 
\end{Definition}

\begin{Theorem}
若$A$上的二元运算$\circ$满足结合律与交换律,则$a_1 \circ a_2 \circ \cdots \circ a_n$ 可以任意交换顺序.
\end{Theorem}

\begin{Definition}
$ \astrosun, \oplus $ 都是 $A$上的二元运算, 
\begin{itemize}
\item 若
$ b \astrosun (a_1 \oplus a_2) = (b \astrosun a_1) \oplus (b \astrosun a_2), \forall b, a_1, a_2 $, 则称 $ \astrosun, \oplus $  满足第一分配率.
\item 若
$ (a_1 \oplus a_2) \astrosun b = (a_1 \astrosun b) \oplus ( a_2 \astrosun b), \forall b, a_1, a_2 $, 则称 $ \astrosun, \oplus $  满足第二分配率.
\end{itemize}
\end{Definition}


\begin{Theorem}
若$A$上的二元运算$\oplus$ 满足结合律, $ \astrosun, \oplus $ 满足第一分配率,则
$$
b \astrosun ( a_1 \oplus a_2 \oplus \cdots \oplus a_n ) =  ( b \astrosun a_1) \oplus (b \astrosun a_2) \oplus \cdots \oplus (b \astrosun a_n)
$$
\end{Theorem}

\subsection{同态}

\begin{Definition}[满射]
映射$\phi: A \rightarrow \bar{A}$被称为\textbf{满射}, 如果
$\forall \hat{a} \in \bar{A}, \exists a \in A \text{ s.t. } \bar{a} = \hat{a}$. 
($\phi^{-1}$都有象)
\end{Definition}

\begin{Definition}[单射]
映射$\phi: A \rightarrow \bar{A}$被称为\textbf{单射}, 如果
$\forall a, b \in A, a \neq b \Rightarrow \bar{a} \neq \bar{b}$.
 ($\phi^{-1}$象唯一)
\end{Definition}

\begin{Definition}[一一映射]
既是满射又是单射.
\end{Definition}

\begin{Remark}
% 	\begin{itemize}
 		%\item 
 		判断一个法则$\phi$是一一映射的充要条件: (i) 都有象 (ii) 象唯一 (iii)满的 (iv) 单的.
\end{Remark}

\begin{Definition}[变换]
从$A$到$A$的映射 $\phi: A \rightarrow A$ 叫$A$上的变换.
\begin{itemize}
	\item 如果$\phi$是满的, 则称为\textbf{满变换}.
	\item 如果$\phi$是单的, 则称为\textbf{单变换}.
	\item 如果$\phi$是一一的, 则称为\textbf{一一变换}.
\end{itemize}
\end{Definition}

\begin{Definition}[同态映射]
对于$\phi: A \rightarrow \bar{A}$, $A$上有二元运算$\circ$, $\bar{A}$上有二元运算$\bar{\circ}$.
如果
$ \overline{a \circ b} = \bar{a} \bar{\circ} \bar{b}$, 则称 $\phi$是 $A$到 $\bar{A}$的同态映射.
\end{Definition}

\begin{Remark}[同态映射判别]
 判断一个法则$\phi$是同态映射的充要条件: \\

\centerline{(i) 都有象 (ii) 象唯一 (iii) $ \overline{a \circ b} = \bar{a} \bar{\circ} \bar{b}$}
\end{Remark}



\section{群}
\section{环}
\section{域}
\section{TODO}
\begin{itemize}
	\item 括号斜体难看
\end{itemize}



\end{document}
