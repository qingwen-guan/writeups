\documentclass[UTF8]{ctexart}
\usepackage{amsmath}
\usepackage{amssymb}
\usepackage{wasysym}
\usepackage{geometry}
\usepackage{ntheorem}
\usepackage{enumerate}
\usepackage[shortlabels]{enumitem} %https://tex.stackexchange.com/questions/229540/customized-enumerate-items

\geometry{a4paper,left=2cm,right=2cm,top=2cm,bottom=1cm}

\theorembodyfont{\upshape}
\newtheorem{Definition}{定义}%[subsection]
\newtheorem{Theorem}[Definition]{定理}
\newtheorem{Remark}[Definition]{注意}
\newtheorem{Lemma}[Definition]{引理}
\newtheorem{Corollary}[Definition]{推论}
\newtheorem{Proposition}[Definition]{命题}


\setenumerate[1]{itemsep=0pt,partopsep=0pt,parsep=\parskip,topsep=0pt}
\setitemize[1]{itemsep=0pt,partopsep=0pt,parsep=\parskip,topsep=0pt}

\begin{document}

\title{近世代数(抽象代数)笔记}
\author{管清文}
\maketitle
\tableofcontents
\clearpage

\section{基本概念}

\subsection{代数运算}

\begin{Remark}
近世代数(或抽象代数)的主要内容就是研究所谓\textbf{代数系统},即带有运算的集合。
\end{Remark}

\begin{Definition}[映射]
$$ \begin{aligned}
A_1 \times A_2 \times \cdots \times A_n &\rightarrow D \\
 (a_1, a_2, \cdots, a_n) &\mapsto d = \phi (a_1, a_2, \cdots, a_n) = \overline{(a_1, a_2, \cdots, a_n)} \end{aligned}$$ 
\end{Definition}

\begin{Remark}
判断一个法则$\phi$是映射的充要条件: (i) 都有象 (ii) 象唯一.
\end{Remark}

\begin{Definition}[代数运算] 
$$\begin{aligned}
A \times B &\rightarrow D \\ 
(a, b) &\mapsto d = \phi(a, b) = \circ (a, b) = a \circ b \end{aligned}$$
\end{Definition}

\begin{Remark}
$A = B$时, 对于代数运算$ A \times A \rightarrow D $, $ a \circ b $ 和 $ b \circ a $ 都有意义,但不一定相等.
\end{Remark}

\begin{Definition}[$A$的代数运算, 二元运算] 
假如 $ \circ $ 是一个 $ A \times A \rightarrow A$的代数运算(即$A = B = D$),我们说集合$A$对于代数运算$\circ$来说是闭的, 也说, $\circ$是\textbf{$A$的代数运算}或\textbf{二元运算}.
\end{Definition}

\subsection{运算律}

\begin{Definition}[结合率]
我们说,一个集合$A$的代数运算$\circ$满足结合律,假如对于$A$的任何三个元素$a, b, c$来说都有$$
(a \circ b) \circ c = a \circ (b \circ c)
$$
\end{Definition}

\begin{Definition}
假如对于$A$的$n$ ($n \ge 2$)个固定的元素 $a_1, a_2, \cdots, a_n$来说,所有的加括号方式 $\pi(a_1 \circ a_2 \circ \cdots \circ a_n)$都相等,我们就把这些步骤可以得到的唯一的结果,用$a_1 \circ a_2 \circ \cdots \circ a_n $ 来表示.
\end{Definition}

\begin{Theorem}
若$A$的代数运算$\circ$满足结合律,则对于$A$的任意$n$($n \ge 2$)个元素 $a_1, a_2, \cdots, a_n$来说,对于任意的加括号的方法$\pi$, $\pi(a_1 \circ a_2 \circ \cdots \circ a_n)$ 都相等,$a_1 \circ a_2 \circ \cdots \circ a_n$ 也就总有意义.
\end{Theorem}

\begin{Definition}[交换律]
$A$上的二元运算$\circ$, $a \circ b = b \circ a$\;(a与b可交换) $\forall a, b \in A$成立,则称$\circ$满足\textbf{交换律}. 
\end{Definition}

\begin{Theorem}
若$A$上的二元运算$\circ$满足结合律与交换律,则$a_1 \circ a_2 \circ \cdots \circ a_n$ 可以任意交换顺序.
\end{Theorem}

\begin{Definition}[分配率]
$ \astrosun $和$ \oplus $ 都是 $A$上的二元运算, 
\begin{enumerate}[i)]
\item 若
$ a \, \astrosun \, (b \oplus c) = (a \, \astrosun \, b) \oplus (a \, \astrosun \, c), \forall a, b, c $, 则称 $ \astrosun$和$ \oplus $  满足第一分配率.
\item 若
$ (a \oplus b) \, \astrosun \, c = (a \, \astrosun \, c) \oplus ( b \, \astrosun \, c), \forall a, b, c$, 则称 $ \astrosun$和$\oplus $  满足第二分配率.
\end{enumerate}
\end{Definition}


\begin{Theorem}
若$A$上的二元运算$\oplus$ 满足结合律, $ \astrosun $和$\oplus $ 满足第一分配率,则
$$
a \, \astrosun \, ( b_1 \oplus b_2 \oplus \cdots \oplus b_n ) =  ( a \, \astrosun \, b_1) \oplus (a \, \astrosun \, b_2) \oplus \cdots \oplus (a \, \astrosun \, b_n)
$$
\end{Theorem}

\begin{Theorem}
若$A$上的二元运算$\oplus$ 满足结合律, $ \astrosun $和$\oplus $ 满足第二分配率,则
$$
( a_1 \oplus a_2 \oplus \cdots \oplus a_n ) \, \astrosun \, b =  ( a_1 \, \astrosun \, b) \oplus ( a_2 \, \astrosun \, b) \oplus \cdots \oplus (  a_n \, \astrosun \, b)
$$
\end{Theorem}

\subsection{同态}

\begin{Definition}[满射]
映射$\phi: A \rightarrow \bar{A}$被称为\textbf{满射}, 如果
$\forall \hat{a} \in \bar{A}, \exists a \in A \text{ s.t. } \bar{a} = \hat{a}$. 
($\phi^{-1}$都有象)
\end{Definition}

\begin{Definition}[单射]
映射$\phi: A \rightarrow \bar{A}$被称为\textbf{单射}, 如果
$\forall a, b \in A, a \neq b \Rightarrow \bar{a} \neq \bar{b}$.
 ($\phi^{-1}$象唯一)
\end{Definition}

\begin{Definition}[一一映射]
既是满射又是单射.
\end{Definition}

\begin{Remark}[一一映射判别]
(i) 是映射(都有象、象唯一) (ii)满的 (iii) 单的.
\end{Remark}

\begin{Definition}[变换]
从$A$到$A$的映射 $\tau: A \rightarrow A, a \mapsto \tau(a) = a^{\tau}$ 叫$A$上的变换.
\begin{itemize}
	\item 如果$\tau$是满的, 则称为\textbf{满变换}.
	\item 如果$\tau$是单的, 则称为\textbf{单变换}.
	\item 如果$\tau$是一一的, 则称为\textbf{一一变换}.
\end{itemize}
\end{Definition}

\begin{Definition}[同态映射]
对于$\phi: A \rightarrow \bar{A}$, $A$上有二元运算$\circ$, $\bar{A}$上有二元运算$\bar{\circ}$.
如果
$ \overline{a \circ b} = \bar{a} \, \bar{\circ} \, \bar{b}$, 则称 $\phi$是 $A$到 $\bar{A}$的同态映射.
\end{Definition}

\begin{Remark}[同态映射判别]
(i) 是映射(都有象、象唯一) (ii) $ \overline{a \circ b} = \bar{a} \,{\bar{\circ}}\, \bar{b}$
\end{Remark}

\begin{Definition}[同态满射、同态]
如果$A$到$\bar{A}\;${\fbox{存在}}\;一个同态映射$\phi$, 且它是满的, 则称$A$与$\bar{A}$\;(关于$\circ$与$\bar{\circ}$来说)\textbf{同态}. 称这个映射是一个\textbf{同态满射}.
\end{Definition}

\begin{Remark}[同态满射判别]
(i) 是映射(都有象、象唯一) (ii) 同态 (iii) 满
\end{Remark}

\begin{Definition}[同构映射、同构]
如果$A$到$\bar{A}\;${\fbox{存在}}\;一个同态映射$\phi$, 且它是既是满的又是单的(一一的), 则称$A$与$\bar{A}$(关于$\circ$与$\bar{\circ}$)\textbf{同构}, 记为$A \cong
 \bar{A}$. 称这个映射是一个(关于$\circ$与$\bar{\circ}$的)\textbf{同构映射}(简称\text{同构}).
\end{Definition}

\begin{Remark}[同构映射判别]
(i) 是映射(都有象、象唯一) (ii) 同态 (iii) 满 (iv) 单
\end{Remark}

\begin{Theorem}
假定对于代数运算$\circ$和$\bar{\circ}$来说, $A$与$\bar{A}$同态, 那么
\begin{enumerate}[i)]
\item 若 $\circ$ 满足结合律, $\bar{\circ}$也满足结合律;
\item 若 $\circ$ 满足交换律, $\bar{\circ}$也满足交换律.
\end{enumerate}
\end{Theorem}

\begin{Theorem}
$ \astrosun $和$ \oplus $ 是 $A$的两个代数运算, 
$ \bar{\astrosun} $和$\bar{\oplus} $ 是 $\bar{A}$的两个代数运算,
有
$\phi$既是$A$与$\bar{A}$的关于$ \astrosun $和$\bar{\astrosun}$ 的同态满射,
$\phi$也是$A$与$\bar{A}$的关于$ \oplus $和$\bar{\oplus}$ 的同态满射,
则 
\begin{enumerate}[i)]
\item 若$ \astrosun$和$ \oplus $  满足第一分配率, 则 $ \bar{\astrosun} $和$\bar{\oplus} $ 也满足第一分配率.
\item 若$ \astrosun$和$\oplus $  满足第二分配率, 则 $ \bar{\astrosun} $和$\bar{\oplus} $ 也满足第二分配率.
\end{enumerate}
\end{Theorem}

\begin{Definition}[自同构]
对于$\circ$和$\circ$来说的一个$A$与$A$之间的\;\fbox{同构映射}\;叫做一个对于$\circ$来说的$A$的\textbf{自同构}.
\end{Definition}

\subsection{等价关系与集合分类}

\begin{Definition}[关系\mbox{[Relation]}]
$R: A \times A \rightarrow D = \{\text{对}, \text{错}\} $, 
若
$R(a, b) = \text{对}$
, 称
$(a, b)$
满足关系$R$, 记为$a \, R \, b$.
\end{Definition}

\begin{Definition}[等价关系]
如果$\sim$是$A$的元素间的关系,满足 
\begin{enumerate}[i)]
\item 自反性, $\forall a \in A, a \sim a$.
\item 对称性, $\forall a, b \in A$, 若$a \sim b$, 则$b \sim a$.
\item 传递性, $\forall a, b, c \in A$, 若$a \sim b$, $b\sim c$, 则$a \sim c$.
\end{enumerate}
则称$\sim$为等价关系.
\end{Definition}

\begin{Definition}[集合分类、划分]
集合$A$分成若干子集,满足 (i) 每个元素属于都某子集 (ii) 每个元素只属于某子集. 这些类的全体叫做\textbf{集合$A$的一个分类}.
$$ A = A_1 \cup A_2 \cup \cdots \cup A_n, A_i \cap A_j = \emptyset, i \neq j$$
\end{Definition}

\begin{Theorem}
集合上的一个分类,确定一个集合的元素之间的等价关系.
\end{Theorem}

\begin{Theorem}
集合上的一个等价关系,确定一个集合的分类.
\end{Theorem}

\begin{Definition}[模$n$的剩余类]
$ \{ [0], [1], \cdots, [n-1] \} $, $[i] = \{ k n + i \mid k \in \mathbb{Z} \}$
\end{Definition}

\section{群论}

\subsection{群的定义和性质}

\begin{Remark}
群是一个代数系统(定义代数运算的集合), 其中群里只有一个代数运算. 便利起见$\phi(a, b) = a \circ b$写成$a b$
\end{Remark}

\begin{Definition}[群\mbox{[Group]}的第一定义]
在集合$G \neq \emptyset$上规定一个叫做乘法的\;\fbox{代数运算}\;. 这个代数系统被称为群, 如果
\begin{itemize}
	\item[\uppercase\expandafter{\romannumeral1}] 乘法封闭, $\forall a, b \in G, ab \in G$
	\item[\uppercase\expandafter{\romannumeral2}] 乘法结合, $\forall a, b, c \in G, (ab)c = a(bc)$
	\item[\uppercase\expandafter{\romannumeral3}] $ \forall a, b \in G$, $ax = b, ya = b$在$G$中都有解.
\end{itemize}
\end{Definition}

\begin{Remark}[乘法]
以后提到乘法,都是指某个集合$A$上的代数运算$A \times A \rightarrow A$, 自然要求I(乘法封闭).
\end{Remark}

\begin{Theorem}[左单位元]
对于群$G$中至少有一个元$e$, 叫做$G$的一个\textbf{左单位元},使得$\forall a \in G$都有 $ea = a$.
\end{Theorem}

\begin{Theorem}[左逆元]
对于群$G$中的任何一个元素$a$, 在$G$中存在一个元$a^{-1}$,叫做$a$的\textbf{左逆元}, 能让$a^{-1} a = e$.
\end{Theorem}

\begin{Definition}[群\mbox{[Group]}的第二定义]
在集合$G \neq \emptyset$上规定乘法. 这个代数系统被称为\textbf{群}, 如果
\begin{itemize}
	\item[\uppercase\expandafter{\romannumeral1}] 乘法封闭
	\item[\uppercase\expandafter{\romannumeral2}] 乘法结合
	\item[IV] 左单位元: $\exists e \in G$使 $ea =a$ 对 $\forall a \in G$都成立.
	\item[V] 左逆元: $\forall a \in G, \exists \, a^{-1}$使$a^{-1}a = e$.
\end{itemize}
\end{Definition}

\begin{Definition}[群的阶]
如果$|G|$有限, 称其为\textbf{有限群}, 称他的\textbf{阶}是$G$的元素个数. \\
如果$G$中有无穷多个元素, 称其为\textbf{无限群}, 称他的\textbf{阶}无限.
\end{Definition}

\begin{Definition}[交换群、Abel群]
群中交换律不一定成立, 如果乘法满足交换律($\forall a, b \in G, ab = ba$), 则称之为\textbf{交换群}(\textbf{Abel群}).
\end{Definition}

\begin{Theorem}[单位元]
在一个群$G$里存在且只存在一个元$e$, 使得$ea = ae = a$对于$\forall a \in G$成立. 这个元素被称为群$G$的\textbf{单位元}.
\end{Theorem}

\begin{Theorem}[逆元]
对于群$G$的任意一个元素$a$来说, 有且只有一个元素$a^{-1}$, 
使 $a^{-1} a = a a^{-1} = e$. 这个元素被称为$a$的\textbf{逆元}, 或者简称\textbf{逆}.
\end{Theorem}

\begin{Remark}
证明$a^{-1}$是$a$的逆的方法: $a^{-1}a = e$或者$aa^{-1} = e$
\end{Remark}

\begin{Definition}
规定$a^n = \underbrace{a a \cdots a}_{n\text{个}}, a^0 = e, a^{-n} = (a^{-1})^n, n \in \mathbb{Z}^{+}$
\end{Definition}

\begin{Theorem}
$ a^n a^m = a^{n+m}, (a^n)^m, n, m \in \mathbb{Z} $ (灰色: 它说明了${(a^{-1})}^{-1} = a$)
\end{Theorem}

\begin{Proposition}[乘积的逆等于逆的乘积]
$\forall a, b \in G, {(ab^{-1})}^{-1} = b a^{-1}$
\end{Proposition}

\begin{Definition}[元素的阶]
在一个群$G$中,使得$a^n = e$的最小正整数, 叫做$a$的\textbf{阶}. 若这样的$n$不存在, 称$a$是无穷阶的,或者叫$a$的阶是无穷.
\end{Definition}

\begin{Theorem}
假定群的元$a$的阶是$n$, 则$a^r$的阶是$\displaystyle \frac{n}{\text{gcd}(r, n)}$.
\end{Theorem}

\begin{Theorem}[III'\mbox{[消去律]}]
群的乘法满足: $ax = ax' \Rightarrow x = x', ya = y'a \Rightarrow y = y'$
\end{Theorem}

\begin{Corollary}
在群里, $ax = b$ 和 $ya = b$都有唯一解.
\end{Corollary}

\begin{Theorem}[有限群的另一定义]
一个带有乘法的 \fbox{有限集合}  $G \neq \emptyset$, 若满足I、II、III', 则$G$是一个\textbf{群}.
\end{Theorem}

\subsection{群的同态}

\begin{Theorem}
$G$与$\bar{G}$关于他们的乘法同态, 则 $G$是群$\Rightarrow \bar{G}$也是群.
\end{Theorem}

\begin{Theorem}
假定$G$和$\bar{G}$是两个群, 在$G$到$\bar{G}$的一个同态满射之下, $G$的单位元$e$的象是$\bar{G}$的单位元, $G$的元$a$的逆元$a^{-1}$的象是$a$的象的逆元($\overline{a^{-1}} = \bar{a}^{-1}$).
\end{Theorem}

\begin{Remark}
总结下来, 如果$A$与$\bar{A}$同态,那么前者有什么后面就也有什么:
\begin{itemize}
	\item 前面有结合,后面就也有结合
	\item 前面有交换,后面就也有交换
	\item 前面有分配,后面就也有分配
	\item 前面是群,后面就也是群
\end{itemize}
\end{Remark}

\begin{Theorem}
$G$与$\bar{G}$关于他们的乘法同构, 则 $G$是群$\Leftrightarrow \bar{G}$是群.
\end{Theorem}

\subsection{变换群}

\begin{Definition}[变换的乘法]
$\tau_1 \tau_2: a \mapsto {\left(a^{\tau_1}\right)}^{\tau_2}$
\end{Definition}

\begin{Theorem}[变换乘法结合]
$(\tau_1 \tau_2) \tau_3 = \tau_1 ( \tau_2 \tau_3 )$
\end{Theorem}

\begin{Theorem}
$G$是集合$A$的若干变换构成的集合, 如果$G$基于变换的乘法做成一个群,
则$G$中的变换一定是一一变换.
\end{Theorem}

\begin{Definition}[变换群]
如果一个集合$A$的若干\;\fbox{一一变换}\;对于变换的乘法能够做成一个群,则称这个群为A的一个\textbf{变换群}.
\end{Definition}

\begin{Theorem}
一个集合$A$上的所有一一变换做成一个变换群$G$.
\end{Theorem}

\begin{Theorem}
任何一个群都与一个变换群同构.
\end{Theorem}

\begin{Theorem}
一个变换群的单位元一定是恒等变换.
\end{Theorem}

\subsection{置换群}

\begin{Definition}[置换]
\fbox{有限集合}\;上的\;\fbox{一一变换}\;叫做\textbf{置换}, 一般用$\pi$表示.
\end{Definition}

\begin{Definition}[置换群]
有限集合上的若干置换做成的群叫\textbf{置换群}.
\end{Definition}

\begin{Definition}[对称群]
一个$n$元集合$A = \{ a_1, a_2, \cdots, a_n \}$上的所有置换
(有$n!$个)做成的群叫做$n$次\textbf{对称群}, 用$S_n$来表示.
\end{Definition}

\begin{Theorem}
$$
\left.
\begin{aligned}
\pi_1 &= \begin{pmatrix} 
j_1       & \cdots & j_k       & j_{k+1} & \cdots & j_n \\
j_1^{(1)} & \cdots & j_k^{(1)} & j_{k+1} & \cdots & j_n \\
\end{pmatrix} \\
\pi_2 &= \begin{pmatrix} 
j_1       & \cdots & j_k       & j_{k+1}       & \cdots & j_n      \\
j_1       & \cdots & j_k       & j_{k+1}^{(2)} & \cdots & j_n^{(2)} \\
\end{pmatrix} 
\end{aligned}
\right\}
\Rightarrow
\pi_1 \pi_2 = \begin{pmatrix} 
j_1       & \cdots & j_k       & j_{k+1}       & \cdots & j_n \\
j_1^{(1)} & \cdots & j_k^{(1)} & j_{k+1}^{(2)} & \cdots & j_n^{(2)} \\
\end{pmatrix}
$$
\end{Theorem}

\begin{Definition}[$k$-循环置换]
如果$S_n$中的置换满足$a_{i_1}$的象是$a_{i_2}$, $a_{i_2}$的象是$a_{i_3}$, $\cdots$, 
$a_{i_{k-1}}$的象是$a_{i_k}$, $a_{i_{k}}$的象是$a_{i_1}$, 其他元素,如果还有的话,象是不变的, 则称之为\textbf{$k$-循环置换}.
用$(i_1 \, i_2 \, i_3 \, \cdots \, i_{k-1} \, i_k)$ 或 
$(i_2 \, i_3 \, \cdots \, i_{k-1} \, i_k \, i_1)$ 或 $\cdots$ 或   
$( i_k \, i_1 \, i_2 \, i_3 \, \cdots \, i_{k-1})$来表示.
\end{Definition}

\begin{Theorem}
$(i_1 \, i_2 \, \cdots \i_k)^{-1} = (i_k \, \cdots \, i_2 \, i_1)$.
\end{Theorem}

\begin{Theorem}
$k$-循环置换的阶是$k$.
\end{Theorem}

\begin{Theorem}
任何一个置换都可以写成若干没有共同数字的循环置换的乘积.
\end{Theorem}

\begin{Theorem}
两个没有共同数字的循环置换可以交换.
\end{Theorem}

\begin{Theorem}
任何一个有限群都与一个置换群同构.
\end{Theorem}

\subsection{循环群}

\begin{Definition}[循环群]
若一个群$G$的每一个元都是$G$的某一固定元$a$的乘方, 我们就称$G$是一个\textbf{循环群}, $a$是$G$的一个\textbf{生成元}, 并记$G = (a)$, 且说$G$是由元$a$生成的。
\end{Definition}

\begin{Definition}[$\mathbb{Z}_n$\mbox{[模$n$的剩余类加群]}]
$G$包含所有模$n$的剩余类,$G = \{ [0], [1], \cdots, [n-1] \}$, 定义乘法(叫做加法) $[a] + [b] = [a +b]$,可以证明$(G, +)$做成一个群, 叫做\textbf{模$n$的剩余类加群}.
\end{Definition}

\begin{Theorem}
假定$G$是由$a$生成的循环群, 则$G$的构造可以完全由$a$的阶来决定:
\begin{itemize}
\item 如果$a$的阶无限, 则$G \cong \mathbb{Z}$.
\item 如果$a$的阶为$n$, 则$G \cong \mathbb{Z}_n$. !灰色(自然$|G| = n$, 或者说$|(a)| = n$)灰色!
\end{itemize}
\end{Theorem}

\begin{Theorem}
一个循环群一定是交换群.
\end{Theorem}

\begin{Theorem}
$a$生成一个阶是$n$的循环群$G$, 则$a^r$也生成$G$,如果$\text{gcd}(r, d) = 1$.
\end{Theorem}

\begin{Theorem}
$G$是循环群, 且$G$与$\bar{G}$同态,则$\bar{G}$也是循环群.
\end{Theorem}

\begin{Theorem}
$G$是无限阶循环群, $\bar{G}$是任何循环群, 则$G$与$\bar{G}$同态.
\end{Theorem}

\subsection{子群}

\begin{Definition}[子群]
如果一个群$G$的一个子集$H$关于群$G$的乘法也能做成一个群,则称$H$为$G$的一个\textbf{子群}.
\end{Definition}

\begin{Theorem}
一个群$G$的一个非空子集$H$做成$G$的子群,当且仅当
\begin{enumerate}[(i)]
\item $a, b \in H \Rightarrow ab \in H$
\item $a \in H \Rightarrow a^{-1} \in H$
\end{enumerate}
\end{Theorem}

\begin{Corollary}
若$H$是$G$的子群, 则, $H$的单位元就是$G$的单位元, $a$在$H$中的逆就是$a$的$G$中的逆.
\end{Corollary}

\begin{Theorem}
一个群$G$的一个非空子集$H$做成$G$的子群,当且仅当 (iii) $a, b \in H \Rightarrow ab^{-1} \in H$
\end{Theorem}

\begin{Theorem}
一个群$G$的一个非空\;\fbox{有限}\;子集$H$做成$G$的子群,当且仅当 (i) $a, b \in H \Rightarrow ab \in H$
\end{Theorem}

\begin{Remark}[验证非空集合是群的方法]
(1) I, II, III (2) I、II、IV, V (3) 有限集: I, II, III'
(4) 子群: (i), (ii) \mbox{(5) 子群: (iii)} (6) 有限子群: (i)
\end{Remark}

\begin{Definition}[生成子群]
对于群$G$的非空子集$S$, 包含$S$的最小子群, 被称为由$S$生成的子群, 记为$(S)$.
\end{Definition}

\begin{Theorem}
$S = \{a\}$时, $(S) = (a)$.
\end{Theorem}

\begin{Proposition}
群$G$的两个子群的交集也是$G$的子群.
\end{Proposition}

\begin{Proposition}
循环群的子群也是循环群.
\end{Proposition}

\begin{Proposition}
$H$是群$G$的一个非空子集, 且$H$的每个元素的阶都有限,
则$H$做成子群的充要条件是 (i) $a, b \in H \Rightarrow ab \in H$.
\end{Proposition}

\subsection{子群的陪集}

\begin{Definition}
群$G$, 子群$H$, 规定$G$上的关系$\sim: a \sim b \Leftrightarrow a b^{-1} \in H$
\end{Definition}

\begin{Theorem}
上面规定的关系$\sim$是等价关系.
\end{Theorem}

\begin{Definition}[右陪集]
由上述等价关系确定集合的分类叫做$H$的\textbf{右陪集}.
\end{Definition}

\begin{Theorem}
包含元$a$的右陪集$ = Ha =  \{ ha \mid h \in H \}$
\end{Theorem}

\begin{Definition}
群$G$, 子群$H$, 规定$G$上的关系$\sim': a \sim 'b \Leftrightarrow b^{-1} a \in H$. 可以证明$~'$是等价关系.
\end{Definition}

\begin{Definition}[左陪集]
由上述等价关系$\sim': a \sim' b \Leftrightarrow b^{-1} a \in H$, 确定集合的分类叫做$H$的\textbf{左陪集}, 包含元$a$的左陪集可以用$aH = \{ ah \mid h \in H \}$表示.
\end{Definition}

\begin{Theorem}
一个子群的右陪集与左陪集个数相等: 个数或者都是无穷大, 或者都有限且相等.
\end{Theorem}

\begin{Definition}[指数]
一个群$G$的一个子群$H$的右陪集(或左陪集)的个数叫做$H$在$G$里的指数.
\end{Definition}

\begin{Theorem}
右陪集所含元素的个数等于子群$H$所含元素的个数.
\end{Theorem}

\begin{Theorem}
$H$是一个有限群$G$的子群, 那么$H$的阶$n$和他在$G$中的指数$j$都能整除$G$的阶$N$, 并且$N = nj$
\end{Theorem}

\begin{Theorem}[元素的阶整除群的阶]
一个有限群$G$的任何一个元$a$的阶能够整除$G$的阶$|G|$.
\end{Theorem}

\begin{Remark}
待证明: 阶是$n$的元素生成的循环子群的阶是$n$.
\end{Remark}

\begin{Proposition}
阶是素数的群一定是循环群.
\end{Proposition}

\begin{Proposition}
阶是$p^m$的群($p$是素数)一定包含一个阶是$p$的子群.
\end{Proposition}

\begin{Proposition}
若我们把同构的群看做一样的,一共只存在两个阶是4的群,它们都是交换群.
\end{Proposition}

\begin{Proposition}
有限非交换群至少有6个元素.
\end{Proposition}

\subsection{不变子群、商群}

\begin{Definition}[不变子群]
群$G$的子群$N$叫做$G$的\textbf{不变子群}, 如果$\forall a \in G$, 有$Na = aN$. 一个不变子群$N$的一个左(或右)陪集叫做$N$的一个\textbf{陪集}.
\end{Definition}

\begin{Definition}
$S_1, S_2, \cdots, S_m \subseteq $群$G$, 规定子集的乘法
$S_1 S_2 \cdots S_m = \{s_1 s_2 \cdots s_m \mid s_i \in S_i \}$. 可以证明这个乘法满足结合律.
\end{Definition}

\begin{Theorem}
已知一个群$G$有一个子群$N$, $N$是不变子群的充要条件是$aNa^{-1} = N, \forall a \in G$.
\end{Theorem}

\begin{Theorem}
已知一个群$G$有一个子群$N$, $N$是不变子群的充要条件是
$a \in G, n \in N \Rightarrow ana^{-1} \in N $.
\end{Theorem}

\begin{Theorem}
如果$N$刚好包含$G$的所有具有以下性质的元$n$,
$$
	na = an, \forall a \in G
$$
则$N$是$G$的不变子群. 我们称这个不变子群是G的\textbf{中心}.
\end{Theorem}

\begin{Theorem}
$N$是群$G$的不变子群, 在其陪集$\{ aN, bN, cN, \cdots \}$上定义的乘法$(xN, yN) \mapsto (xy)N$, 则这个乘法是此陪集的二元运算,且此陪集对于上面规定的乘法来说构成一个群.
\end{Theorem}

\begin{Definition}[商群]
一个群$G$的一个不变子群$N$的所有陪集关于陪集的乘法做成的群叫做$G$的\textbf{商群},用$G/N$表示.
\end{Definition}

\begin{Theorem}
对于有限群, $ \displaystyle | G/N | $ = $\frac{|G|}{|N|}$.
\end{Theorem}

\begin{Proposition}
两个不变子群的交集还是不变子群.
\end{Proposition}

\begin{Proposition}
$H$是$G$的子群, $N$是$G$的不变子群, 则$HN$是$G$的子群.
\end{Proposition}

\subsection{同态与不变子群}

\begin{Theorem}
一个群$G$与它的商群$G/N$同态.
\end{Theorem}

\begin{Definition}[核]
$\phi$是群$G$到群$\bar{G}$的一个同态满射, $\bar{G}$的单位元$\bar{e}$在$\phi$之下的所有原象做成的$G$的子集叫做$\phi$的\textbf{核}.
\end{Definition}

\begin{Theorem}
$G$和$\bar{G}$是两个群,且$G$与$\bar{G}$同态,则这个同态满射的核$N$是$G$的一个不变子群,且$G/N \cong \bar{G}$.
\end{Theorem}

\begin{Remark}
一个群只和``相当于''它的商群同态
\end{Remark}

\begin{Definition}
$\phi$是$A \rightarrow \bar{A}$的满射, 取$S \subseteq A$, 定义$S$的象是$S$中所有元素的象做成的集合. 取$\bar{S} \subseteq \bar{A}$, 定义$\bar{S}$的原象是$\bar{S}$中所有元素的原象做成的集合.
\end{Definition}

\begin{Theorem}
$G$和$\bar{G}$是两个群,且$G$与$\bar{G}$同态,则在这个同态满射之下:
\begin{itemize}
\item[(1)] $G$的一个子群$H$的象$\bar{H}$也是$\bar{G}$的一个子群.
\item[(2)] $G$的一个不变子群$N$的象$\bar{N}$也是$\bar{G}$的一个不变子群.
\item[(1')] $\bar{G}$的一个子群$\bar{H}$的原象$H$也是$G$的一个子群.
\item[(2')] $\bar{G}$的一个不变子群$\bar{N}$的原象$N$也是$G$的一个不变子群.
\end{itemize}
\end{Theorem}

\begin{Remark}
这也体现了同态的性质,前面有的后面也有!
\end{Remark}

\begin{Proposition}
假定群$G$与群$\bar{G}$同态, $\bar{N}$是$\bar{G}$的不变子群, $N$是$\bar{N}$的逆象,则
$ G/N \sim \bar{G}/\bar{N} $.
\end{Proposition}

\begin{Proposition}
假定群$G$与$\bar{G}$是两个有限循环群,他们的阶各是$m$和$n$,则$G$与$\bar{G}$同态$\Leftrightarrow n \mid m$
\end{Proposition}

\begin{Proposition}
假定群$G$是一个循环群, $N$是$G$的一个子群, 则$G/N$也是循环群.
\end{Proposition}

\section{环与域}

\begin{Definition}[加群]
一个交换群叫做一个的\textbf{加群}, 如果我们把这个群的代数运算称为加法, 并且用符号$+$表示.
\end{Definition}

\begin{Definition}
$n$的元的和$a_1 + a_2 + \cdots + a_n$用符号$\displaystyle \sum_{i=1}^{n} a_n$ 来表示.
\end{Definition}

\begin{Definition}
$n$个$a$的和$\displaystyle \sum_{i=1}^{n} a$我们用$na$表示.
\end{Definition}

\begin{Definition}[零元]
加群唯一的单位元用$0$来表示, 并且把它叫做\textbf{零元}.
\end{Definition}

\begin{Definition}[负元]
元$a$的唯一的逆元我们用$-a$来表示,并且把它叫做$a$的\textbf{负元}. $a + (-b)$我们简写成$a - b$.
\end{Definition}

\begin{Theorem}
加群满足以下运算规则
\begin{enumerate}[(1)]
\item $0 + a = a + 0 = a$
\item $-a + a = a - a = 0$
\item $-(-a) = a$
\item[(4: 移项)] $a + c = b \Leftrightarrow c = b - a$
\item $-(a +b) = -a - b, -(a-b) = -a +b$
\item $ma + na = (m+n)a, m(na) = (mn)a, n(a+b) = na + nb, \forall m, n \in \mathbb{Z}^+$
\end{enumerate}
\end{Theorem}

\begin{Remark}
非空子集$S$做成子群的充要条件变成了 
\begin{itemize}
\item (i) $a, b \in S \Rightarrow a+b \in S$ (ii) $a \in S \Rightarrow -a \in S$
\item 或者 (iii) $a, b \in S \Rightarrow a - b \in S$.
\end{itemize}
\end{Remark}

\begin{Definition}[环]
一个集合$R$叫做一个环, 如果
\begin{enumerate}
	\item $R$是一个加群: $R$关于一个叫做加法的代数运算做成一个交换群.
	\item $R$对于另一个叫做乘法的代数运算是封闭的.
	\item $R$关于乘法结合
	\item 分配率: $a(b+c) = bc + ac, (a+b)c = ac + bc$
\end{enumerate}
\end{Definition}

\begin{Remark}[环的判别] 判断一个代数系统是环的条件: 
\begin{enumerate}[(i)]
\item 加法(构成交换群): I.加法封闭 II 加法结合 IV 有零元 V 负元 X 加法交换
\item 乘法: I.乘法封闭 II 乘法结合
\item 两种运算的关系: 分配率
\end{enumerate}
\end{Remark}

\begin{Theorem}
环还满足以下运算规则
\begin{enumerate}[(1)]
\item[(7)] $(a-b)c = ac - bc, c(a - b) = ca - cb$
\item[(8)] $0a = a0 = 0$
\item[(9)] $(-a)b = a(-b) = -(ab)$
\item[(10)] $(-a)(-b) = ab$
\item[(11)] $a(b_1 + b_2 + \cdots + b_n) = ab_1 +ab_2 + \cdots + ab_n, (b_1 + b_2 + \cdots + b_n)a = b_1a + b_2a + \cdots + b_na$
\item[(12)] 
$ \displaystyle \left( \sum_{i=1}^m a_i \right) \left( \sum_{j=1}^n b_j \right) 
= \sum_{a=1}^m \sum_{b=1}^{n} a_i b_j $
           $$
\begin{aligned}
(a_1 + a_2 + \cdots + a_m) (b_1 + b_2 + \cdots + b_n) 
= 
a_1b_1 + a_1 b_2 + &\cdots + a_1 b_n \\
+ a_2 b_1 + a_2 b_2 + &\cdots + a_2 b_n \\
+ &\cdots \\
+ a_m b_1 + a_m b_2 + &\cdots + a_m b_n
\end{aligned}
 $$
 \item[(13)] $ (na)b = a(nb) = n(ab), n \in \mathbb{Z}^+$ 
 \item[(14)] 规定 $a^n = \underbrace{a a \cdots a}_{n\text{个}}, n \in \mathbb{Z}^{+}$, 则 $ a^m a^n = a^{m+n}, (a^m)^n = a^{mn} $
\end{enumerate}
\end{Theorem}

\subsection{交换律、单位元、零因子、整环}

\begin{Definition}[交换环]
一个环$R$叫做\textbf{交换环}, 如果$ab = ba,  \forall a, b \in R$.
\end{Definition}

\begin{Proposition}
在一个交换环中${(ab)}^n = a^n b^n$.
\end{Proposition}

\begin{Definition}[单位元]
对于环$R$, 如果$ea = ae = a, \forall a \in R$, 则称$e$是环$R$的单位元. 一般,一个环未必有单位元.
\end{Definition}

\begin{Proposition}
一个环如果有单位元, 则唯一. 用$1$来表示.
\end{Proposition}

\begin{Definition}[整数环]
整数关于普通加法和乘法构成的环.
\end{Definition}

\begin{Definition}[逆元]
若$ba = 1$, 则称$b$为$a$的\textbf{左逆元}. 若$ba = ab = 1$, 则称$b$为$a$的\textbf{逆元}. 
\end{Definition}

\begin{Proposition}
如果$a$有逆元, 则唯一.
\end{Proposition}

\begin{Proposition}
如果$a$有逆元, 则规定$a^{-m} = {(a^{-1})}^m, a^0 = 1$. 则$a^m a^n = a^{m+n}, {(a^m)}^n = a^{mn}, \forall m, n \in \mathbb{Z}$.
\end{Proposition}

\begin{Proposition}[模$n$的剩余类环]
$R = \{ [0], [1], \cdots, [n-1] \}$, 加法: $[a] + [b] = [a+b]$, 乘法: $[a][b] = [ab]$做成一个交换环, 被称为\textbf{模$n$的剩余类环}, 零元$0 = [0]$, 单位元$1 = [1]$.
\end{Proposition}

\begin{Proposition}
$ab = 0 \Rightarrow a = 0 $ 或者 $b = 0$ 在环里不一定对.
\end{Proposition}

\begin{Definition}[零因子]
在一个环$R$中, 若$a \neq 0, b \neq 0$但$ab = 0$, 则称$a$是$R$的\textbf{左零因子}, $b$是$R$的\textbf{右零因子}.
\end{Definition}

\begin{Remark}
左零因子不一定是右零因子. 但是如果有左零因子, 就一定有右零因子. 如果$R$是交换环, 则左零因子一定是右零因子.
\end{Remark}

\begin{Theorem}
在一个没有零因子的环里, 两个消去律都成立.
\begin{enumerate}
	\item $a \neq 0, ab = ac \Rightarrow b = c$
	\item $a \neq 0, ba = ca \Rightarrow b = c$
\end{enumerate}
反过来, 在一个环里如果\;\fbox{有一个}\;消去律成立,那么这个环没有零因子.
\end{Theorem}

\begin{Corollary}
在一个环$R$中如果有一个消去律成立,那么另一个消去律也成立.
\end{Corollary}

\begin{Definition}[整环]
一个环$R$叫做一个\textbf{整环}, 如果
\begin{enumerate}
	\item 乘法适合交换律: $ab = ba$.
	\item $R$有单位元1: $1a = a1 = a$.
	\item $R$没有零因子: $ab = 0 \Rightarrow a = 0$或$b = 0$
\end{enumerate}
\end{Definition}

\begin{Proposition}
整数环是一个整环.
\end{Proposition}

\end{document}
