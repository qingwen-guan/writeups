\documentclass[UTF8]{ctexart}
\usepackage{amsmath}
\usepackage{amssymb}
\usepackage{wasysym}
\usepackage{geometry}
\usepackage{ntheorem}
\usepackage{enumerate}
\usepackage[shortlabels]{enumitem} %https://tex.stackexchange.com/questions/229540/customized-enumerate-items

\geometry{a4paper,left=2cm,right=2cm,top=2cm,bottom=1cm}

\theorembodyfont{\upshape}
\newtheorem{Definition}{定义}%[subsection]
\newtheorem{Theorem}[Definition]{定理}
\newtheorem{Remark}[Definition]{注意}
\newtheorem{Lemma}[Definition]{引理}

\setenumerate[1]{itemsep=0pt,partopsep=0pt,parsep=\parskip,topsep=0pt}
\setitemize[1]{itemsep=0pt,partopsep=0pt,parsep=\parskip,topsep=0pt}

\begin{document}

\title{近世代数(抽象代数)笔记}
\author{管清文}
\maketitle
\tableofcontents

\section{基本概念}

\subsection{代数运算}
%\begin{Definition}[笛卡尔积] 令 $A_1, A_2, \cdots, A_n$是$n$个集合
%$$ A_1 \times A_2 \times \cdots \times A_n = \{ (a_1, a_2, \cdots, a_n) \mid a_i \in A_i, i = 1, 2, \cdots, n \} $$
%\end{Definition}

\begin{Remark}
近世代数(或抽象代数)的主要内容就是研究所谓\textbf{代数系统},即带有运算的集合。
\end{Remark}

\begin{Definition}[映射]
$$ \begin{aligned}
A_1 \times A_2 \times \cdots \times A_n &\rightarrow D \\
 (a_1, a_2, \cdots, a_n) &\mapsto d = \phi (a_1, a_2, \cdots, a_n) = \overline{(a_1, a_2, \cdots, a_n)} \end{aligned}$$ 
% \begin{itemize}
% 	\item 法则 $\phi$ 叫做集合 $A_1 \times A_2 \times \cdots \times A_n$ 到 集合 $D$ 的一个\textbf{映射}
% 	\item 元素 $d$ 叫做 元素  $ (a_1, a_2, \cdots, a_n) $ 在映射 $
% 	\phi$ 下的\textbf{象}
% 	\item 元素 $ (a_1, a_2, \cdots, a_n) $  叫做元素 $d$ 在 $\phi$下的 \textbf{原象}
% \end{itemize}
\end{Definition}

\begin{Remark}
% 	\begin{itemize}
 		%\item 
 		判断一个法则$\phi$是映射的充要条件: (i) 都有象 (ii) 象唯一.
% 	\end{itemize}
\end{Remark}

\begin{Definition}[代数运算] 
$$\begin{aligned}
A \times B &\rightarrow D \\ 
(a, b) &\mapsto d = \phi(a, b) = \circ (a, b) = a \circ b \end{aligned}$$
\end{Definition}

\begin{Remark}
% 	\begin{itemize}
 %		\item 
 $A = B$时, 对于代数运算$ A \times A \rightarrow D $, $ a \circ b $ 和 $ b \circ a $ 都有意义,但不一定相等.
\end{Remark}
% 	\end{itemize}

\begin{Definition}[$A$的代数运算, 二元运算] 
假如 $ \circ $ 是一个 $ A \times A \rightarrow A$的代数运算(即$A = B = D$),我们说集合$A$对于代数运算$\circ$来说是闭的, 也说, $\circ$是\textbf{$A$的代数运算}或\textbf{二元运算}.
\end{Definition}

\begin{Definition}[结合率]
我们说,一个集合$A$的代数运算$\circ$满足结合律,假如对于$A$的任何三个元素$a, b, c$来说都有$$
(a \circ b) \circ c = a \circ (b \circ c)
$$
\end{Definition}

\begin{Definition}
假如对于$A$的$n$ ($n \ge 2$)个固定的元素 $a_1, a_2, \cdots, a_n$来说,所有的加括号方式 $\pi(a_1 \circ a_2 \circ \cdots \circ a_n)$都相等,我们就把这些步骤可以得到的唯一的结果,用$a_1 \circ a_2 \circ \cdots \circ a_n $ 来表示.
\end{Definition}

\begin{Theorem}
若$A$的代数运算$\circ$满足结合律,则对于$A$的任意$n$($n \ge 2$)个元素 $a_1, a_2, \cdots, a_n$来说,对于任意的加括号的方法$\pi$, $\pi(a_1 \circ a_2 \circ \cdots \circ a_n)$ 都相等,$a_1 \circ a_2 \circ \cdots \circ a_n$ 也就总有意义.
\end{Theorem}

\begin{Definition}
$A$上的二元运算$\circ$, $a \circ b = b \circ a$\;(a与b可交换) $\forall a, b \in A$成立,则称$\circ$满足交换律. 
\end{Definition}

\begin{Theorem}
若$A$上的二元运算$\circ$满足结合律与交换律,则$a_1 \circ a_2 \circ \cdots \circ a_n$ 可以任意交换顺序.
\end{Theorem}

\begin{Definition}[分配率]
$ \astrosun $和$ \oplus $ 都是 $A$上的二元运算, 
\begin{enumerate}[i)]
\item 若
$ b \astrosun (a_1 \oplus a_2) = (b \astrosun a_1) \oplus (b \astrosun a_2), \forall b, a_1, a_2 $, 则称 $ \astrosun$和$ \oplus $  满足第一分配率.
\item 若
$ (a_1 \oplus a_2) \astrosun b = (a_1 \astrosun b) \oplus ( a_2 \astrosun b), \forall a_1, a_2, b $, 则称 $ \astrosun$和$\oplus $  满足第二分配率.
\end{enumerate}
\end{Definition}


\begin{Theorem}
若$A$上的二元运算$\oplus$ 满足结合律, $ \astrosun $和$\oplus $ 满足第一分配率,则
$$
b \astrosun ( a_1 \oplus a_2 \oplus \cdots \oplus a_n ) =  ( b \astrosun a_1) \oplus (b \astrosun a_2) \oplus \cdots \oplus (b \astrosun a_n)
$$
\end{Theorem}

\subsection{同态}

\begin{Definition}[满射]
映射$\phi: A \rightarrow \bar{A}$被称为\textbf{满射}, 如果
$\forall \hat{a} \in \bar{A}, \exists a \in A \text{ s.t. } \bar{a} = \hat{a}$. 
($\phi^{-1}$都有象)
\end{Definition}

\begin{Definition}[单射]
映射$\phi: A \rightarrow \bar{A}$被称为\textbf{单射}, 如果
$\forall a, b \in A, a \neq b \Rightarrow \bar{a} \neq \bar{b}$.
 ($\phi^{-1}$象唯一)
\end{Definition}

\begin{Definition}[一一映射]
既是满射又是单射.
\end{Definition}

\begin{Remark}[一一映射判别]
(i) 是映射(都有象、象唯一) (ii)满的 (iii) 单的.
\end{Remark}

\begin{Definition}[变换]
从$A$到$A$的映射 $\phi: A \rightarrow A$ 叫$A$上的变换.
\begin{itemize}
	\item 如果$\phi$是满的, 则称为\textbf{满变换}.
	\item 如果$\phi$是单的, 则称为\textbf{单变换}.
	\item 如果$\phi$是一一的, 则称为\textbf{一一变换}.
\end{itemize}
\end{Definition}

\begin{Definition}[同态映射]
对于$\phi: A \rightarrow \bar{A}$, $A$上有二元运算$\circ$, $\bar{A}$上有二元运算$\bar{\circ}$.
如果
$ \overline{a \circ b} = \bar{a} \, \bar{\circ} \, \bar{b}$, 则称 $\phi$是 $A$到 $\bar{A}$的同态映射.
\end{Definition}

\begin{Remark}[同态映射判别]
(i) 是映射(都有象、象唯一) (ii) $ \overline{a \circ b} = \bar{a} \,{\bar{\circ}}\, \bar{b}$
\end{Remark}

\begin{Definition}[同态满射、同态]
如果$A$到$\bar{A}\;${\fbox{存在}}\;一个同态映射$\phi$, 且它是满的, 则称$A$与$\bar{A}$\;(关于$\circ$与$\bar{\circ}$来说)\textbf{同态}. 称这个映射是一个\textbf{同态满射}.
\end{Definition}

\begin{Remark}[同态满射判别]
(i) 是映射(都有象、象唯一) (ii) 同态 (iii) 满
\end{Remark}


\begin{Definition}[同构映射、同构]
如果$A$到$\bar{A}\;${\fbox{存在}}\;一个同态映射$\phi$, 且它是既是满的又是单的(一一的), 则称$A$与$\bar{A}$(关于$\circ$与$\bar{\circ}$)\textbf{同构}, 记为$A \cong
 \bar{A}$. 称这个映射是一个(关于$\circ$与$\bar{\circ}$的)\textbf{同构映射}(简称\text{同构}).
\end{Definition}

\begin{Remark}[同构映射判别]
(i) 是映射(都有象、象唯一) (ii) 同态 (iii) 满 (iv) 单
\end{Remark}

\begin{Theorem}
假定对于代数运算$\circ$和$\bar{\circ}$来说, $A$与$\bar{A}$同态, 那么
\begin{enumerate}[i)]
\item 若 $\circ$ 满足结合律, $\bar{\circ}$也满足结合律;
\item 若 $\circ$ 满足交换律, $\bar{\circ}$也满足交换律.
\end{enumerate}
\end{Theorem}

\begin{Theorem}
$ \astrosun $和$ \oplus $ 是 $A$的两个代数运算, 
$ \bar{\astrosun} $和$\bar{\oplus} $ 是 $\bar{A}$的两个代数运算,
有
$\phi$既是$A$与$\bar{A}$的关于$ \astrosun $和$\bar{\astrosun}$ 的同态满射,
$\phi$也是$A$与$\bar{A}$的关于$ \oplus $和$\bar{\oplus}$ 的同态满射,
则 
\begin{enumerate}[i)]
\item 若$ \astrosun$和$ \oplus $  满足第一分配率, 则 $ \bar{\astrosun} $和$\bar{\oplus} $ 也满足第一分配率.
\item 若$ \astrosun$和$\oplus $  满足第二分配率, 则 $ \bar{\astrosun} $和$\bar{\oplus} $ 也满足第二分配率.
\end{enumerate}
\end{Theorem}

\begin{Remark}
总结下来, 如果$A$与$\bar{A}$同态,则若前者有什么算律(结合、交换、分配),后者就也有什么算律(结合、交换、分配).
\end{Remark}

\begin{Definition}[自同构]
对于$\circ$和$\circ$来说的一个$A$与$A$之间的\;\fbox{同构映射}\;叫做一个对于$\circ$来说的$A$的\textbf{自同构}.
\end{Definition}

\begin{Definition}[关系\mbox{[Relation]}]
$R: A \times A \rightarrow D = \{\text{对}, \text{错}\} $, 
若
$R(a, b) = \text{对}$
, 称
$(a, b)$
满足关系$R$, 记为$aRb$.
\end{Definition}

\begin{Definition}[等价关系]
如果$\sim$是$A$的元素间的关系,满足 
\begin{enumerate}[i)]
\item 自反性, $\forall a \in A, a \sim a$.
\item 对称性, $\forall a, b \in A$, 若$a \sim b$, 则$b \sim a$.
\item 传递性, $\forall a, b, c \in A$, 若$a \sim b$, $b\sim c$, 则$a \sim c$.
\end{enumerate}
则称$\sim$为等价关系.
\end{Definition}

\begin{Definition}[集合分类、划分]
集合$A$分成若干子集,满足 (i) 每个元素属于都某子集 (ii) 每个元素只属于某子集. 这些类的全体叫做\textbf{集合$A$的一个分类}.
$$ A = A_1 \cup A_2 \cup \cdots \cup A_n, A_i \cap A_j = \emptyset, i \neq j$$
\end{Definition}

\begin{Theorem}
集合上的一个分类,确定一个集合的元素之间的等价关系.
\end{Theorem}

\begin{Theorem}
集合上的一个等价关系,确定一个集合的分类.
\end{Theorem}

\begin{Definition}[模$n$的剩余类]
$ \{ [0], [1], \cdots, [n-1] \} $, $[i] = \{ k n + i \mid k \in \mathbb{Z} \}$
\end{Definition}


\section{群论}

\begin{Remark}
群是一个代数系统(定义代数运算的代数系统), 其中群里只有一个代数运算. 便利起见$\phi(a, b) = a \circ b$写成$a b$
\end{Remark}

\begin{Definition}[群\mbox{[Group]}的第一定义]
在集合$G \neq \emptyset$上规定一个叫做乘法的代数运算. 这个代数系统被称为群, 如果
\begin{itemize}
	\item[\uppercase\expandafter{\romannumeral1}] 乘法封闭, $\forall a, b \in G, ab \in G$ (由代数运算要求)
	\item[\uppercase\expandafter{\romannumeral2}] 乘法结合, $\forall a, b, c \in G, (ab)c = a(bc)$
	\item[\uppercase\expandafter{\romannumeral3}] $ \forall a, b \in G$, $ax = b, ya = b$在$G$中都有解.
\end{itemize}
\end{Definition}

\begin{Theorem}[IV\mbox{[左单位元]}]
对于群$G$中至少有一个元$e$, 叫做$G$的一个\textbf{左单位元},使得$\forall a \in G$都有 $ea = a$.
\end{Theorem}


\begin{Theorem}[V\mbox{[左逆元]}]
对于群$G$中的任何一个元素$a$, 在$G$中存在一个元$a^{-1}$,叫做$a$的\textbf{左逆元}, 能让$a^{-1} a = e$.
\end{Theorem}


\begin{Definition}[群\mbox{[Group]}的二定义]
在集合$G \neq \emptyset$上规定乘法. 这个代数系统被称为群, 如果
\begin{itemize}
	\item[\uppercase\expandafter{\romannumeral1}] 乘法封闭
	\item[\uppercase\expandafter{\romannumeral2}] 乘法结合
	\item[IV] $\exists e \in G$使 $ea =a$ 对 $\forall a \in G$都成立.
	\item[V]$\forall a \in G, \exists \, a^{-1}$使$a^{-1}a = e$.
\end{itemize}
\end{Definition}


%\begin{Lemma}
%群$G$中一个元素$a$的左逆元也是右逆元($a^{-1}a = e \Rightarrow a a^{-1} = e$).
%\end{Lemma}

%\begin{Lemma}
%群$G$中左单位元也是右单位元($ea = a \Rightarrow ae = a, \forall a \in G$).
%\end{Lemma}

\begin{Definition}[群的阶]
如果$|G|$有限, 称其为\textbf{有限群}, 称他的\textbf{阶}是$G$的元素个数. \\
如果$G$中有无穷多个元素, 称其为\textbf{无限群}, 称他的\textbf{阶}无限.
\end{Definition}



\begin{Definition}[交换群、Abel群]
群中交换律不一定成立, 如果乘法满足交换律($\forall a, b \in G, ab = ba$), 则称之为\textbf{交换群}(\textbf{Abel群})
\end{Definition}

\begin{Theorem}[单位元]
在一个群$G$里存在且只存在一个元$e$, 使得$ea = ae = a$对于$\forall a \in G$成立. 这个元素被称为群$G$的\textbf{单位元}.
\end{Theorem}

\begin{Theorem}[逆元]
对于群$G$的任意一个元素$a$来说, 有且只有一个元素$a^{-1}$, 
使 $a^{-1} a = a a^{-1} = e$. 这个元素被称为$a$的\textbf{逆元}, 或者简称\textbf{逆}.
\end{Theorem}

\begin{Definition}
规定$a^n = \underbrace{a a \cdots a}_{n\text{个}}, a^0 = e, a^{-n} = (a^{-1})^n, n \in \mathbb{Z}^{+}$
\end{Definition}

\begin{Theorem}
$ a^n a^m = a^{n+m}, (a^n)^m, n, m \in \mathbb{Z} $
\end{Theorem}

\begin{Definition}[元素的阶]
在一个群$G$中,使得$a^n = e$的最小正整数, 叫做$a$的\textbf{阶}. 若这样的$n$不存在, 称$a$是无穷阶的,或者叫$a$的阶是无穷.
\end{Definition}

\section{环}
\section{域}

\end{document}
